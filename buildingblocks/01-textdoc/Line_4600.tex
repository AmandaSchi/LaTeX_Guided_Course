% %%%%%%%%%%%%%%%%%%%%%%%%%%%%%%%%%%%%%%%%%%%%%%%%%%%%%%%%%%%%%%%%%%%%%%% %
%                                                                         %
% The Project Gutenberg EBook of Elliptic Functions, by Arthur L. Baker   %
%                                                                         %
% This eBook is for the use of anyone anywhere at no cost and with        %
% almost no restrictions whatsoever.  You may copy it, give it away or    %
% re-use it under the terms of the Project Gutenberg License included     %
% with this eBook or online at www.gutenberg.org                          %
%                                                                         %
%                                                                         %
% Title: Elliptic Functions                                               %
%        An Elementary Text-Book for Students of Mathematics              %
%                                                                         %
% Author: Arthur L. Baker                                                 %
%                                                                         %
% Release Date: January 25, 2010 [EBook #31076]                           %
%                                                                         %
% Language: English                                                       %
%                                                                         %
% Character set encoding: ISO-8859-1                                      %
%                                                                         %
% *** START OF THIS PROJECT GUTENBERG EBOOK ELLIPTIC FUNCTIONS ***        %
%                                                                         %
% %%%%%%%%%%%%%%%%%%%%%%%%%%%%%%%%%%%%%%%%%%%%%%%%%%%%%%%%%%%%%%%%%%%%%%% %

\def\ebook{31076}
%%%%%%%%%%%%%%%%%%%%%%%%%%%%%%%%%%%%%%%%%%%%%%%%%%%%%%%%%%%%%%%%%%%%%%
%%                                                                  %%
%% Packages and substitutions:                                      %%
%%                                                                  %%
%% book:     Required.                                              %%
%% inputenc: Standard DP encoding. Required.                        %%
%%                                                                  %%
%% fix-cm:   For larger title page fonts. Optional.                 %%
%% ifthen:   Logical conditionals. Required.                        %%
%%                                                                  %%
%% amsmath:  AMS mathematics enhancements. Required.                %%
%% amssymb:  Additional mathematical symbols. Required.             %%
%%                                                                  %%
%% alltt:    Fixed-width font environment. Required.                %%
%% array:    Enhanced tabular features. Required.                   %%
%%                                                                  %%
%% mathpazo: Postscript fonts with old style numerals. Required.    %%
%%                                                                  %%
%% footmisc: Extended footnote capabilities. Required.              %%
%% perpage:  Start footnote numbering on each page. Required.       %%
%%                                                                  %%
%% indentfirst: Indent first word of each sectional unit. Optional. %%
%% textcase: Apply \MakeUppercase (et al.) only to text, not math.  %%
%%           Required.                                              %%
%%                                                                  %%
%% calc:     Length calculations. Required.                         %%
%% soul:     Spaced text. Optional.                                 %%
%%                                                                  %%
%% fancyhdr: Enhanced running headers and footers. Required.        %%
%%                                                                  %%
%% graphicx: Standard interface for graphics inclusion. Required.   %%
%% wrapfig:  Illustrations surrounded by text. Required.            %%
%%                                                                  %%
%% geometry: Enhanced page layout package. Required.                %%
%% hyperref: Hypertext embellishments for pdf output. Required.     %%
%%                                                                  %%
%%                                                                  %%
%% Producer's Comments:                                             %%
%%                                                                  %%
%%   Changes are noted in this file in three ways.                  %%
%%   1. \DPnote{} for in-line `placeholder' notes.                  %%
%%   2. \DPtypo{}{} for typographical corrections, showing original %%
%%      and replacement text side-by-side.                          %%
%%   3. [** PP: Note]s for lengthier or stylistic comments.         %%
%%                                                                  %%
%%                                                                  %%
%% Compilation Flags:                                               %%
%%                                                                  %%
%%   The following behavior may be controlled by boolean flags.     %%
%%                                                                  %%
%%   ForPrinting (false by default):                                %%
%%   Compile a screen-optimized PDF file. Set to false for print-   %%
%%   optimized file (pages cropped, one-sided, blue hyperlinks).    %%
%%                                                                  %%
%%                                                                  %%
%% Things to Check:                                                 %%
%%                                                                  %%
%%                                                                  %%
%% Spellcheck: .................................. OK                %%
%% Smoothreading pool: ......................... yes                %%
%%                                                                  %%
%% lacheck: ..................................... OK                %%
%%   Numerous false positives from commented code                   %%
%%                                                                  %%
%% PDF pages: 147 (if ForPrinting set to false)                     %%
%% PDF page size: 5.25 x 6.25in (if ForPrinting set to false)       %%
%% PDF bookmarks: created, point to ToC entries                     %%
%% PDF document info: filled in                                     %%
%% Images: 4 pdf diagrams                                           %%
%%                                                                  %%
%% Summary of log file:                                             %%
%% * Five overfull hboxes (Widest is ~1.2pt overfull).              %%
%%                                                                  %%
%%                                                                  %%
%% Compile History:                                                 %%
%%                                                                  %%
%% January, 2010: adhere (Andrew D. Hwang)                          %%
%%              texlive2007, GNU/Linux                              %%
%%                                                                  %%
%% Command block:                                                   %%
%%                                                                  %%
%%     pdflatex x3 (Run pdflatex three times)                       %%
%%                                                                  %%
%%                                                                  %%
%% January 2010: pglatex.                                           %%
%%   Compile this project with:                                     %%
%%   pdflatex 31076-t.tex ..... THREE times                         %%
%%                                                                  %%
%%   pdfTeXk, Version 3.141592-1.40.3 (Web2C 7.5.6)                 %%
%%                                                                  %%
%%%%%%%%%%%%%%%%%%%%%%%%%%%%%%%%%%%%%%%%%%%%%%%%%%%%%%%%%%%%%%%%%%%%%%
\listfiles
\documentclass[12pt,leqno]{book}[2005/09/16]

%%%%%%%%%%%%%%%%%%%%%%%%%%%%% PACKAGES %%%%%%%%%%%%%%%%%%%%%%%%%%%%%%%
\usepackage[latin1]{inputenc}[2006/05/05]

\usepackage[osf]{mathpazo}[2005/04/12]

\usepackage{ifthen}[2001/05/26]  %% Logical conditionals

\usepackage{amsmath}[2000/07/18] %% Displayed equations
\usepackage{amssymb}[2002/01/22] %% and additional symbols

\usepackage{alltt}[1997/06/16]   %% boilerplate, credits, license

\usepackage{array}[2005/08/23]   %% extended array/tabular features

                                 %% extended footnote capabilities
\usepackage[symbol]{footmisc}[2005/03/17]
\usepackage{perpage}[2006/07/15]

\usepackage{graphicx}[1999/02/16]%% For diagrams
\usepackage{wrapfig}[2003/01/31] %% and wrapping text around them

\usepackage{indentfirst}[1995/11/23]
\usepackage{textcase}[2004/10/07]

\usepackage{calc}[2005/08/06]

\IfFileExists{fix-cm.sty}{%      %% For larger title page fonts
  \usepackage{fix-cm}[2006/03/24]%
  \newcommand{\MyHuge}{\fontsize{38}{48}\selectfont}%
}{% else
  \newcommand{\MyHuge}{\Huge}%
}

\IfFileExists{soul.sty}{%        %% For spaced publisher's line
  \usepackage{soul}[2003/11/17]
}{% else
  \newcommand{\so}[1]{#1}%
}

% for running heads
\usepackage{fancyhdr}

\newcommand{\Titleskip}[1]{#1\TmpLen}

%%%%%%%%%%%%%%%%%%%%%%%%%%%%%%%%%%%%%%%%%%%%%%%%%%%%%%%%%%%%%%%%%
%%%% Interlude:  Set up PRINTING (default) or SCREEN VIEWING %%%%
%%%%%%%%%%%%%%%%%%%%%%%%%%%%%%%%%%%%%%%%%%%%%%%%%%%%%%%%%%%%%%%%%

% ForPrinting=true (default)           false
% Asymmetric margins                   Symmetric margins
% Black hyperlinks                     Blue hyperlinks
% Start Preface, ToC, etc. recto       No blank verso pages
%
% Chapter-like ``Sections'' start both recto and verso in the scanned
% book. This behavior has been retained.
\newboolean{ForPrinting}

%% UNCOMMENT the next line for a PRINT-OPTIMIZED VERSION of the text %%
%\setboolean{ForPrinting}{true}

%% Initialize values to ForPrinting=false
\newcommand{\Margins}{hmarginratio=1:1}     % Symmetric margins
\newcommand{\HLinkColor}{blue}              % Hyperlink color
\newcommand{\PDFPageLayout}{SinglePage}
\newcommand{\TransNote}{Transcriber's Note}
\newcommand{\TransNoteCommon}{%
  This book was produced from images provided by the Cornell
  University Library: Historical Mathematics Monographs collection.
  \bigskip

  Minor typographical corrections and presentational changes have
  been made without comment. The calculations preceding
  equation~\Eqno{(15)} on \Pageref{page}{12} (page~12 of the original)
  have been re-formatted.
  \bigskip
}

\newcommand{\TransNoteText}{%
  \TransNoteCommon

  This PDF file is optimized for screen viewing, but may easily be
  recompiled for printing. Please see the preamble of the \LaTeX\
  source file for instructions.
}
%% Re-set if ForPrinting=true
\ifthenelse{\boolean{ForPrinting}}{%
  \renewcommand{\Margins}{hmarginratio=2:3} % Asymmetric margins
  \renewcommand{\HLinkColor}{black}         % Hyperlink color
  \renewcommand{\PDFPageLayout}{TwoPageRight}
  \renewcommand{\TransNote}{Transcriber's Note}
  \renewcommand{\TransNoteText}{%
    \TransNoteCommon

    This PDF file is optimized for printing, but may easily be
    recompiled for screen viewing. Please see the preamble of the
    \LaTeX\ source file for instructions.
  }
}{% If ForPrinting=false, don't skip to recto
  \renewcommand{\cleardoublepage}{\clearpage}
}
%%%%%%%%%%%%%%%%%%%%%%%%%%%%%%%%%%%%%%%%%%%%%%%%%%%%%%%%%%%%%%%%%
%%%%  End of PRINTING/SCREEN VIEWING code; back to packages  %%%%
%%%%%%%%%%%%%%%%%%%%%%%%%%%%%%%%%%%%%%%%%%%%%%%%%%%%%%%%%%%%%%%%%


%% Set page dimensions
%\usepackage[body={5.25in,8.4in},\Margins]{geometry}[2002/07/08]
% If text block size is changed, illustrations must be relocated,
% so print and screen versions use the same text block size.
\ifthenelse{\boolean{ForPrinting}}{%
  \setlength{\paperwidth}{8.5in}
  \setlength{\paperheight}{11in}
  \usepackage[body={5.25in,7in},\Margins]{geometry}[2002/07/08]
}{%
  \setlength{\paperwidth}{5.5in}
  \setlength{\paperheight}{7.333in}
  \raggedbottom
  \usepackage[body={5.25in,6.25in},\Margins,includeheadfoot]{geometry}[2002/07/08]
}

\providecommand{\ebook}{00000}    % Overridden during white-washing
\usepackage[pdftex,
  hyperref,
  hyperfootnotes=false,
  pdftitle={The Project Gutenberg eBook \#\ebook: Elliptic Functions},
  pdfauthor={Arthur L. Baker},
  pdfkeywords={Brenda Lewis, Andrew D. Hwang,
               Project Gutenberg Online Distributed Proofreading Team,
               Cornell University Historical Mathematical Monographs
               Collection},
  pdfstartview=Fit,    % default value
  pdfstartpage=1,      % default value
  pdfpagemode=UseNone, % default value
  bookmarks=true,      % default value
  linktocpage=false,   % default value
  pdfpagelayout=\PDFPageLayout,
  pdfdisplaydoctitle,
  pdfpagelabels=true,
  bookmarksopen=true,
  bookmarksopenlevel=1,
  colorlinks=true,
  linkcolor=\HLinkColor]{hyperref}[2007/02/07]

% Re-crop screen-formatted version, accommodating wide displays
\ifthenelse{\boolean{ForPrinting}}
  {}
%  {\hypersetup{pdfpagescrop= 85 80 527 765}}
  {\hypersetup{pdfpagescrop= 0 0 396 528}}


%%%% Fixed-width environment to format PG boilerplate %%%%
% 9.2pt leaves no overfull hbox at 80 char line width
\newenvironment{PGtext}{%
\begin{alltt}
\fontsize{9.2}{10.5}\ttfamily\selectfont}%
{\end{alltt}}

%% No hrule in page header
\renewcommand{\headrulewidth}{0pt}

%% Larger spacing in arrays
\newcommand{\ASTR}{1.5} % Default array stretch
\newcommand{\SMSTR}{1.2}% Small array stretch
\renewcommand{\arraystretch}{\ASTR}
\newcommand{\Dots}[1]{\hdotsfor[6]{#1}}

\newcommand{\NegMathSkip}{\vspace*{-1.8\baselineskip}}

% Top-level footnote numbers restart on each page
\MakePerPage{footnote}

\newcommand{\ToCFont}{\normalfont\small\scshape}
\newcommand{\Heading}{\normalfont\Large\scshape}
\newcommand{\mychapno}{}% Set to roman number of current chapter

\makeatletter
\renewcommand\l@chapter{\@dottedtocline{0}{0em}{3.5em}}
\renewcommand{\@dotsep}{24}
\makeatother

% Running heads
\newcommand{\SetRunningHeads}[1]{%
  \fancyhead{}
  \setlength{\headheight}{15pt}
  \thispagestyle{empty}
  \fancyhead[CE]{\textit{ELLIPTIC FUNCTIONS.}}
  \fancyhead[CO]{\textit{\MakeTextUppercase{#1}}}

  \ifthenelse{\boolean{ForPrinting}}
             {\fancyhead[RO,LE]{\thepage}}
             {\fancyhead[R]{\thepage}}
}

% ToC line for generic chapter
\newcommand{\SetContentsLine}[3]{%
  \addcontentsline{toc}{chapter}{%
    \protect\texorpdfstring{\protect\makebox[\TmpLen][r]{%
        \protect\ToCFont#2\protect\@.}
      \protect\ToCFont #3}{#2. #1}%
  }
}

%\Chapter[PDF name]{Number.}{Heading title}
\newcommand{\Chapter}[3][]{%
  \cleardoublepage
  \phantomsection
  \renewcommand{\mychapno}{#2}
  \label{chapter:#2}
  \section*{\centering\Heading%
    CHAPTER #2.\rule[-16pt]{0pt}{16pt}\break%
    \MakeTextUppercase{\large #3}}
  \ifthenelse{\equal{#2}{I}}{% ** Chapter{I} has no optional argument
    % Set \TmpLen to width of `Chap. I.' in ToC file
    \addtocontents{toc}{%
      \protect\settowidth{\protect\TmpLen}{\protect\ToCFont Chap.\ I.}
    }%
    \addcontentsline{toc}{chapter}{\protect\texorpdfstring{%
        \protect\ToCFont Chap.~I\protect\@. #3}{I. #3}}%
  }{%
    \ifthenelse{\equal{#1}{}}% ** Need to pass alt. title to texorpdfstring?
      {\SetContentsLine{#3}{#2}{#3}}{\SetContentsLine{#1}{#2}{#3}}%
  }%

  \SetRunningHeads{#3}
}

\newcommand{\IntroChapter}[2]{%
  \cleardoublepage
  \phantomsection
  \begin{center}
    \Heading\Huge \MakeTextUppercase{#2}\\
    \tb
  \end{center}
  \section*{\centering\Heading \MakeTextUppercase{#1}\break}
  \addtocontents{toc}{\protect\thispagestyle{empty}}
  \addtocontents{toc}{\protect\vspace*{-2\baselineskip}}
  \addtocontents{toc}{}
  \addtocontents{toc}{%
    \protect\hfill\protect\smash{\protect\scshape\protect\footnotesize page}
  }%
  \addtocontents{toc}{}
  \addtocontents{toc}{\protect\vspace*{-\baselineskip}}
  \addtocontents{toc}{}
  \addcontentsline{toc}{chapter}{%
    \texorpdfstring{\protect\ToCFont}{} Introductory Chapter.}%

  \SetRunningHeads{#1}
}

\newcommand{\Section}[1]
  {\subsection*{\centering\normalfont\textsc{\MakeTextUppercase{\small#1}}}}

\newenvironment{Remark}
  {\medskip\par\footnotesize\textsc{Note}.---}
  {\medskip\normalsize}

\DeclareMathOperator{\ac}{a. c.}
\DeclareMathOperator{\am}{am}
\DeclareMathOperator{\cn}{cn}
\DeclareMathOperator{\dn}{dn}
\DeclareMathOperator{\sn}{sn}
\DeclareMathOperator{\tn}{tn}

\newcommand{\sinP}{\operatorname{\PadTo[l]{\tan^{2}}{\sin}}}
\newcommand{\cosP}{\operatorname{\PadTo[l]{\tan^{2}}{\cos}}}

\DeclareMathSizes{12}{11}{9}{8}

\newcommand{\Prod}{\bigl[{\textstyle\prod}\bigr]}
\newcommand{\Prodlim}[1][1]{\bigl[{\textstyle\prod\limits^{#1}_n}\bigr]}

\newcommand{\Eta}{H}

\newcommand{\Input}[2][]
  {\ifthenelse{\equal{#1}{}}
  {\includegraphics{./images/#2.pdf}}
  {\includegraphics[width=#1]{./images/#2.pdf}}%
}

% Usage: \Example. (including the period)
\newcommand{\Example}[1]{\textsc{Example#1}\quad}

\newcommand{\primo}{1\textsuperscript{o}}
\newcommand{\secundo}{2\textsuperscript{o}}

\newcommand{\First}[1]{\textsc{#1}}

% For corrections.
\newcommand{\DPtypo}[2]{#2}
\newcommand{\DPnote}[1]{}

% \PadTo[#1]{#2}{#3} sets #3 in a box of width #2, aligned at #1 (default [c])
% Examples: \PadTo{feet per sec.}{\Ditto}, \PadTo{The value is}{2.}
\newlength{\TmpLen}
\newcommand{\PadTo}[3][c]{%
  \settowidth{\TmpLen}{$#2$}%
  \makebox[\TmpLen][#1]{$#3$}%
}

\newcommand{\tb}{\rule{1in}{0.5pt}}
\newcommand{\Z}{\phantom{0}}

\newcommand{\stretchyspace}{\spaceskip0.5em plus 0.25em minus 0.125em}

\DeclareInputText{176}{\ifmmode{{}^\circ}\else\textdegree\fi}
\DeclareInputText{183}{\ifmmode\cdot\else\textperiodcentered\fi}

% ToC formatting
\AtBeginDocument{\renewcommand{\contentsname}%
  {\centering\ToCFont\Large CONTENTS\\\tb}}

% Cross-referencing: anchors
\newcommand{\Pagelabel}[1]
  {\phantomsection\label{page:#1}}

\newcommand{\Figlabel}[1]
  {\phantomsection\label{fig:#1}}

\newcommand{\Tag}[2][]{%
  \phantomsection
  \ifthenelse{\equal{#1}{}}
             {\label{eqn:\mychapno.#2}}
             {\label{eqn:\mychapno.(#1)}}
  \tag*{\normalsize\ensuremath{#2}}
}

% and links
\newcommand{\Pageref}[2]
  {\hyperref[page:#2]{#1~\pageref{page:#2}}}

\newcommand{\Eqrefchap}{I}
\newcommand{\Eqrefname}{equation}

% \Eqref[text name]{Link text}{chapter}{(equation)}, e.g.
% \Eqref{equation}{}{(5)}, \Eqref[14sub1]{eq.}{III}{(14_1)}, etc.
\newcommand{\Eqref}[4][]{%
  \ifthenelse{\equal{#3}{}}{%
    \renewcommand{\Eqrefchap}{\mychapno}%
  }{%
    \renewcommand{\Eqrefchap}{#3}%
  }%
  \ifthenelse{\equal{#2}{}}{%
    \renewcommand{\Eqrefname}{}%
  }{%
    \renewcommand{\Eqrefname}{#2~}%
  }%
  \ifthenelse{\equal{#1}{}}{%
    \hyperref[eqn:\Eqrefchap.#4]{\Eqrefname\Eqno{#4}}%
  }{%
    \hyperref[eqn:\Eqrefchap.(#1)]{\Eqrefname\Eqno{#4}}%
  }%
}

\newcommand{\Eqno}[1]{\normalsize\ensuremath{#1}}

\newcommand{\Chapref}[2]{\hyperref[chapter:#2]{#1~#2}}


% DPalign, DPgather
\makeatletter
\providecommand\shortintertext\intertext
\newcount\DP@lign@no
\newtoks\DP@lignb@dy
\newif\ifDP@cr
\newif\ifbr@ce
\def\f@@zl@bar{\null}
\def\addto@DPbody#1{\global\DP@lignb@dy\@xp{\the\DP@lignb@dy#1}}
\def\parseb@dy#1{\ifx\f@@zl@bar#1\f@@zl@bar
    \addto@DPbody{{}}\let\@next\parseb@dy
  \else\ifx\end#1
    \let\@next\process@DPb@dy
    \ifDP@cr\else\addto@DPbody{\DPh@@kr&\DP@rint}\@xp\addto@DPbody\@xp{\@xp{\the\DP@lign@no}&}\fi
    \addto@DPbody{\end}
  \else\ifx\intertext#1
    \def\@next{\eat@command0}%
  \else\ifx\shortintertext#1
    \def\@next{\eat@command1}%
  \else\ifDP@cr\addto@DPbody{&\DP@lint}\@xp\addto@DPbody\@xp{\@xp{\the\DP@lign@no}&\DPh@@kl}
          \DP@crfalse\fi
    \ifx\begin#1\def\begin@stack{b}
      \let\@next\eat@environment
  \else\ifx\lintertext#1
    \let\@next\linter@text
  \else\ifx\rintertext#1
    \let\@next\rinter@text
  \else\ifx\\#1
    \addto@DPbody{\DPh@@kr&\DP@rint}\@xp\addto@DPbody\@xp{\@xp{\the\DP@lign@no}&\\}\DP@crtrue
    \global\advance\DP@lign@no\@ne
    \let\@next\parse@cr
  \else\check@braces#1!Q!Q!Q!\ifbr@ce\addto@DPbody{{#1}}\else
    \addto@DPbody{#1}\fi
    \let\@next\parseb@dy
  \fi\fi\fi\fi\fi\fi\fi\fi\@next}
\def\process@DPb@dy{\let\lintertext\@gobble\let\rintertext\@gobble
  \@xp\start@align\@xp\tw@\@xp\st@rredtrue\@xp\m@ne\the\DP@lignb@dy}
\def\linter@text#1{\@xp\DPlint\@xp{\the\DP@lign@no}{#1}\parseb@dy}
\def\rinter@text#1{\@xp\DPrint\@xp{\the\DP@lign@no}{#1}\parseb@dy}
\def\DPlint#1#2{\@xp\def\csname DP@lint:#1\endcsname{\text{#2}}}
\def\DPrint#1#2{\@xp\def\csname DP@rint:#1\endcsname{\text{#2}}}
\def\DP@lint#1{\ifbalancedlrint\@xp\ifx\csname
DP@lint:#1\endcsname\relax\phantom
  {\csname DP@rint:#1\endcsname}\else\csname DP@lint:#1\endcsname\fi
  \else\csname DP@lint:#1\endcsname\fi}
\def\DP@rint#1{\ifbalancedlrint\@xp\ifx\csname
DP@rint:#1\endcsname\relax\phantom
  {\csname DP@lint:#1\endcsname}\else\csname DP@rint:#1\endcsname\fi
  \else\csname DP@rint:#1\endcsname\fi}
\def\eat@command#1#2{\ifcase#1\addto@DPbody{\intertext{#2}}\or
  \addto@DPbody{\shortintertext{#2}}\fi\DP@crtrue
  \global\advance\DP@lign@no\@ne\parseb@dy}
\def\parse@cr{\new@ifnextchar*{\parse@crst}{\parse@crst{}}}
\def\parse@crst#1{\addto@DPbody{#1}\new@ifnextchar[{\parse@crb}{\parseb@dy}}
\def\parse@crb[#1]{\addto@DPbody{[#1]}\parseb@dy}
\def\check@braces#1#2!Q!Q!Q!{\def\dp@lignt@stm@cro{#2}\ifx
  \empty\dp@lignt@stm@cro\br@cefalse\else\br@cetrue\fi}
\def\eat@environment#1{\addto@DPbody{\begin{#1}}\begingroup
  \def\@currenvir{#1}\let\@next\digest@env\@next}
\def\digest@env#1\end#2{%
  \edef\begin@stack{\push@begins#1\begin\end \@xp\@gobble\begin@stack}%
  \ifx\@empty\begin@stack
    \@checkend{#2}
    \endgroup\let\@next\parseb@dy\fi
    \addto@DPbody{#1\end{#2}}
    \@next}
\def\lintertext{lint}\def\rintertext{rint}
\newif\ifbalancedlrint
\let\DPh@@kl\empty\let\DPh@@kr\empty
\def\DPg@therl{&\omit\hfil$\displaystyle}
\def\DPg@therr{$\hfil}

\newenvironment{DPalign*}[1][a]{%
  \if m#1\balancedlrintfalse\else\balancedlrinttrue\fi
  \global\DP@lign@no\z@\DP@crfalse
  \DP@lignb@dy{&\DP@lint0&}\parseb@dy
}{%
  \endalign
}
\newenvironment{DPgather*}[1][a]{%
  \if m#1\balancedlrintfalse\else\balancedlrinttrue\fi
  \global\DP@lign@no\z@\DP@crfalse
  \let\DPh@@kl\DPg@therl
  \let\DPh@@kr\DPg@therr
  \DP@lignb@dy{&\DP@lint0&\DPh@@kl}\parseb@dy
}{%
  \endalign
}
\makeatother

%%%%%%%%%%%%%%%%%%%%%%%% START OF DOCUMENT %%%%%%%%%%%%%%%%%%%%%%%%%%

\begin{document}

\pagestyle{empty}
\pagenumbering{Alph}

\phantomsection
\pdfbookmark[-1]{Front Matter}{Front Matter}

%%%% PG BOILERPLATE %%%%
\Pagelabel{PGBoilerplate}
\phantomsection
\pdfbookmark[0]{PG Boilerplate}{Project Gutenberg Boilerplate}

\begin{center}
\begin{minipage}{\textwidth}
\small
\begin{PGtext}
The Project Gutenberg EBook of Elliptic Functions, by Arthur L. Baker

This eBook is for the use of anyone anywhere at no cost and with
almost no restrictions whatsoever.  You may copy it, give it away or
re-use it under the terms of the Project Gutenberg License included
with this eBook or online at www.gutenberg.org


Title: Elliptic Functions
       An Elementary Text-Book for Students of Mathematics

Author: Arthur L. Baker

Release Date: January 25, 2010 [EBook #31076]

Language: English

Character set encoding: ISO-8859-1

*** START OF THIS PROJECT GUTENBERG EBOOK ELLIPTIC FUNCTIONS ***
\end{PGtext}
\end{minipage}
\end{center}

\clearpage


%%%% Credits and transcriber's note %%%%
\begin{center}
\begin{minipage}{\textwidth}
\begin{PGtext}
Produced by Andrew D. Hwang, Brenda Lewis and the Online
Distributed Proofreading Team at http://www.pgdp.net (This
file was produced from images from the Cornell University
Library: Historical Mathematics Monographs collection.)
\end{PGtext}
\end{minipage}
\end{center}
\vfill

\begin{minipage}{0.85\textwidth}
\small
\phantomsection
\pdfbookmark[0]{Transcriber's Note}{Transcriber's Note}
\subsection*{\centering\normalfont\scshape%
\normalsize\MakeLowercase{\TransNote}}%

\raggedright
\TransNoteText
\end{minipage}


%%%%%%%%%%%%%%%%%%%%%%%%%%% FRONT MATTER %%%%%%%%%%%%%%%%%%%%%%%%%%

%% -----File: 001.png---Folio xx-------
%[** Title page]

\frontmatter
\pagenumbering{roman}
\pagestyle{empty}

\enlargethispage{36pt}

% [** PP: Set line skip]
\ifthenelse{\boolean{ForPrinting}}{%
  \setlength{\TmpLen}{0.2in}%
}{%
  \setlength{\TmpLen}{0.125in}%
}

\begin{center}
\makebox[0pt][c]{\centering\MyHuge\scshape Elliptic Functions.}\\[\Titleskip{4}]
{\Large\scshape An Elementary Text-Book for \\[\Titleskip{1}]
Students of Mathematics.}\\[\Titleskip{4}]
%
{\footnotesize BY}\\[\Titleskip{1}]
%
{\LARGE\scshape ARTHUR~L. BAKER, C.E., Ph.D.,}\\
\makebox[0pt][c]{\centering\footnotesize\scshape
Professor of Mathematics in the Stevens School of the Stevens Institute of}\\
\makebox[0pt][c]{\centering\footnotesize\scshape
Technology, Hoboken, N.~J.; formerly Professor in the Pardee} \\
\makebox[0pt][c]{\centering\footnotesize\scshape
Scientific Department, Lafayette College, Easton, Pa.}\\[\Titleskip{4}]
%
\tb
\[
\vphantom{\Bigg|}\sin \am u = \frac{1}{\sqrt{k}} · \frac{\Eta (u)}{\Theta (u)}.
\]
\tb\\[\Titleskip{4}]
%
NEW YORK: \\[2pt]
{\large \so{JOHN WILEY \& SONS},} \\
{\scshape 53~East Tenth Street.} \\
1890.
\end{center}
\clearpage

%% -----File: 002.png---Folio xx-------
\null\vfill
\begin{center}
Copyright, 1890, \\
{\footnotesize BY} \\
Arthur L. Baker.
\end{center}
\vfill\vfill

\settowidth{\TmpLen}{444 \& 446 Pearl Street,}%
\noindent\parbox[t]{\TmpLen}{\centering
\textsc{Robert Drummond}, \\
\textit{Electrotyper}, \\
444 \& 446 Pearl Street, \\
New York.}
\hfill
\settowidth{\TmpLen}{326 Pearl Street,}%
\parbox[t]{\TmpLen}{\centering
\textsc{Ferris Bros}., \\
\textit{Printers}, \\
326 Pearl Street, \\
New York.}

\clearpage

%% -----File: 003.png---Folio xx-------

\begin{center}
{\Large PREFACE.}\\[8pt]
\tb\\[16pt]
\end{center}
\phantomsection
\pdfbookmark[0]{Preface}{Preface}

\First{In} the works of Abel, Euler, Jacobi, Legendre, and others,
the student of Mathematics has a most abundant supply of
material for the study of the subject of Elliptic Functions.

These works, however, are not accessible to the general
student, and, in addition to being very technical in their treatment
of the subject, are moreover in a foreign language.

It is in the hope of smoothing the road to this interesting
and increasingly important branch of Mathematics, and of
putting within reach of the English student a tolerably complete
outline of the subject, clothed in simple mathematical
language and methods, that the present work has been compiled.

New or original methods of treatment are not to be looked
for. The most that can be expected will be the simplifying of
methods and the reduction of them to such as will be intelligible
to the average student of Higher Mathematics.

I have endeavored throughout to use only such methods as
are familiar to the ordinary student of Calculus, avoiding those
methods of discussion dependent upon the properties of double
periodicity, and also those depending upon Functions of Complex
Variables. For the same reason I have not carried the
discussion of the $\Theta$~and~$\Eta$ functions further.
%% -----File: 004.png---Folio xx-------

Among the minor helps to simplicity is the use of zero
subscripts to indicate decreasing series in the Landen Transformation,
and of numerical subscripts to indicate increasing
series. I have adopted the notation of Gudermann, as being
more simple than that of Jacobi.

{\stretchyspace
I have made free use of the following works: \textsc{Jacobi's}
Fundamenta Nova Theoriæ Func.\ Ellip.; \textsc{Houel's} Calcul
Infinitésimal; \textsc{Legendre's} Traité des Fonctions Elliptiques;
\textsc{Durege's} Theorie der Elliptischen Functionen; \textsc{Hermite's}
Théorie des Fonctions Elliptiques; \textsc{Verhulst's} Théorie des
Functions Elliptiques; \textsc{Bertrand's} Calcul Intégral; \textsc{Laurent's}
Théorie des Fonctions Elliptiques; \textsc{Cayley's} Elliptic
Functions; \textsc{Byerly's} Integral Calculus; \textsc{Schlomilch's} Die
Höheren Analysis; \textsc{Briot et Bouquet's} Fonctions Elliptiques.}

I have refrained from any reference to the Gudermann or
Weierstrass functions as not within the scope of this work,
though the Gudermannians might have been interesting
examples of verification formulæ. The arithmetico-geometrical
mean, the march of the functions, and other interesting investigations
have been left out for want of room.

%% -----File: 005.png---Folio xx-------

\clearpage
\phantomsection
\pdfbookmark[0]{Contents}{Contents}%
\tableofcontents

\iffalse

CONTENTS.

PAGE

Introductory Chapter, 1

Chap. I. Elliptic Integrals, 4

II. Elliptic Functions, 16

III. Periodicity of the Functions, 22

IV. Landen's Transformation, 30

V. Complete Functions, 45

VI. Evaluation for \phi, 48

% [** PP: Next entry matches neither chapter title nor running head;
% using chapter title]
VII. Factorization of Elliptic Functions, 51

VIII. The \Theta Function, 66

IX. The \Theta and \Eta Functions, 69

X. Elliptic Integrals of the Second Order, 81

XI. Elliptic Integrals of the Third Order, 90

XII. Numerical Calculations, q, 94

XIII. Numerical Calculations, K, 98

XIV. Numerical Calculations, u, 102

XV. Numerical Calculations, \phi, 108

XVI. Numerical Calculations, E(k, \phi), 111

XVII. Applications, 115

\fi
%% -----File: 006.png---Folio xx-------
%[Blank Page]
%% -----File: 007.png---Folio 1-------

\mainmatter
\phantomsection
\pdfbookmark[-1]{Main Matter}{Main Matter}
\pagenumbering{arabic}
\pagestyle{fancy}
\fancyfoot{}

\IntroChapter{Introductory Chapter.\protect\footnotemark}{Elliptic Functions.}
\footnotetext{Condensed from an article by Rev.\ Henry Moseley, M.A., F.R.S., Prof.\ of Nat.\ Phil.\ and Ast., King's College, London.}

\First{The} first step taken in the theory of Elliptic Functions
was the determination of a relation between the amplitudes of
three functions of either order, such that there should exist an
algebraic relation between the three functions themselves of
which these were the amplitudes. It is one of the most remarkable
discoveries which science owes to Euler. In 1761 he
gave to the world the complete integration of an equation of
two terms, each an elliptic function of the first or second order,
not separately integrable.

This integration introduced an arbitrary constant in the
form of a third function, related to the first two by a given
equation between the amplitudes of the three.

In 1775 Landen, an English mathematician, published his
celebrated theorem showing that any arc of a hyperbola may
be measured by two arcs of an ellipse, an important element
of the theory of Elliptic Functions, but \emph{then} an isolated result.
The great problem of comparison of Elliptic Functions of different
moduli remained unsolved, though Euler, in a measure,
exhausted the comparison of functions of the same modulus.
It was completed in 1784 by Lagrange, and for the computation
%% -----File: 008.png---Folio 2-------
of numerical results leaves little to be desired. The value of
a function may be determined by it, in terms of increasing or
diminishing moduli, until at length it depends upon a function
having a modulus of zero, or unity.

For all practical purposes this was sufficient. The enormous
task of calculating tables was undertaken by Legendre.
His labors did not end here, however. There is none of the
discoveries of his predecessors which has not received some
perfection at his hands; and it was he who first supplied to
the whole that connection and arrangement which have made
it an independent science.

The theory of Elliptic Integrals remained at a standstill
from 1786, the year when Legendre took it up, until the year
1827, when the second volume of his Traité des Fonctions
Elliptiques appeared. Scarcely so, however, when there appeared
the researches of Jacobi, a Professor of Mathematics in
Königsberg, in the 123d~number of the Journal of Schumacher,
and those of Abel, Professor of Mathematics at Christiania, in
the 3d~number of Crelle's Journal for 1827.

These publications put the theory of Elliptic Functions
upon an entirely new basis. The researches of Jacobi have for
their principal object the development of that general relation
of functions of the first order having different moduli, of which
the scales of \DPtypo{Legrange}{Lagrange} and Legendre are particular cases.

It was to Abel that the idea first occurred of treating the
Elliptic Integral as a function of its amplitude. Proceeding
from this new point of view, he embraced in his speculations
all the principal results of Jacobi. Having undertaken to develop
the principle upon which rests the fundamental proposition
of Euler establishing an algebraic relation between three
functions which have the same moduli, dependent upon a certain
relation of their amplitudes, he has extended it from three
to an indefinite number of functions; and from Elliptic Functions
to an infinite number of other functions embraced under
an indefinite number of classes, of which that of Elliptic Functions
%% -----File: 009.png---Folio 3-------
is but one; and each class having a division analogous to
that of Elliptic Functions into three orders having common
properties.

The discovery of Abel is of infinite moment as presenting
the first step of approach towards a more complete theory of
the infinite class of ultra elliptic functions, destined probably
ere long to constitute one of the most important of the
branches of transcendental analysis, and to include among the
integrals of which it effects the solution some of those which at
present arrest the researches of the philosopher in the very
elements of physics.
%% -----File: 010.png---Folio 4-------


\Chapter{I}{Elliptic Integrals.}

\First{The} integration of irrational expressions of the form
\begin{gather*}
X\, dx \sqrt{A + Bx + Cx^{2}},\\
\intertext{or}
\frac{X\, dx}{\sqrt{A + Bx + Cx^{2}}},
\end{gather*}
$X$ being a rational function of~$x$, is fully illustrated in most elementary
works on Integral Calculus, and shown to depend upon
the transcendentals known as logarithms and circular functions,
which can be calculated by the proper logarithmic and trigonometric
tables.

When, however, we undertake to integrate irrational expressions
containing higher powers of~$x$ than the square, we meet
with insurmountable difficulties. This arises from the fact that
the integral sought depends upon a new set of transcendentals,
to which has been given the name of \emph{elliptic functions}, and
whose characteristics we will learn hereafter.

The name of Elliptic Integrals has been given to the simple
integral forms to which can be reduced all integrals of the form
\[
\Tag{(1)}
V = \int F(X, R)\, dx,
\]
where $F(X, R)$ designates a rational function of $x$~and~$R$, and
$R$~represents a radical of the form
\[
R = \sqrt{Ax^{4} + Bx^{3} + Cx^{2} + Dx + E},
\]
%% -----File: 011.png---Folio 5-------
where $A$,~$B$,~$C$, $D$,~$E$ indicate constant coefficients.

We will show presently that all cases of \Eqref{Eq.}{}{(1)} can be
reduced to the three typical forms
\[
\Tag{(2)}
\begin{aligned}
&\int_{0}^{x} \frac{dx}{\sqrt{(1 - x^{2})(1 - k^{2}x^{2})}}, \\
&\int_{0}^{x} \frac{x^{2}\, dx}{\sqrt{(1 - x^{2})(1 - k^{2}x^{2})}},\\
&\int_{0}^{x} \frac{dx}{(x^{2} + a) \sqrt{(1 - x^{2})(1 - k^{2}x^{2})}},
\end{aligned}
\]
which are called elliptic integrals of the first, second, and third
order.

Why they are called \emph{Elliptic} Integrals we will learn further
on. The transcendental functions which depend upon these
integrals, and which will be discussed in \Chapref{Chapter}{IV}, are called \emph{Elliptic Functions}.

The most general form of \Eqref{Eq.}{}{(1)} is
\[
\Tag{(3)}
V = \int \frac{A + BR}{C + DR}\, dx;
\]
where $A$,~$B$,~$C$, and~$D$ stand for rational integral functions of~$x$.

$\dfrac{A + BR}{C + DR}$ can be written
\begin{align*}
\frac{A + BR}{C + DR}
  &= \frac{AC - BDR^{2}}{C^{2} - D^{2}R^{2}}
   - \frac{(AD - CB)R^{2}}{C^{2} - D^{2}R^{2}} · \frac{1}{R}\\
  &= N - \frac{P}{R};
\end{align*}
%% -----File: 012.png---Folio 6-------
$N$~and~$P$ being rational integral functions of~$x$. Whence \Eqref{Eq.}{}{(3)}
becomes
\[
\Tag{(4)}
V = \int N\, dx - \int \frac{P\, dx}{R}.
\]

\Eqref{Eq.}{}{(4)} shows that the most general form of~$V$ can be made
to depend upon the expressions
\[
\Tag{(5)}
V' = \int \frac{P\, dx}{R},
\]
and
\[
\int N\, dx.
\]

This last form is rational, and needs no discussion here.

We can write
\begin{align*}
P &= \frac{G_{0} + G_{1}x + G_{2}x^{2} + \dotsb}
          {H_{0} + H_{1}x + H_{2}x^{2} + \dotsb} \\
  &= \frac{G_{0} + G_{2}x^{2} + G_{4}x^{4} + \dotsb + (G_{1} + G_{3}x^{2} + \dotsb)x}
          {H_{0} + H_{2}x^{2} + H_{4}x^{4} + \dotsb + (H_{1} + H_{3}x^{2} + \dotsb)x}.
\end{align*}
Multiplying both numerator and denominator by
\[
H_{0} + H_{2}x^{2} + H_{4}x^{4} + \dotsb - (H_{1} + H_{3}x^{2} + H_{5}x^{4} + \dotsb)x,
\]
we have a new \DPtypo{numerator}{denominator} which contains only powers of~$x^{2}$.
The result takes the following form:
\begin{align*}
P &= \frac{M_{0} + M_{2}x^{2} + M_{4}x^{4} + \dotsb
        + (M_{1} + M_{3}x^{2} + M_{5}x^{4} + \dotsb)x}
          {N_{0} + N_{2}x^{2} + N_{4}x^{4} + N_{6}x^{6} + \dotsb} \\
  &= \Phi(x^{2}) + \Psi(x^{2})·x.
\end{align*}

\Eqref{Equation}{}{(5)} thus becomes
\[
\Tag{(6)}
V' = \int \frac{\Phi(x^{2})\, dx}{R} + \int \frac{\Psi(x^{2}) · x · dx}{R}.
\]
%% -----File: 013.png---Folio 7-------

We shall see presently that $R$ can always be assumed to be
of the form
\[
\sqrt{(1 - x^{2})(1 - k^{2}x^{2})}.
\]

Therefore, putting $x^{2} = z$, the second integral in \Eqref{Eq.}{}{(6)}
takes the form
\[
\frac{1}{2} \int \frac{\Psi(z) · dz}{\sqrt{(1 - z)(1 - k^{2}z)}},
\]
which can be integrated by the well-known methods of Integral
Calculus, resulting in logarithmic and circular transcendentals.

There remains, therefore, only the form
\[
\int \frac{\Phi(x^{2})\, dx}{R}
\]
to be determined.

We will now show that $R$ can always be assumed to be in
the form
\[
\sqrt{(1 - x^{2})(1 - k^{2}x^{2})}.
\]
We have
\begin{align*}
R &= \sqrt{Ax^{4} + Bx^{3} + Cx^{2} + Dx + E} \\
  &= \sqrt{G(x - a)(x - b)(x - c)(x - d)},
\end{align*}
$a$,~$b$,~$c$, and~$d$ being the roots of the polynomial of the fourth
degree, and $G$~any number, real or imaginary, depending upon
the coefficients in the given polynomial.

Substituting in \Eqref{equation}{}{(1)}
\begin{align*}
x &= \frac{p + qy}{1 + y}, \\
\intertext{we have}
\Tag{(7)}
V &= \int \phi(y, \rho)\, dy,
\end{align*}
%% -----File: 014.png---Folio 8-------
$\rho$~designating the radical
\[
\rho = \sqrt{G[p - a + (q - a)y]
              [p - b + (q - b)y]
              [p - c + (q - c)y]\DPtypo{}{\dotsm}} \DPtypo{\dotsm}{\;}.
\]

In order that the odd powers of~$y$ under the radical may
disappear we must have their coefficients equal to zero; i.e.,
\begin{align*}
(p - a)(q - b) + (p - b)(q - a) &= 0, \\
(p - c)(q - d) + (p - d)(q - c) &= 0;
\end{align*}
whence
\begin{align*}
2pq - (p + q)(a + b) + 2ab &= 0, \\
2pq - (p + q)(c + d) + 2cd &= 0,
\end{align*}
and
\[
\Tag{(8)}
\left\{
\begin{aligned}
pq    &= \frac{ab(c + d) - cd(a + b)}{a + b - (c + d)}, \\
p + q &= \frac{2ab - 2cd}{a + b - (c + d)}.
\end{aligned}
\right.
\]
\Eqref{Equation}{}{(8)} shows that $p$ and~$q$ are real quantities, whether
the roots $a$,~$b$,~$c$, and~$d$ are real or imaginary; $a$,~$b$, and $c$,~$d$ being
the conjugate pairs.

Hence \Eqref{equation}{}{(1)} can always be reduced to the form of
\Eqref{equation}{}{(7)}, which contains only the second and fourth powers
of the variable.

This transformation seems to fail when $a + b - (c + d) = 0$;
but in that case we have
\[
R = \sqrt{G[x^{2} - (a + b)x + ab][x^{2} - (a + b)x + cd]},
\]
and substituting
\[
x = y - \frac{a + b}{2}
\]
will cause the odd powers of~$y$ to disappear as before.

If the radical should have the form
\[
\sqrt{G(x - a)(x - b)(x - c)},
\]
%% -----File: 015.png---Folio 9-------
placing $x = y^{2} + a$, we get
\begin{align*}
V    &= \int \phi(y, \rho)\, dy, \\
\rho &= \sqrt{G(y^{2} + a - b)(y^{2} + a - c)},
\end{align*}
$\phi$~designating a rational function of $y$~and~$\rho$.

Thus all integrals of the form contained in \Eqref{equation}{}{(1)}, in
which $R$~stands for a quadratic surd of the third or fourth
degree, can be reduced to the form
\[
\Tag{(9)}
V = \int \phi(x, R)\, dx,
\]
$R$~designating a radical of the form
\[
\sqrt{G(1 + mx^{2})(1 + nx^{2})},
\]
$m$~and~$n$ designating constants.

It is evident that if we put
\[
x' = x\sqrt{-m},\quad k^{2} = -\frac{n}{m},
\]
we can reduce the radical to the form
\[
\sqrt{(1 - x^{2})(1 - k^{2}x^{2})}.
\]

We shall see later on that the quantity~$k^{2}$, to which has
been given the name \emph{modulus}, can always be considered real
and less than unity.

Combining these results with \Eqref{equation}{}{(6)}, we see that the
integration of \Eqref{equation}{}{(1)} depends finally upon the integration
of the expression
\[
\Tag{(10)}
V'' = \int \frac{\phi(x^{2})\, dx}{\sqrt{(1 - x^{2})(1 - k^{2}x^{2})}}
  = \int \frac{\phi(x^{2})\, dx}{R}.
\]
%% -----File: 016.png---Folio 10-------

The most general form of~$\phi(x^{2})$ is
\begin{align*}
\phi(x^{2})
  &= \frac{M_{0} + M_{2}x^{2} + M_{4}x^{4} + \dotsb}
          {N_{0} + N_{2}x^{2} + N_{4}x^{4} + \dotsb} \\
  &= P_{0} + P_{2}x^{2} + P_{4}x^{4} + P_{6}x^{\DPtypo{2}{6}} + \dotsb \\
  &+ \sum \frac{L}{(x^{2} + a)^{n}}. %[** PP: Textstyle sums in original]
\end{align*}

Hence
\[
\Tag{(11)}
V'' = \sum P \int \frac{x^{2m}\, dx}{R}
    + \sum L \int \frac{dx}{(x^{2} + a)^{n}\, R}.
\]

But $\displaystyle\int \frac{x^{2m}\, dx}{R}$ depends upon $\displaystyle\int \frac{dx}{R}$ and $\displaystyle\int \frac{x^{2}\, dx}{R}$, which can
be shown as follows:

Differentiating $Rx^{2m-3}$, we have
\begin{align*}
d[x^{2m-3}R]
  &= d\left[x^{2m-3} \sqrt{\alpha + \beta x^{2} + \gamma x^{4}}\right] \\
  &= (2m - 3)x^{2m-4}\, dx \sqrt{\alpha + \beta x^{2} + \gamma x^{4}}
%\end{aligned} \\
   + \frac{x^{2m-3}(\beta x + 2\gamma x^{3})\, dx}
          {\sqrt{\alpha + \beta x^{2} + \gamma x^{4}}}.
\end{align*}
Integrating and collecting, we get
\begin{align*}
Rx^{2m-3}
  &= \begin{aligned}[t]
       (2m - 3)\alpha \int \frac{x^{2m-4}\, dx}{R}
    &+ (2m - 2)\beta  \int \frac{x^{2m-2}\, dx}{R} \\
    &+ (2m - 1)\gamma \int \frac{x^{2m}\, dx}{R}
    \end{aligned} \\
%
\Tag{(12)}
  &= \alpha' \int \frac{x^{2m-4}\, dx}{R}
   + \beta'  \int \frac{x^{2m-2}\, dx}{R}
   + \gamma' \int \frac{x^{2m}\, dx}{R}.
\end{align*}
%% -----File: 017.png---Folio 11-------
Whence we get, by taking $m=2$,
\[
\Tag{(13)}
Rx = \alpha \int \frac{dx}{R}
   + \beta  \int \frac{x^{2}\, dx}{R}
   + \gamma \int \frac{x^{4}\, dx}{R},
\]
which shows that the general expression $\displaystyle\int \frac{x^{2m}\, dx}{R}$ can be found
by successive calculations, when we are able to integrate the
expressions
\[
\int \frac{dx}{R}\quad \text{and}\quad \int \frac{x^{2}\, dx}{R},
\]
the first and second of \Eqref{equation}{}{(2)}.

We will now consider the second class of terms in \Eqref{eq.}{}{(11)},
viz., $\dfrac{L\, dx}{(x^{2} + a)^{n}\, R}$.

This second term is as follows:
\begin{align*}
\Tag{(14)}
\sum \int \frac{L}{(x^{2} + a)^{n}\, R}
   = \int \frac{A\, dx}{(x^{2} + a)^{n}\, R}
  &+ \int \frac{B\, dx}{(x^{2} + a)^{n-1}\, R} \\
  &+ \int \frac{C\, dx}{(x^{2} + a)^{n-2}\, R} + \dotsb
\end{align*}

Each of these terms can be shown to depend ultimately
upon terms of the form
\[
\frac{x^{2}\, dx}{R},\quad \frac{dx}{R},\quad \text{and}\quad \frac{dx}{(x^{2} + a)\, R}.
\]

The two former will be recognized as the two ultimate forms
already discussed, the first and second of \Eqref{equation}{}{(2)}. The
third is the third one of \Eqref{equation}{}{(2)}.

This dependence of \Eqref{equation}{}{(14)} can be shown as follows:
%% -----File: 018.png---Folio 12-------

We have
\begin{align*}
d\left[\frac{xR}{(x^{2} + a)^{n-1}}\right]
  &= \frac{(x^{2} + a)^{n-1} (x\, dR + R\, dx) - 2x^{2} R(n + 1)(x^{2} + a)^{n-2}\, dx}
          {(x^{2} + a)^{2n-2}} \\
  &= \frac{(x^{2} + a)(x\, dR + R\, dx) - 2x^{2} R(n - 1)\, dx}{(x^{2} + a)^{n}}.
\end{align*}

Substituting the value of
\[
R  = \sqrt{\alpha + \beta x^{2} + \gamma x^{4}}\quad\text{and}\quad
dR = (\beta x + 2 \gamma x^{3})\, \frac{dx}{R},
\]
we get
\Pagelabel{12}%
\begin{gather*}
\begin{aligned}
&d\left[\frac{xR}{(x^{2} + a)^{n-1}}\right] \\ %[** PP: Moving = to next line]
%
&= \frac{(x^{2} + a)(\beta x^{2} + 2 \gamma x^{4} + \alpha + \beta x^{2} + \gamma x^{4})
      - 2x^{2}(n-1)(\alpha + \beta x^{2} + \gamma x^{4})}
     {(x^{2} + a)^{n}} · \frac{dx}{R}
\end{aligned} \\
%
\begin{aligned}
&= \frac{\left\{\begin{aligned}
        \bigl(3\gamma - 2(n-1)\gamma\bigr) x^6
      &+ \bigl(2\beta + 3a\gamma - 2(n-1)\beta\bigr) x^4 \\
      &+ \bigl(2a\beta + \alpha - 2(n-1)\alpha\bigr) x^2
       + a\alpha\end{aligned}\right\}}
  {(x^{2} + a)^{n}} · \frac{dx}{R} \\
%
&= \frac{-(2n - 5)\gamma x^6
      + \bigl(\DPnote{[**A]} - (2n - 4)\beta + 3a\gamma\bigr) x^4
      + \bigl(\DPnote{[**B]} - (2n - 3)\alpha + \DPnote{[**C]} 2a\beta\bigr) x^2
      + a\alpha}
  {(x^{2} + a)^{n}} · \frac{dx}{R};
\end{aligned}
\end{gather*}
%[** PP: This display has been completely re-set. Minor modifications
% were required to maintain algebraic correctness:
% **A & **B: (+ inserted, matching ) immediately precedes x^4, x^2 resp.;
% **C: - typo corrected to +]
or, by substituting in the numerator $x^{2} = z - a$,
\[
= \frac{\left\{\begin{aligned}&- (2n - 5)\gamma z^3 \\
        &+ \bigl((2n - 5) 3a\gamma - (2n - 4)\beta + 3a\gamma\bigr) z^2 \\
        &+ \bigl(\DPnote{[**D]} - (2n - 5) 3a^2\gamma + (2n - 4) 2a\beta - 6a^2\gamma
                - (2n - 3)\alpha + 2a\beta\bigr) z \\
        &+ \bigl((2n - 5)a^3\gamma - (2n - 4) a^2\beta + 3a^3\gamma
                + (2n - 3)a\alpha - 2a^2\beta + a\alpha\bigr)\end{aligned}\right\}}{(x^{2} + a)^{n}} · \frac{dx}{R};
\]
%[**D: (+ inserted, matching ) immediately precedes z]}
%% -----File: 019.png---Folio 13-------
or, after resubstituting $z = x^2 + a$, and integrating,
\begin{align*}
\Tag{(15)}
\frac{xR}{(x^2 + a)^{n - 1}}
  &= -(2n - 5)\gamma \int \frac{dx}{(x^2 + a)^{n - 3} R} \\
  &\quad -(2n - 4)(\beta - 3a \gamma) \int \frac{dx}{(x^2 + a)^{n - 2} R}\\
  &\quad -(2n - 3)(3 a^2 \gamma - 2a \beta + \alpha) \int \frac{dx}{(x^2 + a)^{n - 1} R}\\
  &\quad +(2n - 2)(a^3 \gamma - a^2 \beta + a \alpha) \int \frac{dx}{(x^2 + a)^n R}.
\end{align*}
\begin{align*}
= \alpha_{1}  \int \frac{dx}{(x^2 + a)^{n - 3}R}
 + \beta_{1}  \int \frac{dx}{(x^2 + a)^{n - 2}R}
&+ \gamma_{1} \int \frac{dx}{(x^2 + a)^{n - 1}R}\\
&+ \delta_{1} \int \frac{dx}{(x^2 + a)^n R}.
\end{align*}
Making $n = 2$, we have
\begin{align*}
\Tag{(16)}
\frac{xR}{(x^2 + a)^{\DPtypo{-1}{1}}}
  = \alpha_{1} \int \frac{(x^2 + a)\, dx}{R}
   + \beta_{1} \int \frac{dx}{R}
 &+ \gamma_{1} \int \frac{dx}{(x^2 + a)R} \\
 &+ \delta_{1} \int \frac{dx}{(x^2 + a)^2 R}.
\end{align*}

\Eqref{Equation}{}{(16)} shows that
\[
\int \frac{dx}{(x^2 + a)^2 R}
\]
depends upon the three forms
\[
\int \frac{x^2\, dx}{R},\quad
\int \frac{dx}{R},\quad \text{and}\quad
\int \frac{dx}{(x^2 + a)R},
\]
%% -----File: 020.png---Folio 14-------
the three types of \Eqref{equation}{}{(2)}, and \Eqref{equation}{}{(15)} shows that
the general form
\[
\int \frac{dx}{(x^2 + a)^n R}
\]
depends ultimately upon the same three types.

We have now discussed every form which the general \Eqref{equation}{}{(1)}
can assume, and shown that they all depend ultimately
upon one or more of the three types contained in \Eqref{equation}{}{(2)}.

These three types are called the three Elliptic Integrals of
the first, second, and third kind, respectively.

Legendre puts $x = \sin \phi$, and reduces the three integrals
to the following forms:
\begin{align*}%[** PP: Aligning next three lines]
\Tag{(17)}
F(k, \phi) &=  \int_{0}^{\phi} \frac{d \phi}{\sqrt{1 - k^2 \sin^2 \phi}}; \\
&\quad \llap{$\dfrac{1}{k^2}$}
               \int_{0}^{\phi} \frac{d \phi}{\sqrt{1 - k^2 \sin^2 \phi}}
     - \frac{1}{k^2} \int_{0}^{\phi} \sqrt{1 - k^2 \sin^{2} \phi} · d \phi; \\
\Tag{(18)}
\varPi(n, k, \phi)
  &= \int_{0}^{\phi} \frac{d \phi}{(1 - n \sin^2 \phi) \sqrt{1 - k^2 \sin^2 \phi}};
\end{align*}
the first being Legendre's integral of the first kind; the form
\[
\Tag{(19)}
E(k, \phi) = \int_{0}^{\phi} \sqrt{1 - k^2 \sin^2 \phi} · d \phi
\]
being the integral of the second kind; and the third one being
the integral of the third kind.

The form of the integral of the second kind shows why they
are called Elliptic Integrals, the arc of an elliptic quadrant
being equal to
\[
a \int_{0}^{\frac{\pi}{2}} \sqrt{1 - e^2 \sin^2 \phi} · d\phi,
\]
$\phi$ being the complement of the eccentric angle.
%% -----File: 021.png---Folio 15-------

By easy substitutions, we get from \Eqref{Eqs.}{}{(17)},~\Eqref{}{}{(18)}, and~\Eqref{}{}{(19)}
the following solutions:
\setlength{\TmpLen}{1.5ex}%
\begin{align*}
\int_{0}^{\phi} \frac{\sin^2 \phi}{\Delta}\, d\phi
  &= \frac{F - E}{k^2}; \\[\TmpLen]
%
\int_{0}^{\phi} \frac{\cos^2 \phi}{\Delta}\, d\phi
  &= \frac{E - (1 - k^2)F}{k^2}; \\[\TmpLen]
%
\int_{0}^{\phi} \frac{\tan^2 \phi}{\Delta}\, d\phi
  &= \frac{\Delta \tan \phi - E}{1 - k^2}; \\[\TmpLen]
%
\int_{0}^{\phi} \frac{\sec^2 \phi}{\Delta}\, d\phi
  &= \frac{\Delta \tan \phi + (1 - k^2)F - E}{1 - k^2}; \\[\TmpLen]
%
\int_{0}^{\phi} \frac{1}{\Delta^3}\, d\phi
  &= \frac{1}{1 - k^2} \left(E - \frac{k^2 \sin\phi \cos\phi}{\Delta} \right); \\[\TmpLen]
%
\int_{0}^{\phi} \frac{\sin^2 \phi}{\Delta^3}\, d\phi
  &= \frac{1}{1 - k^2} \left( \frac{E - (1 - k^2)F}{k^2}
                            - \frac{\sin\phi \cos\phi}{\Delta} \right); \\[\TmpLen]
%
\int_{0}^{\phi} \frac{\cos^2 \phi}{\Delta^3}\, d\phi
  &= \frac{F - E}{k^2} + \frac{\sin\phi \cos\phi}{\Delta}.
\end{align*}
%% -----File: 022.png---Folio 16-------


\Chapter{II}{Elliptic Functions.}

\begin{DPgather*}
\lintertext{\indent\First{Let}}
u = \int_0^\phi \frac{d\phi}{\sqrt{1 - k^2 \sin^2 \phi}}.
\end{DPgather*}

$\phi$\footnotemark~is called the \emph{amplitude} corresponding to the \emph{argument}~$u$,
and is written
\footnotetext{Legendre.}
\[
\phi = \am (u, k) = \am u.
\]

The quantity~$k$ is called the \emph{modulus}, and the expression
$\sqrt{1 - k^2 \sin^2 \phi}$ is written\footnotemark[1]
\[
\sqrt{1 - k^2 \sin^2 \phi} = \Delta \am u = \Delta \phi,
\]
and is called the \emph{delta function} of the amplitude of~$u$, or \emph{delta
of~$\phi$}, or simply \emph{delta}~$\phi$.

$u$~can be written
\[
u = F(k, \phi).
\]

The following abbreviations are used:
\begin{align*}
\sin \phi &= \sin \am u = \sn\footnotemark u; \\
\cos \phi &= \cos \am u = \cn\footnotemark[2] u; \\
\Delta \phi &= \Delta \am u = \dn\footnotemark[2] u = \Delta u; \\
\tan \phi &= \tan \am u = \tn u.
\end{align*}
\footnotetext{\textit{Gudermann}, in his ``Theorie der Modularfunctionen'': Crelle's Journal,
  Bd.~18.}%

Let $\phi$ and~$\psi$ be any two arbitrary angles, and put
\begin{align*}
\phi &= \am u;\\
\psi &= \am \nu.\\
\end{align*}
%% -----File: 023.png---Folio 17-------

%[Illustration]
\begin{wrapfigure}{r}{1.25in}
  \Input[1in]{023a}
\end{wrapfigure}
In the spherical triangle~$ABC$ we have from
Trigonometry, $\DPtypo{c}{\mu}$~and~$C$ being constant,
\[
\frac{d\phi}{\cos B} + \frac{d\psi}{\cos A} = 0.
\]

Since $C$~and~$\DPtypo{c}{\mu}$ are constant, denoting by~$k$ an arbitrary constant,
we have
\[
\Tag{(1)}
\frac{\sin C}{\sin \mu} = k.
\]

But
\[
\sin A
  = \sin\psi \frac{\sin B}{\sin \phi}
  = \sin\psi \frac{\sin C}{\sin \mu}
  = k \sin\psi.
\]

Whence
\[
\cos A
  = \sqrt{1 - \sin^{2}A}
  = \sqrt{1 - k^{2} \sin^{2} \psi}.
\]

In the same manner
\[
\cos B
  = \sqrt{1 - \sin^{2}B}
  = \sqrt{1 - k^{2} \sin^{2} \phi}.
\]

Substituting these values, we get
\[
\Tag{(2)}
\frac{d\phi}{\sqrt{1 - k^{2} \sin^{2} \phi}} +
\frac{d\psi}{\sqrt{1 - k^{2} \sin^{2} \psi}} = 0.
\]

Integrating this, there results
\[
\Tag{(3)}
\int_0^\phi \frac{d\phi}{\sqrt{1 - \DPtypo{k_2}{k^2} \sin^{2}\phi}} +
\int_0^\psi \frac{d\psi}{\sqrt{1 - k^2 \sin^2\psi}} = \text{const}.
\]

When $\phi = 0$, we have $\psi = \mu$, and therefore the constant
must be of the form
\[
\int_0^\mu \frac{d\phi}{\sqrt{1 - k^2 \sin^2 \phi}},
\]
%% -----File: 024.png---Folio 18-------
whence
\[
\Tag{(4)}
\int_0^\phi \frac{d \phi}{\sqrt{1 - k^2 \sin^2 \phi}} +
\int_0^\psi \frac{d \psi}{\sqrt{1 - k^2 \sin^2 \psi}} =
\int_0^\mu  \frac{d \phi}{\sqrt{1 - k^2 \sin^2 \phi}},
\]
or
\[
u + \nu = m;
\]
and evidently the amplitudes $\phi$,~$\psi$, and~$\mu$ can be considered as
the three sides of a spherical triangle, and the relations between
the sides of this spherical triangle will be the same as those
between $\phi$,~$\psi$, and~$\mu$.

%[Illustration]
\begin{wrapfigure}{l}{1.125in}
  \Input{024a}
\end{wrapfigure}
But the sides of this triangle have imposed upon them the
condition
\[
\frac{\sin C}{\sin \mu} = k;
\]
and since $k < 1$, we must have $\mu > C$, which requires that one
of the angles of the triangle shall be obtuse and the other two
acute.

In the figure, let $C$~be an acute angle of the triangle~$ABC$,
and $PQ$~the equatorial great circle of which $C$~is the pole.

%[Illustration]
\ifthenelse{\boolean{ForPrinting}}{%
  \begin{wrapfigure}[11]{r}{2.125in}
    \vspace{-1.5\baselineskip}
    \hfill\Input[2in]{025a}
    \vspace{1.5\baselineskip}
  \end{wrapfigure}
}{%
  \begin{wrapfigure}{r}{2.125in}
    \hfill\Input[2in]{025a}
  \end{wrapfigure}
}
The arc~$PQ$ will be the measure of the angle~$C$.

Let $AG$~and~$AH$ be the arcs of two great
circles perpendicular respectively to $CQ$ and~$CP$.
They will of course be shorter than~$PQ$.
Hence $AB = \mu$ must intersect~$CQ$ in points
between $CG$ and~$HQ$, since $\mu > (C=PQ)$. In
any case either $A$~or~$B$ will be obtuse according
as $B$~falls between $QH$ or~$CG$ respectively; and
the other angle will be acute.

In the case where $C$~is an obtuse angle, it will be easily seen
that the angle at~$A$ must be acute, since the great circle~$AD$,
perpendicular to~$CP$, intersects~$PQ$ in~$D$, $PD$~being a quadrant.
The same remarks apply to the angle~$B$. Hence, in either
%% -----File: 025.png---Folio 19-------
case, one of the angles of the triangle is
obtuse and the other two are acute, as
a result of the condition
\[
\frac{\sin C }{\sin \mu} = k < 1.
\]

From Trigonometry we have
\[
\cos \mu = \cos \phi \cos \psi + \sin \phi \sin \psi \cos C;
\]
and since the angle~$C$ is obtuse,
\[
\cos C = - \sqrt{1- \sin^2C} = -\sqrt{1 - k^2 \sin^2 \mu},
\]
and
\[
\Tag{(5)}
\cos \mu = \cos \phi \cos \psi - \sin \phi \sin \psi \sqrt{1 - k^2 \sin^2 \mu},
\]
the relation sought.

The spherical triangle likewise gives the following relations
between the sides:
\[
\Tag[5st]{(5)^*}
\left\{
\begin{aligned}
\cos \phi &= \cos \mu \cos \psi + \sin \mu \sin \psi \sqrt{1 - k^2 \sin^2 \phi}; \\
\cos \psi &= \cos \mu \cos \phi + \sin \mu \sin \phi \sqrt{1 - k^2 \sin^2 \psi}.
\end{aligned}
\right.
\]

These give, by eliminating $\cos \mu$,
\[
\sin \mu = \frac{\cos^2 \psi - \cos^2 \phi}
                {\sin \phi \cos \psi \Delta \psi - \sin \psi \cos \phi \Delta \phi};
\]
which, after multiplying by the sum of the terms in the denominator
and substituting $\cos^2 = 1 - \sin^2$, can be written
\[
\sin \mu
  = \frac{(\sin^2 \phi - \sin^2 \psi)
          (\sin \phi \cos \psi \Delta \psi + \sin \psi \cos \phi \Delta \phi \DPtypo{}{)}}
         { \sin^2 \phi \cos^2 \psi \Delta^2 \psi - \sin^2 \psi \cos^2 \phi \Delta^2 \phi}.
\]

Since the denominator can be written
\begin{gather*}
(\sin^2 \phi - \sin^2 \psi)(1 - k^2 \sin^2 \phi \sin^2 \psi), \\
\Tag{(6)}
\sin \mu = \frac{\sin \phi \cos \psi \Delta \psi
               + \sin \psi \cos \phi \Delta \phi}
                {1 - k^2 \sin^2 \phi \sin^2 \psi}.
\end{gather*}

In a similar manner we get
\[
\Tag[6st]{(6)^*}
\left\{
\begin{aligned}
\cos \mu   &= \frac{\cos \phi \cos \psi - \sin \phi \sin \psi \Delta \phi \Delta \psi}
                   {1 - k^2 \sin^2 \phi \sin^2 \psi}; \\
%
\Delta \mu &= \frac{\Delta \phi \Delta \psi - k^2 \sin \phi \sin \psi \cos \phi \cos \psi}
                   {1 - k^2 \sin^2 \phi \sin^2 \psi}.
\end{aligned}
\right.
\]
%% -----File: 026.png---Folio 20-------

These equations can also be written as follows:
\[
\Tag{(7)}
\left\{
\makebox[\linewidth-40pt][l]{$\begin{aligned}
\sin \am (u±\nu)
  &= \frac{\sin \am u \cos \am \nu \Delta \am \nu
         ± \sin \am \nu \cos \am u \Delta \am u}
          {1 - k^{2} \sin^{2} \am u \sin^{2} \am \nu}; \\
%
\cos \am (u±\nu)
  &= \frac{\cos \am u \cos \am \nu
       \mp \sin \am u \sin \am \nu \Delta \am u \Delta \am \nu}
          {1 - k^{2} \sin^{2} \am u \sin^{2} \am \nu}; \\
%
\Delta \am (u±\nu)
  &= \frac{\Delta \am u \Delta \am \nu
       \mp k^{2} \sin \am u \sin \am \nu \cos \am u \cos \am \nu}
          {1 - k^{2} \sin^{2} \am u \sin^{2} \am \nu};
\end{aligned}$}
\right.
\]
or
\[
\Tag{(8)}
\left\{
\begin{aligned}
\sn (u±\nu)
  &= \frac{\sn u \cn \nu \dn \nu ± \sn \nu \cn u \dn u}
          {1 - k^{2} \sn^{2} u \sn^{2} \nu}; \\
%
\cn (u±\nu)
  &= \frac{\cn u \cn \nu \mp \sn u \sn \nu \dn u \dn \nu}
          {1 - k^{2} \sn^{2} u \sn ^{2} \nu}; \\
%
\dn (u±\nu)
  &= \frac{\dn u \dn \nu \mp k^{2} \sn u \sn \nu \cn u \cn \nu}
          {1 - k^{2} \sn^{2} u \sn^{2} \nu}.
\end{aligned}
\right.
\]

Making $u = \nu$, we get from the upper sign
\[
\Tag{(9)}
\left\{
\begin{aligned}
\sn 2u &= \frac{2 \sn u \cn u \dn u}{1 - k^{2} \sn^{4} u}; \\
\cn 2u &= \frac{\cn^{2}u - \sn^{2}u \dn^{2}u}{1 - k^{2} \sn^{4}u}
        = \frac{1 - 2\sn^{2}u + k^{2}\sn^{4}u}{1 - k^{2}\sn^{4}u}; \\
\dn 2u &= \frac{\dn^{2}u - k^{2}\sn^{2}u \cn^{2}u}{1 - k^{2}\sn^{4}u}
        = \frac{1 - 2k^{2}\sn^{2}u + k^{2}\sn^{4}u}{1 - k^{2}\sn^{4}u}.
\end{aligned}
\right.
\]

From these
\[
\Tag{(10)}
\left\{
\begin{aligned}
1 - \cn 2u &= \frac{2 \cn^{2} u \dn^{2} u}{1 - k^{2} \sn^{4} u}; \\
1 + \cn 2u &= \frac{2 \cn^{2} u}{1 - k^{2} \sn^{4} u}; \\
1 - \dn \PadTo[l]{2u}{u}  &= \frac{2k^{2} \sn^{2} u \cn^{2} u}{1 - k^{2} \sn^{4} u}; \\
1 + \dn \PadTo[l]{2u}{u}  &= \frac{2 \dn^{2} u}{1 - k^{2} \sn^{4} u};
\end{aligned}
\right.
\]
%% -----File: 027.png---Folio 21-------
and therefore
\[
\Tag{(11)}
\left\{
\begin{aligned}
\sn^{2} u &= \frac{1 - \cn 2u}{1 + \dn 2u}; \\
\cn^{2} u &= \frac{\dn 2u + \cn 2u}{1 + \dn 2u}; \\
\dn^{2} u &= \frac{1 - k^{2} + \dn 2u + k^{2}\cn 2u}{1 +\dn 2u};
\end{aligned}
\right.
\]
and by analogy
\[
\Tag{(12)}
\left\{
\begin{aligned}
\sn \dfrac{u}{2} &= \sqrt{\frac{1 - \cn u}{1 + \dn u}}; \\
\cn \dfrac{u}{2} &= \sqrt{\frac{\cn u + \dn u}{1 + \dn u}}; \\
\dn \dfrac{u}{2} &= \sqrt{\frac{1 - k^{2} + \dn u + k^{2} \cn u}{1 + \dn u}}.
\end{aligned}
\right.
\]

In \Eqref{equations}{}{(7)} making $u = \nu$, and taking the lower sign,
we have
\[
\Tag{(13)}
\left\{
\begin{aligned}
\sn 0 &= 0; \\
\cn 0 &= 1; \\
\dn 0 &= 1.
\end{aligned}
\right.
\]

Likewise, we get by making $u = 0$,
\[
\Tag{(14)}
\left\{
\begin{aligned}
\sn (-u) &= -\sn u; \\
\cn (-u) &= +\cn u; \\
\dn (-u) &= \dn u.
\end{aligned}
\right.
\]
%% -----File: 028.png---Folio 22-------


\Chapter{III}{Periodicity of the Functions.}

\First{When} the elliptic integral
\[
\int_{0}^{\phi} \frac{d \phi}{\sqrt{1 - k^{2} \sin^{2} \phi}}
\]
has for its amplitude~$\dfrac{\pi}{2}$, it is called, following the notation of
Legendre, the \emph{complete} function, and is indicated by~$K$, thus:
\[
K = \int_{0}^{\frac{\pi}{2}} \frac{d \phi}{\sqrt{1-k^{2} \sin^{2} \phi}}.
\]

When~$k$ becomes the complementary modulus,~$k'$, (see \Eqref{eq.}{IV}{\DPtypo{4}{(4)}},
Chap.\ IV,) the corresponding complete function is indicated by~$K'$,
thus:
\[
K' = \int_{0}^{\frac{\pi}{2}} \frac{d \phi}{\sqrt{1-k'^{2} \sin^{2} \phi}}.
\]

From these, evidently,
\[
\am (K, k) = \frac{\pi}{2},\qquad \am (K', k') = \frac{\pi}{2}.
\]
\[
\Tag{(1)}
\left\{
\begin{aligned}
\sn (K, k) &= 1; \\
\cn (K, k) &= 0; \\
\dn (K, k) &= k'.
\end{aligned}
\right.
\]
%% -----File: 029.png---Folio 23-------

From \Eqref{eqs.}{II}{(7)},~\Eqref{}{II}{(8)}, and~\Eqref{}{II}{(9)}, Chap.~II, we have, by the substitution
of the values of~$\sn (K) = 1$, $\cn (K) = 0$, $\dn (K) = k'$,
\[
\Tag{(2)}
\left\{
\begin{aligned}
\sn 2K &= 0; \\
\cn 2K &= -1; \\
\dn 2K &= 1.
\end{aligned}
\right.
\]

These equations, by means of \Eqref{}{II}{(1)},~\Eqref{}{II}{(2)}, and~\Eqref{}{II}{(3)} of Chap.~II,
give
\[
\Tag{(3)}
\left\{
\begin{aligned}
\sn (u+2K) &= - \sn u; \\
\cn (u+2K) &= -\cn u; \\
\dn (u+2K) &= \dn u;
\end{aligned}
\right.
\]
and these, by changing $u$ into~$u+2K$, give
\[
\Tag{(4)}
\left\{
\begin{aligned}
\sn (u+4K) &= \sn u; \\
\cn (u+4K) &= \cn u; \\
\dn (u+4K) &= \dn u.
\end{aligned}
\right.
\]

From these equations it is seen that the elliptic functions
$\sn$,~$\cn$,~$\dn$, are periodic functions having for their period~$4K$.
Unlike the period of trigonometric functions, this period is not
a fixed one, but depends upon the value of~$k$, the modulus.

From the Integral Calculus we have
\begin{align*}
\int_{0}^{n \frac{\pi}{2}} \frac{d \phi}{\Delta \phi}
  &= \int_{0}^{\frac{\pi}{2}} \frac{d \phi}{\Delta \phi}
   + \int_{\frac{\pi}{2}}^{\pi} \frac{d \phi}{\Delta \phi}
   + \int_{\pi}^{\frac{3\pi}{2}} \frac{d \phi}{\Delta \phi} + \cdots
   + \int_{(n-1)\frac{\pi}{2}}^{n \frac{\pi}{2}} \frac{d \phi}{\Delta \phi} \\
  &= n \int_{0}^{\frac{\pi}{2}} \frac{d \phi}{\Delta \phi}
   = nK;
\end{align*}
from which we see that
\begin{DPalign*}
n \frac{\pi}{2} &= \am (nK); \\
%% -----File: 030.png---Folio 24-------
\intertext{or, since $\dfrac{\pi}{2} = \am K$,}
\am (nK) &= n · \am K, \\
\intertext{and}
n \pi &= \am (2nK), \\
\lintertext{and also}
n \pi &= 2n \am K.
\end{DPalign*}
In the case of an Elliptic Integral with the arbitrary angle~$\alpha$,
we can put
\[
\DPtypo{d}{\alpha} = n \pi ± \beta,
\]
where $\beta$ is an angle between $0$~and~$\dfrac{\pi}{2}$, the upper or the lower
sign being taken according as $\dfrac{\pi}{2}$ is contained in~$\alpha$ an even or
an uneven number of times.

In the first case we have
\[
\int_0^{n\pi+\beta} \frac{d \phi}{\Delta \phi}
  = \int_0^{n\pi} \frac{d \phi}{\Delta \phi}
  + \int_{n\pi}^{n\pi+\beta} \frac{d \phi}{\Delta \phi};
\]
or, putting $\phi_{1} = \DPtypo{}{\phi} - n \pi$,
\[
\int_0^{n\pi+\beta} \frac{d \phi}{\Delta \phi}
  = 2nK + \int_0^\beta \frac{d \phi_{1}}{\Delta \phi_{1}}.
\]

In the second case
\[
\int_0^{n\pi-\beta} \frac{d \phi}{\Delta \phi}
  = \int_0^{n\pi} \frac{d \phi}{\Delta \phi}
  - \int_{n\pi-\beta}^{n\pi} \frac{d \phi}{\Delta \phi};
\]
or, putting $\phi_{1} = n \pi - \phi$,
\[
\int_0^{n\pi-\beta} \frac{d \phi}{\Delta \phi}
  = 2nK - \int_0^\beta \frac{d \phi_{1}}{\Delta \phi_{1}};
\]
%% -----File: 031.png---Folio 25-------
or in either case,
\[
\int_0^{n\pi±\beta} \frac{d \phi}{ \Delta \phi}
  = 2nK ± \int_0^\beta \frac{d \phi_{1} }{ \Delta \phi_{1}}.
\]

Thus we see that the Integral with the general amplitude~$\alpha$
can be made to depend upon the complete integral~$K$ and
an Integral whose amplitude lies between $0$~and~$\dfrac{\pi}{2}$.

Put now
\[
\int_0^\beta \frac{d \phi_{1} }{ \Delta \phi_1} = u,\qquad \beta = \am u.
\]

This gives
\begin{DPalign*}
\int_0^{n\pi±\beta} \frac{d \phi }{ \Delta \phi} &= 2nK ± u,\\
\lintertext{or}
\am (2nK ± u)
  &= n \pi ± \beta\\
\Tag{(5)}
  &= n\pi ± \am u\\
\Tag{(6)}
  &= 2n · \am K ± \am u;\\
\lintertext{or, since}
\am (-z)
  &= -\am z,\\
\am (u ± 2nK)
  &= \am u ± n \pi\\
  &= \am u ± 2n · \am K.\\
\end{DPalign*}

Taking the sine and cosine of both sides, we have
\begin{align*}
\sn (u + 2nK) &= ± \sn u;\\
\cn (u + 2nK) &= ± \cn u;
\end{align*}
the upper or the lower sign being taken according as $n$~is even
or odd. By giving the proper values to~$n$ we can get the same
results as in \Eqref{equations}{}{(3)} and~\Eqref{}{}{(4)}.

Putting $n = 1$ in \Eqref{eq.}{}{(5)}, we have
\begin{align*}
\sn (2K - u)
  &= \sin \pi \cn u - \cos \pi \sn u\\
\Tag{(7)}
  &= \sn u.
\end{align*}
%% -----File: 032.png---Folio 26-------

Elliptic functions also have an imaginary period. In order
to show this we will, in the integral
\[
\int_0^\phi \frac{d \phi}{\Delta \phi},
\]
assume the amplitude as imaginary. Put
\[
\sin \phi = i \tan \psi.
\]
From this we get
\[
\Tag{(8)}
\left\{
\begin{aligned}
\cos \phi
  &= \frac{1}{\cos \psi}; \\
\Delta \phi
  &= \frac{\sqrt{1 - k'^{2} \sin^{2} \psi}}{\cos \psi}
   = \frac{\Delta (\psi, k')}{\cos \psi}; \\
d \phi
  &= i \frac{d \psi}{\cos \psi}.
\end{aligned}
\right.
\]
From these, since $\phi$~and~$\psi$ vanish simultaneously, we easily get
\begin{align*}
\int_0^\phi \frac{d \phi}{\Delta \phi}
  &= i \int_0^\psi \frac{d \psi}{\Delta (\psi, k')}. \\
\intertext{Put}
\int_0^\psi \frac{d \psi}{\Delta (\psi, k')} &= u \quad \text{and} \quad
\psi = \am (u, k'), \\
\intertext{whence}
\int_0^\phi \frac{d \phi}{\Delta \phi} &= iu \quad \text{and} \quad
\phi = \am (iu);
\end{align*}
and these substituted in \Eqref{Eq.}{}{(8)} give
\[
\Tag{(9)}
\left\{
\begin{aligned}
\sn iu &= i \tn (u, k'); \\
\cn iu &= \frac{1}{\cn (u, k')}; \\
\dn iu &= \frac{\dn (u, k')}{\cn (u, k')}.
\end{aligned}
\right.
\]
%% -----File: 033.png---Folio 27-------
By assuming
\[
\int_0^\psi \frac{d \psi}{\Delta (\psi, k')} = iu \quad \text{and}\quad
\int_0^\phi \frac{d \phi}{\Delta \phi} = -u,
\]
we get
\begin{align*}
\sn (-u) &= i \tn (iu, k'), \\
\cn (-u) &= \frac{1}{\cn(iu, k')}, \\
\dn (-u) &= \frac{\dn (iu, k')}{\cn (iu, k')};
\end{align*}
or, from \Eqref{eq.}{II}{(14)}, Chap.~II,
\[
\Tag{(10)}
\left\{
\begin{aligned}
\sn u &= -i \tn (iu, k'); \\
\cn u &= \frac{1}{\cn (iu, k')}; \\
\dn u &= \frac{\dn (iu, k')}{\cn (iu, k')}.
\end{aligned}
\right.
\]

From \Eqref{eqs.}{II}{(7)}, Chap.~II, making $\nu = K$, we get, since
$\sn K = 1$, $\cn K = 0$, $\dn K = k'$,
\[
\Tag{(11)}
\left\{
\begin{aligned}%[** PP: Moved \mp out of numerator in second equation]
\sn (u ± K) &= ± \frac{\cn u \dn u}{1 - k^{2} \sn^{2} u} = ± \dfrac{\cn u}{\dn u}; \\
\cn (u ± K) &= \mp \frac{\sn u \dn uk'}{\dn^{2} u} = \mp \frac{k' \sn u}{\dn u}; \\
\dn (u ± K) &= + \frac{k'}{\dn u}.
\end{aligned}
\right.
\]

In these equations, changing $u$ into~$iu$, we get, by means of
\Eqref{eqs.}{}{(9)},
\[
\Tag{(12)}
\left\{
\begin{aligned}
\sn (iu ± K) &= ± \frac{1}{\dn (u, k')}; \\
\cn (iu ± K) &= \mp \frac{ik' \sn (u, k')}{\dn (u, k')}; \\
\dn (iu ± K) &= + \frac{k' \cn (u, k')}{\dn (u, k')}.
\end{aligned}
\right.
\]
%% -----File: 034.png---Folio 28-------

Putting now in \Eqref{eqs.}{}{(9)} $u ± K'$ instead of~$u$, and making use
of \Eqref{eqs.}{}{(10)}, and interchanging $k$~and~$k'$, we have
\[
\Tag{(13)}
\left\{
\begin{aligned}
\sn (iu ± iK') &= - \frac{i \cn (u, k')}{k \sn (u, k')}; \\
\cn (iu ± iK') &= \mp \frac{\dn (u, k')}{k \sn (u, k')}; \\
\dn (iu ± iK') &= \mp \frac{1}{\sn (u, k')}.
\end{aligned}
\right.
\]

Substituting in these~$-iu$ in place of~$u$, we get, by
means of \Eqref{eqs.}{II}{(9)} and \Eqref{eqs.}{II}{(14)} of Chap.~II,
\[
\Tag{(14)}
\left\{
\begin{aligned}
\sn (u ± iK') &= \frac{1}{k \sn u}; \\
\cn (u ± iK') &= \mp \frac{i \dn u}{k \sn u}; \\
\dn (u ± iK') &= \mp i \cot \am u.
\end{aligned}
\right.
\]

In these equations, putting $u + K$ in place of~$u$, we get
\[
\Tag{(15)}
\left\{
\begin{aligned}
\sn (u + K ± iK') &= + \frac{\dn u}{k \cn u}; \\
\cn (u + K ± iK') &= \mp \frac{ik'}{k \cn u}; \\
\dn (u + K ± iK') &= ± ik' \tn u.
\end{aligned}
\right.
\]
Whence for $u = 0$ we get
\[
\Tag{(16)}
\left\{
\begin{aligned}
\sn (K ± iK') &= \frac{1}{k}; \\
\cn (K ± iK') &= \mp \frac{ik'}{k}; \\
\dn (K ± iK') &= 0.
\end{aligned}
\right.
\]
%% -----File: 035.png---Folio 29-------

If in \Eqref{eqs.}{}{(14)} we put $u = 0$, we see that as $u$~approaches
zero, the expressions
\Pagelabel{29}
\[
\sn (± iK'), \quad \cn (± iK'), \quad \dn (± iK')
\]
approach infinity.

We see from what has preceded that Elliptic Functions
have two periods, one a real period, and one an imaginary
period.

In the former characteristic they resemble Trigonometric
Functions, and in the latter Logarithmic Functions.

On account of these two periods they are often called
Doubly Periodic Functions. Some authors make this double
periodicity the starting-point of their investigations. This
method of investigation gives some very beautiful results and
processes, but not of a kind adapted for an elementary work.

It will be noticed that the Elliptic Functions $\sn u$,~$\cn u$, and~$\dn u$
have a very close analogy to trigonometric functions, in
which, however, the independent variable~$u$ is not an angle, as
it is in the case of trigonometric functions.

Like Trigonometric Functions, these Elliptic Functions can
be arranged in tables. These tables, however, require a double
argument, viz., $u$~and~$k$. In \Chapref{Chap.}{IX} these functions are developed
into series, from which their values may be computed
and tables formed.

No complete tables have yet been published, though they
are in process of computation.
%% -----File: 036.png---Folio 30-------


\Chapter{IV}{Landen's Transformation}

%[Illustration]
\begin{wrapfigure}[6]{l}{2in}
  \vspace{-\baselineskip}
  \Input[2in]{036a}
  \vspace{\baselineskip}
  \Pagelabel{30}%
\end{wrapfigure}
\First{Let} $AB$~be the diameter of a circle,
with the centre at~$O$, the radius $AO = r$,
and $C$~a fixed point situated upon~$OB$,
and $OC = k_{0}r$. Denote the angle~$PBC$
by~$\phi$, and the angle~$PCO$ by~$\phi_{1}$. Let
$P'$~be a point indefinitely near to~$P$.

Then
\[
\frac{PP'}{PC} = \frac{\sin PCP'}{\sin PP'C} = \frac{\sin PCP'}{\cos OP'C}.
\]

But $PP' = 2r\, d\phi$, and $\sin PCP' = PCP' = d \phi_{1}$;
therefore
\[
\frac{2r\, d\phi}{PC} = \frac{d \phi_{1}}{\cos OP'C}.
\]

But
\begin{align*}
\overline{PC}^{2}
  &= r^{2} + r^{2}k_{0}^{2} + 2r^{2}k_{0} \cos 2 \phi \\
  &= (r + rk_{0})^{2} \cos^{2} \phi + (r - rk_{0})^{2} \sin^{2} \phi;
\end{align*}
\NegMathSkip
\begin{DPalign*}
\lintertext{also}
r^{2} \cos^{2} OP'C
  &= r^{2} - r^{2} \sin^{2} OP'C \\
  &= r^{2} - r^{2}k_{0}^{2} \sin^{2} \phi_{1}.
\end{DPalign*}

Therefore %[** PP: Next two equations aligned in original]
\[
\frac{2\, d\phi}{\sqrt{(r + rk_{0})^{2} \cos^{2} \phi + \DPtypo{(r - rk_{0})}{(r - rk_{0})^{2}} \sin^{2} \phi}}
  = \frac{d \phi_{1}}{\sqrt{r^{2} - r^{2}k_{0}^{2} \sin^{2} \phi_{1}}},
\]
which can be written
\[
\frac{2}{r + rk_{0}}\,
  \frac{d \phi}{\sqrt{1 - \dfrac{4k_{0}r^{2}}{(r + rk_{0})^{2}} \sin^{2} \phi}}
  = \frac{1}{r}\, \frac{d \phi_{1}}{\sqrt{1 - k_{0}^{2} \sin^{2} \phi_{1}}},
\]
%% -----File: 037.png---Folio 31-------

Putting
\[
\Tag{(1)}
\frac{4k_{0}r^{2}}{(r + rk_{0})^{2}} = \frac{4k_{0}}{(1 + k_{0})^{2}} = k^{2},
\]
we have
\[
\Tag{(2)}
\int_{0}^{\phi} \frac{d \phi}{\sqrt{1 - k^{2} \sin^{2} \phi}}
  = \frac{1 + k_{0}}{2}
    \int_{0}^{\phi_{1}} \frac{d \phi_{1}}{\sqrt{1 - k_{0}^{2} \sin^{2} \phi_{1}}};
\]
no constant being added because $\phi$ and~$\phi_{1}$ vanish simultaneously;
$\phi$ and~$\phi_{1}$ being connected by the equation
\[
\Tag{(3)}
\frac{\sin OPC}{\sin OCP}
  = \frac{\sin (2 \phi - \phi_{1})}{\sin \phi_{1}}
  = \frac{rk_{0}}{r}
  = k_{0}.
\]

From the value of~$k^{2}$ we have
\[
\Tag{(4)}
1 - k^{2} = k'^{2} = \frac{(1 - k_{0})^{2}}{(1 + k_{0})^{2}},
\]
and therefore
\[
\Tag{(5)}
k_{0} = \frac{1 - k'}{1 + k'}.
\]
$k'$~is called the \emph{complementary modulus}, and is evidently the
minimum value of~$\Delta \phi$, the value of~$\Delta \phi$ when~$\phi = 90°$:
\[
\sqrt{1 - k^{2}} = k'.
\]

From \Eqref{eq.}{}{(1)} we evidently have $k > k_{0}$, for, putting \Eqref{eq.}{}{(1)}
in the form
\[
\frac{k^{2}}{k_{0}^{2}} = \frac{4}{k_{0} + 2k_{0}^{2} + k_{0}^{3}},
\]
we see that if $k_{0} = 1$, then $k = k_{0}$, but as $k_{0} < 1$, always, as is
evident from the figure, $k$~must be greater than~$k_{0}$.

It is also evident, from the figure, that~$\phi_{1} > \phi$. Or it may
be deduced directly from \Eqref{eq.}{}{(3)}.

Since $k < 1$, we can write
\[
k = \sin \theta, \qquad k' = \sqrt{1 - k^{2}} = \cos \theta.
\]
%% -----File: 038.png---Folio 32-------

Substituting in \Eqref{eq.}{}{(5)}, we have
\[
k_{0} = \frac{1 - k'}{1 + k'} = \tan^2 \tfrac{1}{2} \theta,
\]
and we can write
\[
k_{0} = \sin \theta_{0}, \qquad k_{1}' = \sqrt{1 - k_{0}^2} = \cos \theta_{0}.
\]

From \Eqref{eq.}{}{(5)} we have
\[
1 + k_{0} = \frac{2}{1 + k'}.
\]

Substituting the value of~$k_{0}$ in that for~$k_{1}'$, we get
\[
k_{1}' = \frac{2\sqrt{k'}}{1 + k'}.
\]

We also have
\begin{DPalign*} %[** aligning next group]
2 \phi - \phi_{1} & = \phi - (\phi_{1} - \phi) \\
\phi_{1} & = \phi + (\phi_{1} - \phi), \\
\lintertext{and, \Eqref{eq.}{}{(3)},}
\sn (2\phi - \phi_{1}) &= k_{0} \sin \phi_{1},
\end{DPalign*}
becomes
\begin{multline*} %[** PP: Moving = to second line]
\sin \phi \cos (\phi_{1} - \phi) - \cos \phi \sin (\phi_{1} - \phi) \\
= k_{0} \sin \phi \cos (\phi_{1} - \phi) + k_{0} \cos \phi \sin (\phi_{1} - \phi),
\end{multline*}
or
\[
\tan \phi - \tan (\phi_{1} - \phi)
  = k_{0} \tan \phi + k_{0} \tan (\phi_{1} - \phi),
\]
or
\begin{align*}
\tan (\phi_{1} - \phi)
  &= \frac{1 - k_{0}}{1 + k_{0}} \tan \phi \\
  &= k' \tan \phi.
\end{align*}

Collecting these results, we have
\begin{align*}
\Tag{(6)}
k &= \frac{2\sqrt{k_{0}}}{1 + k_{0}} = \sin \theta; \\
\Tag{(7)}
k_{0}
  &= \frac{1 - k'}{1 + k'}
   = \sin \theta_{0}
   = \tan^{2} \tfrac{1}{2} \theta;
\end{align*}
%% -----File: 039.png---Folio 33-------
\begin{align*}
\Tag{(8)}
k_{1}' &= \frac{2\sqrt{k'}}{1 + k'} = \cos \theta_{0}; \\
\Tag{(9)}
k' &= \frac{1 - k_{0}}{1 + k_{0}} = \cos \theta; \\
\Tag{(10)}
1 + k_{0} &= \frac{2}{1 + k'}
   = \frac{2\sqrt{k_{0}}}{k}
   = \frac{k_{1}'}{\sqrt{k'}}
   = \frac{1}{\cos^{2} \frac{1}{2} \theta};
\end{align*}
\begin{align*}
\Tag{(11)}
\sin (2 \phi - \phi_{1}) &= k_{0} \sin \phi_{1}; \\
\Tag{(12)}
\tan (\phi_{1} - \phi) &= k' \tan \phi;
\end{align*}
\begin{align*}
\Tag{(13)}
\int_{0}^{\phi} \frac{d \phi}{\Delta (k, \phi)}
  &= \frac{1 + k_{0}}{2} \int_{0}^{\phi_{1}} \frac{d \phi_{1}}{\Delta (k_{0}, \phi_{1})}; \\
\Tag{(14)}
k &= \sqrt{1 - k'^{2}}, \quad k' = \sqrt{1 - k^{2}}.
\end{align*}

Upon examination it will easily appear that $k$ and~$k_{0}$, and $\theta$
and~$\theta_{0}$, are the first two terms of a decreasing series of moduli
and angles; $k'$~and~$k_{1}'$, and $\phi$~and~$\phi_{1}$, of an increasing series;
the law connecting the different terms of the series being deduced
from \Eqref{eqs.}{}{(6)} to~\Eqref{}{}{(12)}.

By repeated applications of these equations we would get
the following series of moduli and amplitudes:
\[
\renewcommand{\arraystretch}{\SMSTR}%
\begin{array}{l*{2}{>{\qquad}l}}
k_{0n} = 0_{(n = \infty)} & k_{n}' = 1_{(n = \infty)} & \phi_{n} \\
\PadTo{k_{00}}{\vdots} & \PadTo{k'_n}{\vdots} & \PadTo{\phi_n}{\vdots} \\
k_{00} & k_{2}' & \phi_{2} \\
k_{0}  & k_{1}' & \phi_{1} \\
k & k' & \phi
\end{array}
\]

The upper limit of the one series of moduli is~$1$, and the
lower limit of the other series is~$0$, as is indicated. $k$~and~$k'$,
%% -----File: 040.png---Folio 34-------
which are bound by the relation $k^{2} + k'^{2} = 1$, are called the
\emph{primitives} of the series.

\begin{Remark}
It will be noticed that the successive terms of a decreasing series
are indicated by the sub-accents $0, 00, 03, 04, \ldots 0n$; and the successive terms
of an increasing series by the sub-accents $1, 2, 3, \ldots n$.
\end{Remark}

Again, by application of these equations, we can form a new
series running up from~$k$, viz., $k_{1}, k_{2}, k_{3}, \ldots k_{n} = 1_{(n = \infty)}$; and
also a new series running down from~$k'$, viz., $k_{0}', k'_{00}, \ldots \DPtypo{k_{0n}}{k_{0n}'} =
0_{(n = \infty)}$. So also with~$\phi$.

Collecting these series, we have
\[
\renewcommand{\arraystretch}{\SMSTR}%
\begin{array}{l@{}*{2}{p{1in}l}}
k_{0n} \rlap{$ = 0$} && k_{n}' \rlap{$ = 1$} && \phi_{n} \\
\PadTo{k_{00}}{\vdots} && \PadTo{k'_n}{\vdots} && \PadTo{\phi_n}{\vdots} \\
k_{02} && k_{2}' && \phi_{2} \\
k_{0}  && k_{1}' && \phi_{1} \\
k      & \Dots{1} & k' & \Dots{1} & \phi \\
k_{1}  && k_{0}' && \phi_{0} \\
k_{2}  && k'_{00}&& \phi_{00} \\
\PadTo{k_{00}}{\vdots} && \PadTo{k'_n}{\vdots} && \PadTo{\phi_n}{\vdots} \\
k_{n} \rlap{$ = 1$} &&  k'_{0n} \rlap{$ = 0$} && \phi_{0n}=0
\end{array}
\]

\begin{Remark}
In practice it will be found that generally $n$ will not need to be very
large in order to reach the limiting values of the terms, often only two or three
terms being needed.
\end{Remark}

Applying \Eqref{eqs.}{}{(7)}, \Eqref{}{}{(12)},~\Eqref{}{}{(13)}, and~\Eqref{}{}{(14)} repeatedly, we get
\[
\Tag[14sub1]{(14_{1})}
\left\{\begin{array}{ll}
k=\sin \theta, & k' = \cos \theta; \\
k_{0} = \dfrac{1-k'}{1+k'} = \tan^{2} \frac{1}{2} \theta = \sin \theta_{0}, \qquad & k_{1}' = \cos \theta_{0}; \\
k_{00} = \tan^{2} \frac{1}{2} \theta_{0} = \sin \theta_{00}, & k_{2}' = \cos \theta_{00}; \\
k_{03} = \tan^{2} \frac{1}{2} \theta_{00} = \sin \theta_{03}, & k_{3}' = \cos \theta_{03}; \\
\Dots{2} \\
k_{0n} = \tan^{2} \frac{1}{2} \theta_{0(n-1)} = \sin \theta_{0n}, & k_{n}' = \cos \theta_{0n}.
\end{array}\right.
\]
%% -----File: 041.png---Folio 35-------
\begin{align*}
\Tag[14sub2]{(14_2)}
&\left\{
\begin{array}{l}
\tan (\phi_{1} - \phi) = k' \tan \phi;\\
\tan (\phi_{2} - \phi_{1}) = k_{1}' \tan \phi_{1};\\
\tan (\phi_{3} - \phi_{2}) = k_{2}' \tan \phi_{2};\\
\Dots{1} \\
\tan (\phi_{n} - \phi_{n - 1}) = k'_{(n - 1)} \tan \phi_{n - 1}.
\end{array} \right. \\
%
\Tag[14sub3]{(14_3)}
&\left\{
\begin{array}{r@{}l}
F(k, \phi) &{}= \dfrac{1 + k_{0}}{2} F(k_{0}, \phi_{1});\\
F(k_{0}, \phi_{1}) &{}= \dfrac{1 + k_{00}}{2} F(k_{00}, \phi_{2});\\
F(k_{00}, \phi_{2}) &{}= \dfrac{1 + k_{03}}{2} F(k_{03}, \phi_{3});\\
\Dots{2} \\
F(k_{0(n - 1)}, \phi_{n - 1}) &{}= \dfrac{1 + k_{0n}}{2} F(k_{0n}, \phi_{n}).
\end{array} \right.
\end{align*}

Multiplying these latter equations together, member by
member, we have
\[
\Tag{(15)}
F(k, \phi) = (1 + k_{0})(1 + k_{00}) \dotsm (1 + k_{0n})
  \frac{F(k_{0n}, \phi_{n})}{2^{n}};
\]
$k_{0}$,~$k_{00}$,~etc., and $\phi_{1}$,~$\phi_{2}$,~etc., being determined from the preceding
equations.

From \Eqref{eqs.}{}{(9)} and~\Eqref{}{}{(10)} we get
\[
1 + k_{0}  = \frac{1}{\cos^{2} \frac{1}{2} \theta},\qquad
1 + k_{00} = \frac{1}{\cos^{2} \frac{1}{2} \theta_{0}},\quad \text{etc.}
\]

Substituting these in \Eqref{eq.}{}{(15)}, we get
\[
\Tag{(16)}
F(k, \phi)
  = \frac{1}{\cos^{2} \dfrac{\theta}{2}
             \cos^{2} \dfrac{\theta_{0}}{2} \dotsm
             \cos^{2} \dfrac{\theta_{0n}}{2}}
  · \frac{F(k_{0n}, \phi_{n})}{2^{n}}.
\]
%% -----File: 042.png---Folio 36-------

From \Eqref{eqs.}{}{(15)} and~\Eqref{}{}{(10)} we get
\[
F(k, \phi)
  = \sqrt{\frac{k_{1}' k_{2}' k_{3}' \dotsm k_{n}'^{2}}{k'}}
    · \frac{F(k_{0n}, \phi_{n})}{2^{n}}.
\]
And this with \Eqref{equations}{}{(8)} and~\Eqref{}{}{(9)} gives
\[
\Tag{(17)}
F(k, \phi)
  = \sqrt{\frac{\cos \theta_{0} \cos \theta_{00} \dotsm \cos^{2} \theta_{0n}}{\cos \theta}}
    · \frac{F(k_{0n}, \phi_{n})}{2^{n}}.
\]

Applying \Eqref{equation}{}{(13)} to $(k_{1}, \phi_{0})$, $(k_{2}, \phi_{00})$,~etc., we get
\[
F(k_{1}, \phi_{0}) = \frac{1 + k}{2} F(k, \phi),\ \text{etc.};
\]
but since, \Eqref{eq.}{}{(10)},
\[
\frac{1 + k}{2} = \frac{1}{1 + k_{0}'},\ \text{etc.},
\]
these become
\[
\begin{array}{r@{}l}
F(k, \phi)
  &{} = (1 + k_{0}') F(k_{1}, \phi_{0}); \\
F(k_{1}, \phi_{0})
  &{} = (1 + k'_{00}) F(k_{2}, \phi_{00}); \\
\Dots{2} \\
\llap{$F(k_{n-1}, \phi_{0(n-1)})$}
  &{} = (1 + k'_{0n}) F(k_{\DPtypo{00}{n}}, \phi_{0n});
\end{array}
\]
whence
\[
\Tag{(18)}
F(k, \phi) = (1 + k_{0}')(1 + k'_{00}) \dotsm (1 + k'_{0n})F(k_{n}, \phi_{0n}),
\]
in which $k_{0}'$, $k_{00}'$,~etc., $k_{1}$, $k_{2}$,~etc., $\phi_{0}$, $\phi_{00}$,~etc., are determined
as follows:
\begin{DPalign*}
\lintertext{\indent Let}
k &= \sin \theta, \\
k_{1} & = \sin \theta_{1}.
\end{DPalign*}
From \Eqref{eq.}{}{(10)},
\[
k_{1} = \frac{2\sqrt{k}}{1 + k} \quad \text{or}\quad
\sin \theta_{1} = \frac{2\sqrt{\sin \theta}}{1 + \sin \theta}.
\]
%% -----File: 043.png---Folio 37-------

Solving this equation for~$\sin \theta$, we get
\[
\sin \theta = \tan^{2} \tfrac{1}{2} \theta_{1}.
\]

Hence we can write
\begin{align*}
\Tag[18sub1]{(18_{1})}
&\left\{\begin{array}{l}
k = \sin \theta = \tan^{2} \frac{1}{2} \theta_{1}; \\
k_{1} = \sin \theta_{1} = \tan^{2} \frac{1}{2} \theta_{2}; \\
\Dots{1} \\
k_{n} = \sin \theta_{n}.
\end{array}\right. \\
\intertext{\indent From \Eqref{equation}{}{(12)} we get}
\Tag[18sub2]{(18_{2})}
&\left\{\begin{array}{l}
\sin (2\phi_{0} - \phi) = k \sin \phi;\footnotemark \\
\sin (2\phi_{00} - \phi_{0}) = k_{1} \sin \phi_{0}; \\
\Dots{1} \\
\rlap{$\sin (2\phi_{0n} - \phi_{0(n-1)}) = k_{n-1} \sin \phi_{0(n-1)}$.}
\end{array}\right.
\end{align*}
\footnotetext{When $\sin \phi = 1$ nearly, $\phi$~is best determined as follows: From \Eqref{eq.}{}{(12)}
  we have
  \begin{align*}
  \tan (\phi - \phi_{0})
    &= k_{0}' \tan \phi_{0} \\
    &= k_{0}' \tan \phi\ \text{nearly};
  \intertext{whence}
  \phi - \phi_{0}
    &= Rk_{0}' \tan \phi\ \text{nearly},
  \end{align*}
  $R$~being the radian in seconds, viz.\ $206264''.806$, and $\log R = 5.3144251$.

  Substituting the approximate value of~$\phi_{0}$, we can get a new approximation.

  \textit{Example}.\qquad\qquad $\phi_{0} = 82°\ 30'$\qquad\qquad\qquad $k'_{00} = \log^{-1} 5.8757219$
  \[
  \begin{array}{l<{\qquad\qquad}@{}r@{}>{\qquad}c}
  \tan 82°\ 30' & 10.8805709 & \\
  k'_{00}   &     5.8757219 & \\
  R     &           5.3144251 & \\
  \cline{2-2}
  &                  2.0707179  & 117''.684 = 1'.9614
  \end{array}
  \]
  \begin{DPalign*}
  \phi_{0} - \phi_{00} & = 1'.9614 \\
  \phi_{00} & = 82°\ 28'.0386   \rintertext{1st approximation.}
  \end{DPalign*}

  This value gives
  \begin{DPalign*}
  \phi_{0} - \phi_{00} &= 117''.1675 = 1'.95279\\
  \therefore \phi_{00} &= 82°\ 28'.04721
  \rintertext{2d~approximation.}\\
  \intertext{\indent This value gives}
  \phi_{0} - \phi_{00} &= 117''.1698 = 1'.95283\\
             \phi_{00} &= 82°\ 28'.04717
  \rintertext{3d~approximation.}
  \end{DPalign*}
}
%% -----File: 044.png---Folio 38-------

To determine $k'_{0}$, $k'_{00}$,~etc., we have
\[
\Tag[18sub3]{(18_3)}
\left\{
\begin{aligned}
k' &= \sin \eta,   & k &= \cos \eta;\\
k'_{0}  &= \frac{1 - k}{1 + k} = \tan^{2} \tfrac{1}{2} \eta = \sin \eta_{0},
               & k_{1} &= \cos \eta_{0};\\
k'_{00} &= \tan^{2} \tfrac{1}{2} \eta_{0} = \sin \eta_{00},
               & k_{2} &= \cos \eta_{00};\\
        & \quad \PadTo[l]{\tan^{2} \tfrac{1}{2} \eta_{0}}{\quad\text{etc.}}
          \phantom{{}={}}
          \PadTo{\sin \eta_{00}}{\quad\text{etc.}} &
        & \PadTo{\cos \eta_{00}}{\text{etc.}}
\end{aligned}
\right.
\]

Or, since $1 + k'_0 = \dfrac{1}{\cos^2 \frac{1}{2} \eta}$,\quad $1 + k'_{00} = \dfrac{1}{\cos^2 \frac{1}{2} \eta_0}$, \text{etc.},
we can put \Eqref{eq.}{}{(18)} in the following form:
\[
\Tag{(19)}
F(k, \phi)
  = \frac{1}{\cos^{2} \frac{1}{2}\eta\,
             \cos^{2} \frac{1}{2}\eta_{0} \dotsm
             \cos^{2} \frac{1}{2}\eta_{0n}}\, F(k_{n}, \phi_{0n}).
\]

From \Eqref{equation}{}{(13)} we have
\begin{align*}
\Tag[19st]{(19)^*}
F(k_1, \phi_0) &= \frac{1 + k}{2} F(k, \phi), \\
\intertext{whence}
F(k, \phi) &= \frac{2}{1 + k} F(k_1, \phi_0).
\end{align*}

By repeated applications this gives, after combining,
\begin{align*}
F(k, \phi)
  &= \frac{2}{1 + k}
   · \frac{2}{1 + k_{1}} \dotsm
     \frac{2}{1 + k_{n - 1}} · F(k_{n}, \phi_{0n}) \\
%% -----File: 045.png---Folio 39-------
  &= \frac{k_{1}}{\sqrt{k}} ·
     \frac{k_{2}}{\sqrt{k_{1}}} \dotsm
     \frac{k_{n}}{\sqrt{k_{n-1}}} · F(k_{n}, \phi_{0n});
\end{align*}
\NegMathSkip
\[
\Tag{(20)}
F(k, \phi) = \sqrt{\frac{k_{1} k_{2} \dotsm k_{n}^{2}}{k}} · F(k_{n}, \phi_{0n});
\]
$k_{1}$,~$k_{2}$,~etc., being determined by repeated applications of
\[
k_{1} = \frac{2\sqrt{k}}{1 + k},
\]
or by \Eqref[18sub1]{equations}{}{(18_{1})}.

In \Eqref[19st]{equation}{}{(19)^*} let us change $k_{1}$~and~$\phi_{0}$ into $k'$~and~$\phi$ respectively,
so that the first member may have for its complete
function
\[
K' = F(k', \phi).
\]

Upon examination of \Eqref[19st]{eq.}{}{(19)^*} we see that the modulus in
the second member must be the one next less than the one in
the first member, that is,~$k_{0}'$; and likewise that the amplitude
must be the one next greater than the amplitude in the first
member, viz.,~$\phi_{1}$; hence we get
\[
F(k', \phi) = \frac{1 + k_{0}'}{2}\, F(k_{0}', \phi_{1}).
\]

Indicating the complete functions by $K'$~and~$K_{0}'$, we have,
since $\phi = \dfrac{\pi}{2}$ when $\phi_{1} = \pi$ (see \Chapref{Chap.}{V}),
\[
K' = (1 + k_{0}')K_{0}';
\]
and in the same manner,
\[
\begin{array}{r@{}l}
K_{0}' &{}= (1+k_{00}') K_{00}', \\
K_{00}' &{}= (1+k_{03}') K_{03}', \\
\Dots{2} \\
\llap{$K_{0(n-1)}'$} &{}= (1 + k_{0n}')K_{0n}';
\end{array}
\]
%% -----File: 046.png---Folio 40-------
whence
\[
K' = (1 + k_{0}')(1 + k_{00}') \dotsm (1+k_{0n}')K_{0n}'.
\]

Since
\begin{DPgather*}
K_{0n}' = \int_{0}^{\frac{\pi}{2}} d \phi = \frac{\pi}{2},
\rintertext{($n =$ limit,)}
\end{DPgather*}
we have
\[
\Tag[20st]{(20)^*}
(1 + k_{0}')(1 + k_{00}') \dotsm (1 + k_{0n}') = \frac{2K'}{\pi}.
\]

From \Eqref[19st]{eq.}{}{(19)^*} we have, since [\Eqref{eq.}{IV}{(10)}, Chap\DPtypo{}{.}~IV]
\begin{gather*}
\frac{1 + k}{2} = \frac{1}{1 + k_{0}'}, \\
(1 + k_{0}') \int_{0}^{\phi_{0}} \frac{d \phi_{0}}{\Delta (\phi_{0}, k_{1})}
  = \int_{0}^{\phi} \frac{d \phi}{\Delta (\phi_{1}, k)};
\end{gather*}
whence also, since for $\phi_{0} = \dfrac{\pi}{2}$, $\phi = \pi$,
\[
\begin{array}{r@{}l}
(1 + k_{0}')K_{1}  &{}= 2K, \\
(1 + k_{00}')K_{2} &{}= 2K_{1}, \\
\Dots{2} \\
(1 + k_{0n}')K_{n} &{}= \rlap{$2K_{n-1}$,}
\end{array}
\]
and
\[
(1 + k_{0}')(1 + k_{00}') \dotsm (1 + k_{0n}')K_{n} = 2^{n}K;
\]
or
\begin{DPalign*}
\frac{K_{n}}{2^{n}}
  &= \frac{K}{(1 + k_{0}')(1 + k_{00}') \dotsm}
\rintertext{($n = \infty$)} \\
\Tag{(21)}
  & = \frac{\pi}{2K_{1}} K.
\end{DPalign*}
%% -----File: 047.png---Folio 41-------

Let us find the limiting value of $F(k_{0n}, \phi_{n})$ in \Eqref{eq.}{}{(15)}. In
the equation $\tan (\phi_{n} - \phi_{n-1}) = k_{n-1} \tan \phi_{n-1}$, we see that when
$k_{n-1}$~reaches the limit~$1$, then $\phi_{n} - \phi_{n-1} = \phi_{n-1}$ or $\phi_{n} = 2\phi_{n-1}$.
Therefore
\begin{align*}
\frac{\phi_{n}}{2^{n}}
  &= \frac{2\phi_{n - 1}}{2^{n}}
   = \frac{\phi_{n-1}}{2^{n - 1}}; \\
%
\frac{\phi_{n+1}}{2^{n+1}}
  &= \frac{2\phi_{n}}{2^{n+1}}
   = \frac{\phi_{n}}{2^{n}}
   = \frac{\phi_{n-1}}{2^{n}}; \\
%
\frac{\phi_{n+m}}{2^{n+m}}
  &= \frac{\phi_{n - 1}}{2^{n}}
   = \text{constant, whatever $m$ may be}.
\end{align*}
Therefore \Eqref{eq.}{}{(15)} becomes
\[
\Tag[21st]{(21)^*}
F(k, \phi) = (1 + k_{0})(1 + k_{00}) \dotsm (1 + k_{0n}) \frac{\phi_{n}}{2^{n}},
\]
$n$~being whatever number will carry $k_{0}$ and~$\dfrac{\phi_{1}}{2}$ to their limiting
values.

In the same way, \Eqref{eqs.}{}{(16)} and~\Eqref{}{}{(17)} become
\begin{align*}
\Tag{(22)}
F(k, \phi)
  &= \frac{1}{\cos^{2} \dfrac{\theta}{2}
              \cos^{2} \dfrac{\theta_{0}}{2} \dotsm
              \cos^{2} \dfrac{\theta_{0n}}{2}} · \frac{\phi_{n}}{2^{n}} \\
\Tag{(23)}
  &= \sqrt{\frac{\cos \theta_{0} \cos \theta_{00} \dotsm \cos^{2} \theta_{0n}}{\cos \theta}}
   · \frac{\phi_{n}}{2^{n}},
\end{align*}
$n - 1$ being the number which makes $k_{n-1}' = 1$.

In these last three \DPtypo{equation}{equations} $k_{0}$,~$k_{00}$ are determined by \Eqref[14sub1]{eqs.}{}{(14_{1})};
$\phi_{1}$,~$\phi_{2}$,~etc., by \Eqref[14sub2]{eqs.}{}{(14_{2})}\footnotemark;
\footnotetext{Taking for~$\phi_{1} - \phi$, etc., not always the least angle given by the tables,
  but that which is nearest to~$\phi$.}%
$\theta$,~$\theta_{0}$,~etc., by \Eqref[14sub1]{eqs.}{}{(14_{1})}; and
$k'$,~$k_{1}'$, $\DPtypo{k_{2}}{k_{2}'}$,~etc., for use in \Eqref[14sub2]{eq.}{}{(14_{2})} by \Eqref[14sub1]{eqs.}{}{(14_{1})}.
%% -----File: 048.png---Folio 42-------

\Section{BISECTED AMPLITUDES.}

We have identically
\begin{DPalign*}
u &= 2 · \frac{u}{2}
   = 2 \raisebox{-1ex}{\scalebox{2}{$\displaystyle\int$}} \frac{d \am \dfrac{u}{2}}{\sqrt{1 - k^{2} \sn^{2} \dfrac{u}{2}}}; \\
\frac{u}{2}
  &= 2 · \frac{u}{4}
   = 2F\left(k, \am \frac{u}{4}\right); \\
  & \quad \text {etc.} \\
\intertext{\indent Therefore}
u &= F(k, \am u) = 2^{n} F\left(k, \am \frac{u}{n}\right) \\
  &= 2^{n} · \am \frac{u}{n},
\rintertext{($n =$ limit,)}
\end{DPalign*}
$\am \dfrac{u}{n}$ being determined by repeated applications of \Eqref{eq.}{II}{(12)} of
Chap.~II, as follows:
\begin{align*}
\sn^{2} \frac{u}{2}
  &= \frac{1 - \cn u}{1 + \dn u}
   = \frac{2 \sin^{2} \frac{1}{2} \am u}{1 + \dn u}; \\
%
\Tag{(24)}
\sn \frac{u}{2}
  &= \frac{\sin \frac{1}{2} \am u}{\sqrt{\dfrac{1 + \cos \beta}{2}}}
   = \frac{\sin \dfrac{\am u}{2}}{\cos \frac{1}{2} \beta};
\end{align*}
$\beta$ being an angle determined by the equation
\[
\Tag{(25)}
\cos \beta = \dn u = \sqrt{1 - k^{2} \sn^{2} u},
\]
and $n$ being the number which makes
\[
2^{n} \am \frac{u}{n} = \text{constant}.
\]
$\am \dfrac{u}{n}$ is found by repeated applications of \Eqref{eq.}{}{(24)}.
%% -----File: 049.png---Folio 43-------

Indicating the amplitudes as follows:
\[
\begin{array}{r@{}l}
\am u &{}= \phi, \\
\am \dfrac{u}{2} &{}= \phi_{02}, \\[2pt]
\am \dfrac{u}{4} &{}= \phi_{04}, \\[2pt]
\am \dfrac{u}{8} &{}= \phi_{08}, \\[2pt]
\Dots{2} \\
%[** PP: Em-dash present in high-res scan, retaining]
\llap{$\am \dfrac{u}{2^{n}}$} &{}= \rlap{$\phi_{02^{n}},\text{---}$} \\
\Tag{(26)}
\llap{$F(k, \phi)$} &{} = \rlap{$2^{n} \phi_{02^{n}}$;}
\end{array}
\]
$n$~being the limiting value.

In \Eqref{eq.}{}{(18)}, when $k_{n}$~reaches its limit~$1$, we have
\[
F(k_{n}, \phi_{0n})
  = \int_{0}^{\phi} \frac{d \phi_{0n}}{\cos \phi_{0n}}
  = \log_{\epsilon} \tan (45^{\circ} + \tfrac{1}{2} \phi_{0n}),
\]
and \Eqref{eqs.}{}{(18)} and~\Eqref{}{}{(19)} become
\begin{align*}
\Tag{(27)}
F(k, \phi)
  &= (1 + k_{0}')(1 + k_{00}') \dotsm (1 + k_{0n}')
     \log_{\epsilon} \tan (45° + \tfrac{1}{2} \phi_{0n}) \\
  &= \frac{1}{\cos^{2} \frac{1}{2} \eta\,
              \cos^{2} \frac{1}{2} \eta_{0} \dotsm
              \cos^{2} \frac{1}{2} \eta_{0n}}
     \log_{\epsilon} \tan (45° + \tfrac{1}{2} \phi_{0n}) \\
\Tag{(28)}
  &= \frac{1}{\cos^{2} \frac{1}{2} \eta\,
              \cos^{2} \frac{1}{2} \eta_{0} \dotsm
              \cos^{2} \frac{1}{2} \eta_{0n}}
   · \frac{1}{M} \log \tan (45° + \tfrac{1}{2} \phi_{0n});
\end{align*}
$n$~being the number which renders $k_{n} = 1$.

\Eqref{Eq.}{}{(20)} becomes
%% -----File: 050.png---Folio 44-------
\begin{align*}%[** PP: Realigning first line on =]
\Tag{(29)}
F(k, \phi)
  &= \sqrt{\frac{k_{1} k_{2} \dotsm k_{n}^{2}}{k}}
   · \log_{\epsilon} \tan (45° + \tfrac{1}{2} \phi_{0n}) \\
%
  &= \sqrt{\frac{k_{1}k_{2} \dotsm k_{n}^{2}}{k}}
   · \frac{1}{M} \log \tan (45° + \tfrac{1}{2} \phi_{0n}) \\
%
  &= \sqrt{\frac{\cos \eta_{0} \cos \eta_{00} \dotsm \cos^{2} \eta_{0n}}{\cos \eta}}
   · \frac{1}{M} \log \tan (45° + \tfrac{1}{2} \phi_{0n}).
\end{align*}

In these equations $k'_{0}$,~$k'_{00}$,~etc., are determined by \Eqref[18sub3]{eqs.}{}{(18_3)};
$\eta$,~$\eta_{0}$,~etc., by \Eqref[18sub3]{eqs.}{}{(18_3)}; $\phi_{0}$,~$\phi_{00}$,~etc., by \Eqref[18sub2]{eqs.}{}{(18_2)}; $k_{1}$,~$k_{2}$,~etc., by \Eqref[18sub1]{eqs.}{}{(18_1)}.

Substituting in \Eqref{eq.}{}{(27)} from \Eqref[20st]{eq.}{}{(20)^*}, we have
\begin{align*}
F(k, \phi)
  &= \frac{2K'}{\pi} \log_{\epsilon} \tan (45° + \tfrac{1}{2} \phi_{0n})\\
\Tag{(30)}
  &= \frac{2K'}{\pi M} \log \tan (45° + \tfrac{1}{2} \phi_{0n}).
\end{align*}
%% -----File: 051.png---Folio 45-------


\Chapter{V}{Complete Functions}

\First{Indicate} by~$K$ the complete integral
\[
\Tag{(1)}
K = \int_{0}^{\frac{\pi}{2}} \frac{d\phi}{\sqrt{1 - k^{2} \sin^{2} \phi}},
\]
and by~$K_{0}$ the complete integral
\[
\Tag{(2)}
K_{0} = \int_{0}^{\frac{\pi}{2}} \frac{d\phi_{1}}{\sqrt{1 - k_{0}^{2} \sin^{2} \phi_{1}}};
\]
and in a similar manner $K_{00}$,~$K_{03}$,~etc.

From \Eqref{eq.}{IV}{(12)}, Chap.~IV, we have
\begin{align*}
\tan (\phi_{1} - \phi)
  &= k' \tan \phi \\
  &= \frac{\tan \phi_{1} - \tan \phi}{1 + \tan \phi_{1} \tan \phi}, \\
\intertext{whence}
\tan \phi_{1}
  &= \frac{(1 + k') \tan \phi}{1 - k' \tan^{2} \phi} \\
  &= \frac{1 + k'}{\dfrac{1}{\tan \phi} - k' \tan \phi}.
\end{align*}
%% -----File: 052.png---Folio 46-------

From this equation we see that when $\phi = \dfrac{\pi}{2}$, $\phi_{1} = \pi$. This
same result might also have been deduced from \hyperref[page:30]{Fig.~1}, Chap.~IV,
or from the equation
\[
\Tag{(3)}
\phi_{1} = 2 \phi - k_{0} \sin 2\phi + \tfrac{1}{2} k_{0}^{2} \sin 4\phi - \text{etc.},
\]
this last being the well-known trigonometrical formula
\begin{gather*}
\tan x = n \tan y, \\
x = y - \frac{1 - n}{1 + n}\, \sin 2y
      + \frac{1}{2} \left(\frac{1 - n}{1 + n}\right)^{2} \sin 4y
      - \frac{1}{3} \left(\frac{1 - n}{1 + n}\right)^{3} \sin 6y + \text{etc.}
\end{gather*}
\begin{DPalign*}
\lintertext{Since}
\int_{0}^{\frac{\pi}{2}} \frac{d \phi_{1}}{\Delta (k_{0}\DPtypo{}{,} \phi_{1})}
  &= K_{0},\ \text{we must have} \\
\int_{0}^{\pi} \frac{d\phi_{1}}{\Delta (k_{0}\DPtypo{}{,} \phi_{1})}
  &= 2K_{0}.
\end{DPalign*}

These values substituted in \Eqref{eq.}{IV}{(13)}, Chap.~IV, give successively
\[
\begin{array}{r@{}l}
\Tag{(4)}
K &{}= (1 + k_{0})K_{0}, \\
K_{0} &{}= (1 + k_{00})K_{00}, \\
\Dots{2} \\
\llap{$K_{0(n - 1)}$} &{}= (1 + k_{0n})K_{0n};
\end{array}
\]
whence
\[
\Tag{(5)}
K = (1 + k_{0})(1 + k_{00}) \dotsm (1 + k_{0n})K_{0n}.
\]

Since the limit of~$k_{0n}$ is~$0$, $K_{0n}$~becomes
\[
K_{0n} = \int_{0}^{\frac{\pi}{2}} d \phi = \frac{\pi}{2},
\]
%% -----File: 053.png---Folio 47-------
and we have
\begin{align*}
\Tag{(6)}
K &= \frac{\pi}{2} (1 + k_{0})(1 + k_{00}) \dotsm \\
\Tag{(7)}
  &= \frac{\tfrac{1}{2} \pi}{\cos^{2} \tfrac{1}{2}\theta
                             \cos^{2} \tfrac{1}{2}\theta_{0} \dotsm
                             \cos^{2} \tfrac{1}{2}\theta_{0n}};
\end{align*}
$k_{1}$,~$k_{0}$,~etc., and $\theta_{1}$,~$\theta_{0}$,~etc., being found by \Eqref[14sub1]{eqs.}{IV}{(14_{1})} of Chap.~IV\@.

From the formulæ in these two chapters we can compute
the values of~$u$ for all values of $\phi$~and~$k$ and arrange them in
tables. These are Legendre's Tables of Elliptic Integrals.
%% -----File: 054.png---Folio 48-------


\Chapter[Evaluation for phi.]{VI}{Evaluation for $\phi$.}

\Section{TO FIND~$\phi$, $u$~AND~$k$ BEING GIVEN.}

\DPtypo{From}{\First{From}} \Eqref{eqs.}{IV}{(21)} and~\Eqref{}{IV}{(23)}, Chap.~IV, we have ($n$~having the
value which makes $\cos \theta_{0n} = 1$)
\[
\Tag{(1)}
\phi_{n}
  = \frac{2^{n}u}{(1 + k_{0})(1 + k_{00}) \dotsm (1 + k_{0n})}
  = \frac{2^{n}u \sqrt{\cos \theta}}{\sqrt{\cos \theta_{0} \dotsm \cos^{2} \theta_{0n}}},
\]
from which $\phi_{n}$~can be calculated, $k_{0},~k_{00}$,~etc., being found by
means of \Eqref[14sub1]{equations}{IV}{(14_1)}, Chap.~IV\@.

Then, having $\phi_{n}$, $k_{0}$,~$k_{00}$,~etc., we can find~$\phi$ by means of
the following equations:
\[
\begin{array}{r@{}l}
\sin (2\phi_{n - 1} - \phi_{n}) &= k_{0n} \sin \phi_{n},\\
\sin (2\phi_{n - 2} - \phi_{n - 1}) &= k_{0(n - 1)} \sin \phi_{n - 1},\\
\Dots{2} \\
\sin (2\phi - \phi_{1}) &= k_{0} \sin \phi_{1};\\
\end{array}
\]
whence we can get the angle~$\phi$.

When $k > \sqrt{\frac{1}{2}}$ the following formulæ will generally be found
to work more rapidly:

From \Eqref{eq.}{IV}{(29)}, Chap.~IV, we have
\[
\Tag{(2)}
\log \tan (45° + \tfrac{1}{2} \phi_{0n})
  = \frac{uM}{\sqrt{\dfrac{k_{1}k_{2} \dotsm k_{n}^{2}}{k}}},
\]
%% -----File: 055.png---Folio 49-------
from which we can get~$\phi_{0n}$; $k_{1}$,~$k_{2}$,~etc., being calculated from
\Eqref[18sub1]{eqs.}{IV}{(18_{1})}, Chap.~IV, and $\phi$~being calculated from the following
equations:
\[
\begin{array}{r@{}l}
\tan (\phi_{0(n - 1)} - \phi_{0n})
  &= k_{n} \tan \phi_{0n}, \\
%
\Dots{2} \\
%
\tan (\phi_{0} - \phi_{00})
  &= k_{2} \tan \phi_{02}, \\
%
\tan (\phi - \phi_{0})
  &= k \tan \phi_{0};
\end{array}
\]
whence we get~$\phi$.

This gives a method of solving the equation
\[
F\psi = n\, F\phi,
\]
where $n$~and~$\phi$ and the moduli are known, and $\psi$~is the
required quantity. $n$~and~$\phi$ give~$F\psi$, and then $\psi$~can be determined
by the foregoing methods.

When $k = 1$ \emph{nearly}, \Eqref{equation}{}{(2)} takes a special form,---

\primo. When~$\tan \phi$ is very much less than~$\dfrac{1}{k'}$. In this case
\begin{align*}
F(k, \phi)
  &= \int \frac{d\phi}{\sqrt{\cos^{2}\phi + k'^{2} \sin^{2} \phi}}
   = \int \frac{d\phi}{\sqrt{(1 + k'^{2} \tan^{2} \phi) \cos^{2} \phi}} \\
  &= \int \frac{d\phi}{\cos \phi}
   = \log \tan (45° + \tfrac{1}{2} \phi);
\end{align*}
whence we can find~$\phi$.

\secundo. When~$\tan \phi$ and~$\dfrac{1}{k'}$ approach somewhat the same value,
and $k' \tan \phi$~cannot be neglected, $F(k, \phi)$~must be transposed
into another where $k'$~shall be much smaller, so that $k' \tan \phi$~can
be neglected.
%% -----File: 056.png---Folio 50-------

These methods for finding~$\phi$ apply only when $\phi < \dfrac{\pi}{2}$, that
is, $u < K$. In the opposite case ($u > K$) put
\[
u = 2nK ± \nu,
\]
the upper or the lower sign being taken according as $K$~is continued
in~$u$ an even or an odd number of times. In either case
$\nu < K$, and we can find~$\nu$ by the preceding methods.

Having found~$\nu$, we have from \Eqref{eq.}{III}{(5)}, Chap.~III,
\begin{align*}
\am u
  &= \am (2nK ± \nu)\\
  &= n \pi ± \am \nu.
\end{align*}
%% -----File: 057.png---Folio 51-------


\Chapter{VII}{Development of Elliptic Functions into Factors.}
\SetRunningHeads{Development into Factors}

From \Eqref{eq.}{IV}{(12)}, Chap.~IV, we readily get
\begin{align*}
\sin (2\phi_{0} - \phi)
  &= k \sin \phi;\\
%
\sin \phi
  &= \frac{\sin 2 \phi_{0}}{\sqrt{1 + k^{2} + 2k \cos 2\phi_0}}\\
  &= \frac{\sin 2 \phi_{0}}{\sqrt{(1 + k)\DPtypo{}{^2} - 4k \sin^{2} \phi_0}}\\
  &= \frac{1 + k_{0}'}{2} · \frac{\sin 2 \phi_{0}}{\sqrt{1 - k_{1}^{2} \sin^{2} \phi_{0}}}\\
\end{align*}
$\Bigl(\text{since } \dfrac{4k}{1 + k} = k_{1} \text{ and } 1 + k = \dfrac{2}{1 + k_{0}'}, \text{\Eqref{eqs.}{IV}{(6)} and~\Eqref{}{IV}{(10)}, Chap.~IV}\Bigr)$;
and thence
\[
\Tag{(1)}
\sin \phi = \frac{(1 + k_{0}') \sin \phi_{0} \cos \phi_{0}}{\Delta (\phi_{0}, k_{1})}.
\]

From \Eqref{eq.}{IV}{(13)}, Chap.~IV, we have
\[
\int_{0}^{\phi_{0}} \frac{d \phi_{0}}{\Delta (\phi_{0}, k_{1})}
  = \frac{1 + k}{2} \int_{0}^{\phi} \frac{d\phi}{\Delta (\phi, k)};
\]
and from \Eqref{eq.}{V}{(4)}, Chap.~V, passing up the scale of moduli
one step,
\[
1 + k = \frac{K_{1}}{K},
\]
%% -----File: 058.png---Folio 52-------
whence %[** PP: Next several displays aligned on = in original]
\[
F(\phi_{0}, k_{1}) = \frac{K_{1}}{2K} F(\phi, k).
\]

Put
\[
F(\phi_{0}, k_{1}) = u_{1}\quad \text{and} \quad F(\phi, k) = u,
\]
whence
\[
u_{1} = \frac{K_{1}}{2K} u.
\]

Furthermore,
\begin{align*}
\phi &= \am (u, k);\\
\phi_{1} = \am (u_{1}, k_{1}) &= \am \left( \frac{K_{1}}{2K} u, k_{1} \right).
\end{align*}

Substituting these values in \Eqref{eq.}{}{(1)}, we have
\[
\sn (u, k)
  = (1 + k'_{0})
    \frac{\sn \left( \dfrac{K_{1}}{2K} u, k_{1} \right)
          \cn \left( \dfrac{K_{1}}{2K} u, k_{1} \right)}
         {\dn \left( \dfrac{K_{1}}{2K} u, k_{1} \right)}.
\]

But from \Eqref{eq.}{III}{(11)}, Chap.~III, we have
\begin{align*}
\frac{\cn (v, k_{1})}{\dn (v, k_{1})} &= \sn (v + K_{1}, k_{1}), \\
\intertext{or}
\frac{\cn \left( \dfrac{K_{1}}{2K} u, k_{1} \right)}
     {\dn \left( \dfrac{K_{1}}{2K} u, k_{1} \right)}
  &=  \sn \left( \frac{K_{1}u}{2K} + K_{1}, k_{1} \right) \\
  &=  \sn \left( \dfrac{K_{1}}{2K} (u + \DPtypo{K_{1}}{2K}), k_{1} \right);
\end{align*}
%% -----File: 059.png---Folio 53-------
whence
\begin{DPalign*}
\Tag{(2)}
\sn (u, k)
  &= (1 + k_{0}') \sn \frac{K_{1}u}{2K}
     \sn \left[\frac{K_{1}}{2K} (u + 2K)\right].\footnotemark \\
& \rintertext{\llap{(Mod.${}= k_{1}$.)}}
\end{DPalign*}
\footnotetext{The analogous formula in Trigonometry is
  \[
  \sin \phi = \DPtypo{\tfrac{1}{2}}{2} \sin \tfrac{1}{2} \phi\,
              \sin \tfrac{1}{2} (\phi + \pi).
  \]}

From this equation, evidently, we have generally
\begin{DPgather*}
\Tag[2st]{(2)^*}
\sn (\nu, k_{n})
  = (1 + k'_{0(n + 1)}) \sn \frac{K_{n + 1}}{2K_{n}} \nu
    \sn \left[\frac{K_{n + 1}}{2K_{n}} (\nu + 2K_{n})\right]. \\
\rintertext{\llap{(Mod.${}= k_{n + 1}$.)}}
\end{DPgather*}

Applying this general formula to the two factors of \Eqref{eq.}{}{(2)},
we have
\begin{DPalign*}
\sn \left(\frac{K_{1}u}{2K}, k_{1}\right)
  &= (1 + k'_{00}) \sn \frac{K_{2}}{2K_{1}} · \frac{K_{1}u}{2K}
      · \sn \left[\frac{K_{2}}{2K_{1}} \left(\frac{K_{1}u}{2K} + 2K_{1}\right)\right] \\
& \rintertext{\llap{(Mod.\ $k_{2}$)}} \\
  &= (1 + k'_{00}) \sn \frac{K_{2}u}{2^{2}K} \sn \frac{K_{2}}{2^{2}K} (u + 4K);
\rintertext{\llap{(Mod.\ $k_{2}$;)}}
\end{DPalign*}
\begin{DPgather*}
\Tag{(3)}
\sn \left[\frac{K_{1}}{2K} (u + 2K), k_{1}\right]
  = (1 + k'_{00}) \sn \frac{K_{2}}{2^{2}K} (u + 2K) \\
  · \sn \frac{K_{2}}{2K_{1}} \left[\frac{K_{1}}{2K} (u + 2K) + 2K_{1}\right].
\rintertext{(Mod.\ $k_{2}$.)}
\end{DPgather*}

The last argument in this equation is equal to
\[
\frac{K_{2}}{2^{2}K} (u + 6K);
\]
and since, \Eqref{eq.}{III}{(7)}, Chap.~III,
\[
\sn (u, k_{2}) = \sn (2K_{2} - u, k_{2}),
\]
%% -----File: 060.png---Folio 54-------
we can put in place of this,
\[
2K_{2} - \frac{K_{2}}{2^{2}K} (u + 6K\DPtypo{}{)} = \frac{K_{2}}{2^{2}K} (2K - u);
\]
whence \Eqref{eq.}{}{(3)} becomes
\begin{DPgather*}%[** PP: Re-breaking]
\sn \left[\frac{K_{1}}{2K} (u + 2K), k_{1}\right]
  = (1 + k'_{00}) \sn \frac{K_{2}}{2^{2}K} (2K + u)
    · \sn \frac{K_{2}}{2^{2}K} (2K - u). \\
\rintertext{\llap{(Mod.\ $k_{2}$.)}}
\end{DPgather*}

Substituting these values in \Eqref{eq.}{}{(2)}, we have
\begin{DPalign*}%[** PP: Aligning on equals sign]
\Tag{(4)}
\sn (u, k)
  &= (1+k'_{0})(1 + k'_{00})^{2} \sn \frac{K_{2}u}{2^{2}K} \\
  &\quad
  · \sn \frac{K_{2}}{2^{2}K} (2K ± u) \sn \frac{K_{2}}{2^{2}K} (4K+u),
\rintertext{(Mod.\ $k_{2}$,)}
\end{DPalign*}
in which the double sign indicates two separate factors which
are to be multiplied together.

By the application of the general \Eqref[2st]{equation}{}{(2)^*} we find that
the arguments in the second member of \Eqref{eq.}{}{(4)} will each give
rise to two new arguments, as follows:
\[
\frac{K_{2}u}{2^{2}K}\quad \text{gives}\quad \frac{K_{3}u}{2^{3}K},
\]
and
\begin{gather*}
\frac{K_{3}}{2K_{2}} \left(\frac{K_{2}u}{2^{2}K} + 2K_{2}\right)
  = \frac{K_{3}}{2^{3}K} (u + 8K); \\
\frac{K_{2}}{2^{2}K} (2K ± u)\quad \text{gives}\quad
\frac{K_{3}}{2^{3}K} (2K ± u),
\end{gather*}
%% -----File: 061.png---Folio 55-------
and
\begin{gather*}
\Tag{(a)}
\frac{K_{3}}{2K_{2}} \left[\frac{K_{2}}{2^{2}K} (2K ± u) + 2K_{2}\right]
  = \frac{K_{3}}{2^{3}K} (10K ± u), \\
\frac{K_{2}}{2^{2}K} (4K + u) \quad \text{gives} \quad \frac{K_{3}}{2^{3}K} (4K + u),
\intertext{and}
\Tag{(b)}
\frac{K_{3}}{2K_{2}} \left[\frac{K_{2}}{2^{2}K} (4K + u) + 2K_{2}\right]
  = \frac{K_{3}}{2^{3}K} (12K + u).
\end{gather*}

Subtracting~\Eqno{(a)} and~\Eqno{(b)} from~$2K_{3}$, by which the sine of the
amplitudes will not be changed [\Eqref{eq.}{III}{(7)}, Chap.~III], and since
our new modulus is~$k_{3}$, we have for the expressions \Eqno{(a)}~and~\Eqno{(b)},
\begin{gather*}
\Tag[apr]{(a')}
\frac{K_{3}}{2^{3}K} (6K \mp u); \\
\Tag[bpr]{(b')}
\frac{K_{3}}{2^{3}K} (4K - u).
\end{gather*}

Substituting these values in \Eqref{eq.}{}{(4)}, and remembering the
factor $(1 + k'_{03})$ introduced by each application of \Eqref[2st]{eq.}{}{(2)^*}, we
have
\begin{DPalign*}%[** PP: Align on equal signs, remove extraneous semicolons]
\sn (u, k)
  &= (1 + k'_{0})(1 + k'_{00})^{2}(1 + k'_{03})^{4}
     \sn \frac{K_{3}u}{2^{3}K}\DPtypo{;}{} \\
  &\quad·
     \sn \frac{K_{3}}{2^{3}K} (2K ± u)
     \sn \frac{K_{3}}{2^{3}K} (4K ± u)\DPtypo{;}{} \\
  &\quad·
     \sn \frac{K_{3}}{2^{3}K} (6K ± u)
     \sn \frac{K_{3}}{2^{3}K} (8K + u).
\rintertext{(Mod.\ $k_{3}$.)}
\end{DPalign*}
%% -----File: 062.png---Folio 56-------

From this the law governing the arguments is clear, and we
can write for the general equation
\begin{DPalign*}
\Tag{(5)}
\sn (u, k)
  &= (1 + k_{0}')(1 + k'_{00})^{2}(1 + k'_{03})^{4} \dotsm (1 + k'_{0n})^{2^{n-1}} \\
  &\quad· \sn \frac{K_{n}u}{2^{n}K}
          \sn \frac{K_{n}}{2^{n}K} (2K ± u) \\
  &\quad· \sn \frac{K_{n}}{2^{n}K} (4K ± u)
          \sn \frac{K_{n}}{2^{n}K} (6K ± u) \\
  &\quad \dotsm \sn \frac{K_{n}}{2^{n}K} \Bigl[(2^{n} - 2)K ± u\Bigr] \\
  &\quad· \sn \frac{K_{n}}{2^{n}K} (2^{n}K + u).
\rintertext{\llap{(Mod.\ $k_{n}$.)}}
\end{DPalign*}

Indicate the continued product of the binomial factors by~$A'$,
and we have
\[
A' = (1 + k_{0}')(1 + k'_{00})^{2}(1 + k'_{03})^{4}(1 + k'_{04})^{8} \dotsm.
\]

Since the limit of~$k'_{0}$, $k'_{00}$,~etc., is zero, it is evident that
these factors converge toward the value unity. It can be shown
that the functional factors also converge toward the value
unity. Thus the argument of the last factor can be written
\[
K_{n} + \frac{K_{n}u}{2^{n}K}.
\]

From \Eqref{eq.}{III}{(11)}, Chap.~III, we get then
\begin{DPgather*}
\sn \left(K_{n} + \frac{K_{n}u}{2^{n}K}\right)
  = \frac{\cn \dfrac{K_{n}u}{2^{n}K}}
         {\dn \dfrac{K_{n}u}{2^{n}K}}.
\rintertext{(Mod.\ $k_{n}$.)}
\end{DPgather*}

But since $k_{n}$ at its limit is equal to unity, $\cn = \dn$; whence
the last factor of \Eqref{eq.}{}{(5)} is unity.
%% -----File: 063.png---Folio 57-------

From \Eqref{eq.}{IV}{(21)}, Chap.~IV, we have
\[
\operatorname{limit} \frac{K_{n}}{2^{n}K} = \frac{2 \pi}{2 K'}.
\]

Therefore for $n = \infty$, \Eqref{eq.}{}{(5)} becomes
\begin{DPalign*}%[** PP: Align on equals sign]
\sn (u, k)
  &= A'   \sn \frac{\pi u}{2K'} \sn \frac{\pi}{2K'} (2K ± u) \\
  &\quad· \sn \frac{\pi}{2K'}(4K ± u) \sn \frac{\pi}{2K'}(6K ± u), \ldots
\rintertext{(Mod.\ $1$,)} \\
\intertext{or}
\Tag{(6)}
\sn (u, k)
  &= A' \sn \frac{\pi u}{2K'} \left[{\textstyle\prod\limits_{1}^{\infty}} h\right]
        \sn \frac{\pi}{2K'} (2hK ± u),
\rintertext{\llap{(Mod.\ $1$,)}}
\end{DPalign*}
where the sign~$\Prod$ indicates the continued product in the same
manner as $\sum$~indicates the continued sum.

When~$k = 1$, $\displaystyle\int_{0}^{\phi} F(\phi, k)\DPtypo{}{\,d\phi}$ becomes
\[
\nu = \int_{0}^{\phi} \frac{d \phi}{\cos \phi}
  = \tfrac{1}{2} \DPtypo{\log^e}{\log_\epsilon} \frac{1 + \sin \phi}{1 - \sin \phi};
\]
whence
\[
e^{2\nu} = \frac{1 + \sin \phi}{1 - \sin \phi},
\]
and
\[
\sin \phi = \frac{e^{2\nu} - 1}{e^{2\nu} + 1}
  = \frac{e^{\nu} - e^{-\nu}}{e^{\nu} + e^{-\nu}}
  = \sn (\nu, 1).
\]
%% -----File: 064.png---Folio 58-------

Hence in \Eqref{equation}{}{(6)}
\begin{align*}
\sin \frac{\pi u}{2K'}
  &= \frac{e^{\frac{\pi u}{2K'}} - e^{-\frac{\pi u}{2K'}}}
          {e^{\frac{\pi u}{2K'}} + e^{-\frac{\pi u}{2K'}}}; \\
\sn \frac{\pi}{2K'}(2hK ± u)
  &= \frac{e^{ \frac{h\pi K}{K'}} e^{±\frac{\pi u}{2K'}}
         - e^{-\frac{h\pi K}{K'}} e^{\mp\frac{\pi u}{2K'}}}
          {e^{ \frac{h\pi K}{K'}} e^{±\frac{\pi u}{2K'}}
         + e^{-\frac{h\pi K}{K'}} e^{\mp\frac{\pi u}{2K'}}}.
\end{align*}

Put
\[
\Tag[6st]{(6)^*}
q = e^{-\frac{\pi K'}{K}}, \quad q' = e^{-\frac{\pi K}{K'}},
\]
and the last expression becomes
\begin{align*}
\sn \frac{\pi}{2K'} & (2hK ± u)
  = \frac{q'^{-h} e^{±\frac{\pi u}{2K'}} - q'^{h} e^{\mp\frac{\pi u}{2K'}}}
         {q'^{-h} e^{±\frac{\pi u}{2K'}} + q'^{h} e^{\mp\frac{\pi u}{2K'}}}; \\
%
\sn \frac{\pi}{2K'} & (2hK + u) \sn \frac{\pi}{2K'}(2hK - u) \\
 &= \frac{q'^{-h} e^{ \frac{\pi u}{2K'}} - q'^{h} e^{-\frac{\pi u}{2K'}}}
         {q'^{-h} e^{ \frac{\pi u}{2K'}} + q'^{h} e^{-\frac{\pi u}{2K'}}}
  · \frac{q'^{-h} e^{-\frac{\pi u}{2K'}} - q'^{h} e^{ \frac{\pi u}{2K'}}}
         {q'^{-h} e^{-\frac{\pi u}{2K'}} + q'^{h} e^{ \frac{\pi u}{2K'}}} \\
%
 &= \frac{q'^{-2h} + q'^{2h} - \left(e^{\frac{\pi u}{K'}} + e^{-\frac{\pi u}{K'}}\right)}
         {q'^{-2h} + q'^{2h} + \left(e^{\frac{\pi u}{K'}} + e^{-\frac{\pi u}{K'}}\right)}.
\end{align*}

From plane trigonometry we have the equations
\[
\frac{e^{x} - e^{-x}}{e^{x} + e^{-x}} = -i \tan ix, \quad e^{x} + e^{-x} = 2 \cos ix;
\]
%% -----File: 065.png---Folio 59-------
where $i = \sqrt{-1}$: which gives
\begin{DPalign*}
\sn \frac{\pi u}{2K'}
  &= -i \tan \frac{\pi i u}{2K'};
\rintertext{\llap{(Mod.\ $1$;)}} \\
%
\sn \frac{\pi}{2K'} & (2hK + u) \sn \frac{\pi}{2K'}(2hK - u) \\
  &= \frac{q'^{-2h} + q'^{2h} - 2 \cos \dfrac{\pi iu}{K'}}
          {q'^{-2h} + q'^{2h} + 2 \cos \dfrac{\pi iu}{K'}} \\
%
%[** PP: No equation label in orig., but text refers specifically to (7).]
\DPtypo{}{\Tag{(7)}}
  &= \frac{1 - 2q'^{2h} \cos \dfrac{\pi iu}{K'} + q'^{4h}}
          {1 + 2q'^{2h} \cos \dfrac{\pi iu}{K'} + q'^{4h}}.
\end{DPalign*}

From \Eqref{eq.}{III}{(10)}, Chap.~III, we have
\[
\sn (u, k) = -i \tn (iu, k').
\]

Substituting these values in eq.~(6), we have
\[
\tn (iu, k') = A' \tan \frac{\pi iu}{2K'} \Prod
  \frac{1 - 2q'^{2h} \cos \dfrac{\pi iu}{K'} + q'^{4h}}
       {1 + 2q'^{2h} \cos \dfrac{\pi iu}{K'} + q'^{4h}}.
\]

Now in place of the series of moduli $k'$,~$k_{0}'$ and the corresponding
complete integral~$K'$, we are at liberty to substitute
the parallel series of moduli $k$,~$k_{0}$ and the corresponding complete
integral~$K$; calling the new integral~$u$, we have
\begin{align*}
\Tag{(8)}
\tn (u, k)
  &= A \tan \frac{\pi u}{2K} \DPtypo{\{\textstyle\prod}{\Prod}
     \frac{1 - 2q^{2h} \cos \dfrac{\pi u}{K} + q^{4h}}
          {1 + 2q^{2h} \cos \dfrac{\pi u}{K} + q^{4h}} \\
%% -----File: 066.png---Folio 60-------
  &= A \tan \frac{\pi u}{2K}
     \frac{1 - 2q^{2} \cos \dfrac{\pi u}{K} + q^{4}}
          {1 + 2q^{2} \cos \dfrac{\pi u}{K} + q^{4}} \\
%[** PP: Adding {} to the right of &s to coax · into math mode]
  &\PadTo[r]{{}= A \tan \dfrac{\pi u}{2K}}{{}·{}}
   \frac{1 - 2q^{4} \cos \dfrac{\pi u}{K} + q^{8}}
        {1 + 2q^{4} \cos \dfrac{\pi u}{K} + q^{8}} \\
  &\PadTo[r]{{}= A \tan \dfrac{\pi u}{2K}}{{}·{}}
   \frac{1 - 2q^{6} \cos \dfrac{\pi u}{K} + q^{12}}
        {1 + 2q^{6} \cos \dfrac{\pi u}{K} + q^{12}} \dotsm,
\end{align*}
where
\[
\Tag{(9)}
A = (1 + k_{0})(1 + k_{00})^{2}(1 + k_{03})^{4}(1 + k_{04})^{8} \dotsm
\]

Now in \Eqref{equation}{}{(6)} put $u+K$~for~$u$, and we have, since
[\Eqref{eq.}{III}{(11)}, Chap.~III] $\sn (u+K)= \dfrac{\cn u}{\dn u}$,
\begin{DPalign*}%[** PP: Re-breaking]
\frac{\cn u}{\dn u}
  &= A' \sn \frac{\pi (u+K)}{2K'} \\
  &\quad \Prod \sn \frac{\pi}{2K'} [(2h+1)K+u]
             · \sn \frac{\pi}{2K'} [(2h-1)K-u]. \\
&\rintertext{\llap{(Mod.\ $1$.)}}
\end{DPalign*}
Now from $2h-1$ and~$2h+1$ we have the following series of
numbers respectively:
\[
\begin{array}{l@{\qquad}*{6}{@{\quad}r}}
2h-1: & 1, & 3, & 5, & 7, & 9, & \text{etc.} \\
2h+1: &    & 3, & 5, & 7, & 9, & \text{etc.}
\end{array}
\]

It will be observed that the factor outside of the sign~$\Prod$,
viz., $\sin \am \dfrac{\pi (u + K)}{2K'}$, would, if placed under the sign~$\Prod$, supply
%% -----File: 067.png---Folio 61-------
the missing first term of the second series. Hence, placing this
factor within the sign, we have
\begin{DPalign*}%[** PP: Re-breaking]
\Tag{(10)}
\frac{\cn u}{\dn u}
  &= A'\Prod \sn \frac{\pi}{2K'} \left[(2h-1)K + u\right]
           · \sn \frac{\pi}{2K'} \left[(2h-1)K - u\right]. \\
  & \rintertext{\llap{(Mod.\ $1$.)}}
\end{DPalign*}

Comparing this with \Eqref{equation}{}{(7)}, we see that the factors
herein differ from those in \Eqref{equation}{}{(7)} only in having $2h-1$
in place of~$2h$; hence we have
\begin{DPgather*}
\sn \frac{\pi}{2K'} [(2h-1)K + u]
\sn \frac{\pi}{2K'} [(2h-1)K - u] \\
  = \frac{1 - 2q'^{2h-1} \cos \dfrac{\pi iu}{K'} + q'^{4h-2}}
         {1 + 2q'^{2h-1} \cos \dfrac{\pi iu}{K'} + q'^{4h-2}}.
\rintertext{(Mod.\ $1$.)}
\end{DPgather*}

From \Eqref{eqs.}{III}{(10)}, Chap.~III, we have
\[
\frac{\cn (u, k)}{\dn (u, k)} = \frac{1}{\dn(iu, k')};
\]
whence \Eqref{eq.}{}{(10)} becomes
\[
\Tag{(11)}
\frac{1}{\dn(iu, k')}
  = A'\Prod \frac{1 - 2q'^{2h-1} \cos \dfrac{\pi iu}{K'} + q'^{4h-2}}
                 {1 + 2q'^{2h-1} \cos \dfrac{\pi iu}{K'} + q'^{4h-2}};
\]
and when in place of~$iu$, $k'$, $K'$, $q'$,~$A'$, we substitute~$u$, $k$, $K$,
$q$ and~$A$, and invert the equation, we have
\[
\Tag{(12)}
\dn (u, k)
  = \frac{1}{A} \Prod \frac{1 + 2q^{2h-1} \cos \dfrac{\pi u}{K} + q^{4h-2}}
                           {1 - 2q^{2h-1} \cos \dfrac{\pi u}{K} + q^{4h-2}}.
\]
%% -----File: 068.png---Folio 62-------

Bearing in mind the remarkable property (Chap.~III, \Pageref{p.}{29})
that the functions $\sn u$~and~$\dn u$ approach infinity for the same
value of~$u$, we see that both these functions, except as to the
factor independent of~$u$, must have the same denominator.
Furthermore, since $\sn u$~and~$\tn u$ disappear for the same value
of~$u$, they must, except for the independent factor, have the
same numerator. Hence, indicating by~$B$ a new quantity,
dependent upon~$k$ but independent of~$u$, we have
\[
\Tag{(13)}
\sn (u, k) = B \sin \frac{\pi u}{2K} \Prod
  \frac{1 - 2q^{2h}   \cos \dfrac{\pi u}{K} + q^{4h}}
       {1 - 2q^{2h-1} \cos \dfrac{\pi u}{K} + q^{4h-2}};
\]
and since
\[
\cn u = \frac{\sn u}{\tn u},
\]
we also have, from \Eqref{eqs.}{}{(8)} and~\Eqref{}{}{(13)},
\[
\Tag{(14)}
\cn (u, k) = \frac{B}{A} \cos \frac{\pi u}{2K} \Prod
  \frac{1 + 2q^{2h}   \cos \dfrac{\pi u}{K} + q^{4h}}
       {1 - 2q^{2h-1} \cos \dfrac{\pi u}{K} + q^{4h-2}}.
\]
Collecting these results, we have the following equations:
\begin{align*}
\Tag{(15)}
\sn (u, k)
  &= B \sin \frac{\pi u}{2K} \Prod
  \frac{1 - 2q^{2h}   \cos \dfrac{\pi u}{K} + q^{4h}}
       {1 - 2q^{2h-1} \cos \dfrac{\pi u}{K} + q^{4h-2}}, \\
\Tag{(16)}
\cn (u, k)
  &= \frac{B}{A} \cos \frac{\pi u}{2K} \Prod
  \frac{1 + 2q^{2h}   \cos \dfrac{\pi u}{K} + q^{4h}}
       {1 - 2q^{2h-1} \cos \dfrac{\pi u}{K} + q^{4h-2}}, \\
%% -----File: 069.png---Folio 63-------
\Tag{(17)}
\dn (u, k) &= \frac{1}{A} \Prod
  \frac{1 + 2q^{2h-1} \cos \dfrac{\pi u}{K} + q^{4h-2}}
       {1 - 2q^{2h-1} \cos \dfrac{\pi u}{K} + q^{4h-2}}.
\end{align*}

To ascertain the values of $A$ and~$B$, we proceed as follows:

In \Eqref{eq.}{}{(17)} we make~$u = 0$, whence, by \Eqref{eq.}{II}{(13)}, Chap.~II,
we have
\begin{align*}
1 &= \frac{1}{A} \Prod \left(\frac{1 + q^{2h-1}}{1 - q^{2h-1}}\right)^{2}; \\
\intertext{whence}
\Tag{(18)}
\frac{1}{A} &= \Prod \left(\frac{1 - q^{2h-1}}{1 + q^{2h-1}}\right)^{2}. \\
\intertext{\indent In \Eqref{equation}{}{(17)}, making~$u=K$, we get, by \Eqref{equation}{III}{(1)},
Chap.~III,}
k' &= \frac{1}{A} \Prod \left(\frac{1 - q^{2h-1}}{1 + q^{2h-1}}\right)^{2}
    = \frac{1}{A^{2}}; \\
\Tag{(19)}
\therefore \frac{1}{A} &= \sqrt{k'}.
\end{align*}
We have identically
\begin{align*}
1 &= B \frac{1}{B} = B \frac{\dfrac{1}{A}}{\dfrac{B}{A}}
   = B \frac{\sqrt{k'}}{\dfrac{B}{A}}; \\
\intertext{whence}
\frac{B}{A} & = B \sqrt{k'}.
\end{align*}

To calculate~$B$, put~$e^{\frac{i\pi u}{2K}} = \nu$; if we change $\dfrac{\pi u}{2K}$ into
%% -----File: 070.png---Folio 64-------
$\dfrac{\pi u}{2K} + \dfrac{i\pi K'}{2K}$, $\nu$~will change into~$\nu \sqrt{q}$, and $\sn u$ will become, by
\Eqref{eq.}{III}{(14)}, Chap.~III,
\[
\sn (u+iK') = \frac{1}{k \sn u}.
\]

Now, replacing $\sin \dfrac{\pi u}{2K}$ and $\cos \dfrac{\pi u}{K}$ by their exponential values,
and observing that
\[
1 - 2q^{n} \cos \frac{\pi u}{K} + q^{2n} = (1 - q^{n}\nu^{2})(1 - q^{n}\nu^{-2}),
\]
we have
\[
\sn u = \frac{B}{2} · \frac{\nu - \nu^{-1} }{\sqrt{-1}}
      · \frac{\Prod (1 - q^{2h}  \nu^{2})(1 - q^{2h}  \nu^{-2})}
             {\Prod (1 - q^{2h-1}\nu^{2})(1 - q^{2h-1}\nu^{-2})}.
\]

Changing $u$ into~$u + iK'$, and consequently $\nu$~into~$\nu \sqrt{q}$, we
have
\[
\frac{1}{k \sn u}
  = \frac{B}{2} · \frac{\nu \sqrt{q} - \nu^{-1} \sqrt{q^{-1}}}{\sqrt{-1}}
  · \frac{\Prod (1 - q^{2h+1}\nu^{2})(1 - q^{2h-1}\nu^{-2})}
         {\Prod (1 - q^{2h}  \nu^{2})(1 - q^{2h-2}\nu^{\DPtypo{2}{-2}})}.
\]

Multiplying these equations together, member by member,
and observing that
\begin{align*}
\nu \sqrt{q} - \nu^{-1} \sqrt{q^{-1}}
  &= \frac{1 - q\nu^{2}}{-\nu \sqrt{q}}, \\
\nu - \nu^{-1}
  &= \nu(1 - \nu^{-2}),
\end{align*}
we get
\begin{align*}
\frac{1}{k}
  &= \frac{B^{2}}{4}
  · \frac{1 - q\nu^{2}}{\nu \sqrt{q}}
  · \nu(1 - \nu^{-2})
  · \frac{\Prod(1 - q^{2h+1}\nu^{2})(1 - q^{2h}  \nu^{-2})}
         {\Prod(1 - q^{2h-1}\nu^{2})(1 - q^{2h-2}\nu^{-2})} \\
%% -----File: 071.png---Folio 65-------
  &= \frac{B^{2}}{4 \sqrt{q}} (1 - q\nu^{2})\nu(1 - \nu^{-2})
     \frac{(1 - q^{3}\nu^{2})(1 - q^{5}\nu^{2}) \dotsm}
          {(1 - q    \nu^{2})(1 - q^{3}\nu^{2}) \dotsm} \\
  &\PadTo[r]{{}=\dfrac{B^{2}}{4 \sqrt{q}} (1 - q\nu^{2})\nu(1 - \nu^{-2})}{{}·{}}
     \frac{(1 - q^{2}\nu^{-2})(1 - q^{4}\nu^{-2}) \dotsm}
          {(1 -      \nu^{-2})(1 - q^{2}\nu^{-2}) \dotsm} \\
  &= \frac{B^{2}}{4} · \frac{1}{\sqrt{q}}.
\end{align*}
\begin{align*}
\therefore B &= \frac{2 \sqrt[4]{q}}{\sqrt{k}}; \\
\intertext{whence}
\frac{B}{A} &= 2 \sqrt[4]{q} \sqrt{\frac{k'}{k}}.
\end{align*}

Substituting these values in \Eqref{eqs.}{}{(15)},~\Eqref{}{}{(16)}, and~\Eqref{}{}{(17)}, we
have
\begin{align*}
\Tag{(20)}
\sn (u, k) &= \frac{2 \sqrt[4]{q}}{\sqrt{k}} \sin \frac{\pi u}{2K} \Prod
  \frac{1 - 2q^{2h} \cos \dfrac{\pi u}{K} + q^{4h}}
       {1 - 2q^{2h-1} \cos \dfrac{\pi u}{K} + q^{4h-2}}; \\
%
\Tag{(21)}
\cn (u, k) &= \frac{2 \sqrt{k'} \sqrt[4]{q}}{\sqrt{k}} \cos \frac{\pi u}{2K} \Prod
  \frac{1 + 2q^{2h}   \cos \dfrac{\pi u}{K} + q^{4h}}
       {1 - 2q^{2h-1} \cos \dfrac{\pi u}{K} + q^{4h-2}}; \\
%
\Tag{(22)}
\dn (u, k) &= \sqrt{k'} \Prod
  \frac{1 + 2q^{2h-1} \cos \dfrac{\pi u}{K} + q^{4h-2}}
       {1 - 2q^{2h-1} \cos \dfrac{\pi u}{K} + q^{4h-2}}.
\end{align*}
%% -----File: 072.png---Folio 66-------


\Chapter[The Theta Function.]{VIII}{The $\Theta$ Function.}

\First{We} will indicate the denominator in \Eqref{eq.}{VII}{(20)}, Chap.~VII, by~$\phi (u)$,
thus:
\[
\Tag{(1)}
\phi (u) = \Prod(1 - 2q^{2h - 1} \cos \frac{\pi u}{K} + q^{4h - 2}).
\]
We will now develop this into a series consisting of the cosines
of the multiples of $\dfrac{\pi u}{K}$. Put $\dfrac{\pi u}{2K} = x$, whence
\begin{align*}
2 \cos \frac{\pi u}{K} &= (e^{2ix} + e^{-2ix});\\
\intertext{but}
1 - 2q^{2h - 1} \cos \frac{\pi u}{K} + q^{4h - 2}
  &= (1 - q^{2h - 1}e^{2ix})(1 - q^{2h - 1}e^{-2ix}),
\end{align*}
and therefore
\begin{align*}
\Tag{(2)}
\phi (u)
  &= (1 - qe^{2ix})(1 - q^{3}e^{2ix})(1 - q^{5}e^{2ix}) \dotsm \\
  &\PadTo{{}={}}{}
     (1 - qe^{-2ix})(1 - q^{3}e^{-2ix})(1 - q^{5}e^{-2ix}) \dotsm
\end{align*}

Putting now $u + 2iK'$ instead of~$u$, we have
\begin{align*}
x_{1} &= \frac{\pi(u + 2iK')}{2K} = x + \frac{\pi iK'}{K},\\
2ix_{1} &= 2ix - \frac{2 \pi K'}{K};
\end{align*}
%% -----File: 073.png---Folio 67-------
and
\begin{align*}
e^{2ix_{1}} &= q^{2}e^{2ix}, \\
e^{-2ix_{1}} &= \frac{1}{q^{2}} e^{-2ix}.
\end{align*}
From these we have
\begin{align*}
\phi (u + 2iK') = - \frac{1}{q} e^{-2ix}
  & (1 - qe^{2ix}) (1 - q^{3}e^{2ix}) \dotsm \\
  & (1 - qe^{-2ix})(1 - q^{3}e^{-2ix}) \dotsm;
\end{align*}
whence
\begin{align*}
\phi (u + 2iK') &= -\frac{1}{q} e^{-2ix} \phi (u), \\
\intertext{or}
\Tag{(3)}
\phi (u + 2iK') &= -q^{-1} e^{-\frac{\pi iu}{K}} \phi (u).
\end{align*}

Now put
\[
\Tag{(4)}
\phi (u) = A + B \cos \frac{\pi u}{K}
  + C \cos \frac{2\pi u}{K}
  + D \cos \frac{3\pi u}{K} + \text{etc.}
\]

Since
\[
\cos \frac{\pi u}{K} = \tfrac{1}{2} \left(e^{2ix} + e^{-2ix}\right),
\]
this becomes
\begin{align*}
\Tag{(5)}
\phi (u) = A & + \tfrac{1}{2} Be^{2ix}  + \tfrac{1}{2} Ce^{4ix}  + \tfrac{1}{2} De^{6ix}  + \dotsb \\
             & + \tfrac{1}{2} Be^{-2ix} + \tfrac{1}{2} Ce^{-4ix} + \tfrac{1}{2} De^{-6ix} + \dotsb;
\end{align*}
whence
\begin{align*}
\Tag{(6)}
-\frac{1}{q} e^{-2ix} \phi (u) =
  & -\frac{A}{q}  e^{-2ix} - \frac{B}{2q} - \frac{C}{2q} e^{2ix} - \frac{D}{2q} e^{4ix} - \dotsb \\
  & -\frac{B}{2q} e^{-4ix} - \frac{C}{2q} e^{-6ix} - \frac{D}{2q} e^{-8ix} - \dotsb\DPtypo{}{.}
\end{align*}
%% -----File: 074.png---Folio 68-------

Now in \Eqref{equation}{}{(5)} put $u + 2iK'$ in place of~$u$, remembering
that $e^{2ix}$~and~$e^{-2ix}$ are thereby changed respectively into
$q^{2}e^{2ix}$~and~$q^{-2}e^{-2ix}$, and we have
\begin{align*}
\Tag{(7)}
\phi (u + 2iK') = A
  & + \frac{Bq^{2}}{2} e^{2ix}  + \frac{Cq^{4}}{2} e^{4ix}  + \frac{Dq^{6}}{2} e^{6ix} + \dotsb \\
  & + \frac{B}{2q^{2}} e^{-2ix} + \frac{C}{2q^{4}} e^{-4ix} + \dotsb.
\end{align*}

Since \Eqref{equations}{}{(6)} and~\Eqref{}{}{(7)} are equal, we have
\[
\begin{array}{r@{}lcr@{}l}
-\dfrac{B}{2q} &{}= A, &\qquad\qquad& B &= -2qA; \\
-\dfrac{C}{2q} &{}= \dfrac{Bq^{2}}{2}, && C &= +2q^{4}A; \\
-\dfrac{D}{2q} &{}= \dfrac{Cq^{4}}{2}, && D &= -2q^{9}A; \\
\Dots{2} && \Dots{2} \\
\end{array}
\]
whence
\[
\Tag{(8)}
\left\{
\begin{aligned} %[** PP: Retain small parentheses]
&\Prod (1 - 2q^{2h-1} \cos \dfrac{\pi u}{K} + q^{4h-2}) \\
&\begin{aligned}
  {} = A(1 - 2q \cos \dfrac{\pi u}{K}
    &+ 2q^{4} \cos \dfrac{2\pi u}{K} - 2q^{9} \cos \dfrac{3\pi u}{K} \\
    &+ 2q^{16} \cos \dfrac{4\pi u}{K} - \ldots).
  \end{aligned}
\end{aligned}
\right.
\]

The series in the second member has been designated by
Jacobi and subsequent writers by~$\Theta (u)$, thus:
\[
\Tag{(9)}
\Theta (u) = 1 - 2q \cos \frac{\pi u}{K} + 2q^{4} \cos \frac{2\pi u}{K} - \dotsb
\]
%% -----File: 075.png---Folio 69-------


\Chapter[The Theta and Eta Functions.]{IX}{The $\Theta$ and $\Eta$ Functions.}

\First{In} \Eqref{equation}{VII}{(20)}, Chap.~VII, viz.,
\[
\sn (u, k) = \frac{2 \sqrt[4]{q}}{\sqrt{k}} \sin \frac{\pi u}{2K}
  \Prod \frac{1 - 2q^{2h}   \cos \dfrac{\DPtypo{u \pi}{\pi u}}{K} + q^{4h}}
             {1 - 2q^{2h-1} \cos \dfrac{\pi u}{K} + q^{4h-2}},
\]
the numerator and the denominator have been considered separately
by Jacobi, who gave them a special notation and developed
from them a theory second only in importance to the
elliptic functions themselves.

Put [see \Eqref{equation}{VIII}{(8)}, Chap.~VIII]
\begin{gather*}
\Tag{(1)}
\Theta (u) = \frac{1}{A} \Prod (1 - 2q^{2h-1} \cos \frac{\pi u}{K} + q^{4h-2}). \\
\Tag{(2)}
\Eta (u) = 2 \frac{1}{A} \sqrt[4]{q} \sin \frac{\pi u}{2K}
  \Prod \left(1 - 2q^{2h} \cos \frac{\pi u}{K} + q^{4h}\right);
\end{gather*}
$A$ being a constant whose value is to be determined later.
From these we have
\[
\Tag{(3)}
\sn (u, k) = \frac{1}{\sqrt{k}} · \frac{\Eta (u)}{\Theta (u)}.
\]
%% -----File: 076.png---Folio 70-------

The functions $\sn u$~and~$\cn u$ can also be expressed in terms
of the new functions; thus we have
\[
\Tag{(4)}
\cn (u, k) = \sqrt{\frac{k'}{k}} · 2 \sqrt[4]{q} \cos \frac{\pi u}{2K}
  \Prod \frac{1 + 2q^{2h}   \cos \dfrac{\pi u}{K} + q^{4h}}
             {1 - 2q^{2h-1} \cos \dfrac{\pi u}{K} + q^{4h-2}};
\]
or, since $\sin x = \cos \left(x + \dfrac{\pi}{2}\right)$ and $\cos x = -\DPtypo{\cos}{\sin} \left(x + \dfrac{\pi}{2}\right)$,
and putting $u = \dfrac{2Kx}{\pi}$,
\begin{align*}
\cn \left(\frac{2Kx}{\pi}, k\right)
  &= \sqrt{\frac{k'}{k}}
     \frac{\Eta   \left[\dfrac{2K}{\pi} \left(x + \dfrac{\pi}{2}\right)\right]}
          {\Theta \left(\dfrac{2Kx}{\pi}\right)} \\
  &= \sqrt{\frac{k'}{k}}
     \frac{\Eta   \left[\dfrac{2Kx}{\pi} + K\right]}
          {\Theta \left(\dfrac{2Kx}{\pi}\right)}.
\end{align*}

Replacing $\dfrac{2Kx}{\pi}$ by its value,~$u$, we have
\[
\Tag{(5)}
\cn (u, k) = \sqrt{\frac{k'}{k}}\, \frac{\Eta (u + K)}{\Theta (u)}.
\]

Furthermore,
\[
\Tag{(6)}
\dn (u, k) = \sqrt{k'} \Prod
  \frac{1 + 2q^{2h-1} \cos \dfrac{\pi u}{K} + q^{4h-2}}
       {1 - 2q^{2h-1} \cos \dfrac{\pi u}{K} + q^{4h-2}}
\]
%% -----File: 077.png---Folio 71-------
gives in the same manner
\begin{align*}
\dn \frac{2Kx}{\pi}
  &= \sqrt{k'}\,
     \frac{\Theta \left[\dfrac{2K}{\pi} \left(x + \dfrac{\pi}{2}\right)\right]}
          {\Theta \left(\dfrac{2Kx}{\pi}\right)},
\intertext{or}
\Tag{(7)}
\dn (u, k)
  &= \sqrt{k'}\, \frac{\Theta (u + K)}{\Theta (u)}.
\end{align*}

If we put
\begin{align*}
\Tag{(8)}
\Eta (u + K)   &= \Eta_{1}(u), \\
\Tag{(9)}
\Theta (u + K) &= \Theta_{1}(u),
\end{align*}
the three elliptic functions can be expressed by the following
formulas:
\begin{align*}
\Tag{(10)}
\sn (u, k) &= \frac{1}{\sqrt{k}} · \frac{\Eta (u)}{\Theta (u)}; \\
\Tag{(11)}
\cn (u, k) &= \sqrt{\frac{k'}{k}} · \frac{\Eta_{1}(u)}{\Theta (u)}; \\
\Tag{(12)}
\dn (u, k) &= \sqrt{k'} \frac{\Theta_{1} (u)}{\Theta (u)}.
\end{align*}

These functions $\Theta$~and~$\Eta$ can be expressed in terms of
each other. By definition,
\[
\Eta (u) = 2C \sqrt[4]{q} \sin \frac{\pi u}{2K}
  \Prod \left(1 - 2q^{2h} \cos \frac{\pi u}{K} + q^{4h}\right);
\]
%% -----File: 078.png---Folio 72-------
but
\begin{align*}
1 - 2q^{h} \cos \frac{\pi u}{K} + q^{2h}
  &= \Bigl(1 - q^{h}e^{ \frac{\pi u \sqrt{-1}}{K}}\Bigr)
     \Bigl(1 - q^{h}e^{-\frac{\pi u \sqrt{-1}}{K}}\Bigr) \\
\sin \frac{\pi u}{2K}
  &= \frac{e^{\frac{\pi iu}{2K}} - e^{-\frac{\pi iu}{2K}}}{2 \sqrt{-1}} \\
  &= e^{\frac{-\pi iu}{2K}} \frac{1 - e^{\frac{\pi iu}{K}}}{2} \sqrt{-1},
\end{align*}
and consequently
\[
\Tag{(13)}
\Eta(u) = C \sqrt[4]{q} e^{-\frac{\pi iu}{2K}}
  \sqrt{-1} \Bigl(1 - e^{\frac{\pi iu}{K}}\Bigr)
            \Bigl(1 - q^{2}e^{-\frac{\pi iu}{K}}\Bigr)
            \Bigl(1 - q^{2}e^{ \frac{\pi iu}{K}}\Bigr) \dotsm.
\]

Now, changing $u$ into~$u + iK'$, and remembering that
$e^{-\frac{\pi K'}{K}} = q$, we have
\begin{multline*} %[** PP: Re-breaking]
\Tag{(14)}
\Eta(u + iK')
  = Cq^{-\frac{1}{4}}e^{\DPtypo{\frac{-\pi iu}{2K}}{-\frac{\pi iu}{2K}}}
  \sqrt{-1}\Bigl(1 - qe^{ \frac{\pi iu}{K}}\Bigr)
           \Bigl(1 - qe^{-\frac{\pi iu}{K}}\Bigr) \\
           \Bigl(1 - q^{3}e^{ \frac{\pi iu}{K}}\Bigr)
           \Bigl(1 - q^{3}e^{-\frac{\pi iu}{K}}\Bigr) \dotsm;
\end{multline*}
and reuniting the factors two by two, this becomes
\begin{multline*} %[** PP: Re-breaking]
\Tag{(15)}
\Eta(u + iK')
  = C \sqrt{-1} q^{-\frac{1}{4}}e^{-\frac{\pi iu}{2K}} \\
    \left(1 - 2q \cos \frac{\pi u}{K} + q^{2}\right)
    \left(1 - 2q^{3} \cos \frac{\pi u}{K} + q^{6}\right) \dotsm;
\end{multline*}
and finally, according to \Eqref{equation}{}{(1)},
\[
\Tag{(16)}
\Eta (u + iK') = \sqrt{-1} q^{-\frac{1}{4}} e^{-\frac{\pi iu}{2K}} \Theta (u).
\]
%% -----File: 079.png---Folio 73-------

In the same manner, we can get
\[
\Tag{(17)}
\Theta (u + iK') = \sqrt{-1} q^{-\frac{1}{4}}e^{-\frac{\pi iu}{2K}} \Eta (u).
\]

Substituting $u+2K$ for~$u$ in \Eqref{equations}{}{(1)} and~\Eqref{}{}{(2)}, we get
\begin{align*}
\Tag{(18)}
\Theta (u + 2K) &= \Theta (u), \\
\Tag{(19)}
\Eta (u + 2K) &= -\Eta (u),
\end{align*}
since $\cos \dfrac{\pi}{K} (u + 2K) = \cos \dfrac{\pi u}{K}$ and $\sin \dfrac{\pi}{2K} (u + 2K) = -\sin \dfrac{\pi u}{2K}$.

The comparison of these four equations with \Eqref{equations}{}{(10)},~\Eqref{}{}{(11)},
and~\Eqref{}{}{(12)} shows the periodicity of the elliptic functions.
For example, comparing \Eqref{eqs.}{}{(10)} and~\Eqref{}{}{(16)} and~\Eqref{}{}{(17)}, we
see that changing~$u$ into $u+iK'$ simply multiplies the numerator
and denominator of the second member of \Eqref{eq.}{}{(10)} by
the same number, and does not change their ratio.

The addition of~$2K$ changes the sign of the function, but
not its value.

We will define $\Theta_{1}$ and~$\Eta_{1}$ as follows:
\begin{align*}
\Tag{(20)}
\Theta_{1}(x) &= \Theta (x + K); \\
\Tag{(21)}
\Eta_{1}(x) &= \Eta (x + K).
\end{align*}
Hence we get, from \Eqref{equation}{}{(17)},
\begin{DPalign*}
\Theta_{1}(x + iK')
  &= \Theta (x + iK' + K) = \Theta (x + K + iK') \\
  &= i\Eta (x + K)e^{-\frac{i\pi}{4K} (2x + 2K + iK')} \\
  &= i\Eta_{1}(x)e^{-\frac{i\pi}{4K} (2x + iK')} (-\sqrt{-1}), \\
\lintertext{since}
e^{-\frac{i\pi}{2}}
  &= \cos \frac{\pi}{2} - \sqrt{-1} \sin \frac{\pi}{2} = -\sqrt{-1};
\end{DPalign*}
%% -----File: 080.png---Folio 74-------
whence
\[
\Tag{(22)}
\Theta_{1}(x + iK') = \Eta_{1}(x)e^{-\frac{i\pi}{4K}(2x + iK')}.
\]
In a similar manner we get
\[
\Tag[22st]{(22)^*}
\Eta_{1}(x + iK') = \Theta_{1}(x)e^{-\frac{i\pi}{4K}(2x + iK')}.
\]

In \Eqref{eq.}{VIII}{(9)}, Chap.~VIII, put $u = \dfrac{2Kz}{\pi}$, and we get
\[
\Tag{(23)}
\Theta \left(\frac{2Kz}{\pi}\right) = 1 - 2q \cos 2z + 2q^{4} \cos 4z - \dotsb.
\]

Now, in this equation, changing~$z$ into $z + \dfrac{\pi}{2}$, and observing
\Eqref{eq.}{}{(20)}, we get
\[
\Tag{(24)}
\Theta_{1} \left(\frac{2Kz}{\pi}\right) = 1 + 2q \cos 2z + 2q^{4} \cos 4z + \dotsb.
\]

Applying \Eqref{eq.}{}{(22)} to this, we have
\begin{align*}
&\begin{aligned}
\Eta_{1} \left(\frac{2Kz}{\pi}\right) %[** Explicit sizing of () required?]
  &= \Theta_{1}\left(\frac{2K}{\pi} \Bigl(z + \frac{\pi iK'}{2K}\Bigr)\right) e^{\frac{\pi i}{4K}\left(\frac{4Kz}{\pi} + iK'\right)} \\
  &= \Theta_{1}\left(\frac{2K}{\pi} \Bigl(z + \frac{\pi iK'}{2K}\Bigr)\right) e^{iz}q^{\frac{1}{4}}
\end{aligned} \\
%
  &= e^{iz}q^{\frac{1}{4}}
     \left[1 + 2q     \cos 2\Bigl(z + \frac{\pi iK'}{2K}\Bigr)
             + 2q^{4} \cos 4\Bigl(z + \frac{\pi iK'}{2K}\Bigr) + \dotsb\right] \\
  &= e^{iz}q^{\frac{1}{4}}
     \biggl[1 + q   \Bigl(e^{2i\bigl(z + \frac{\pi iK'}{2K}\bigr)}
                      + e^{-2i\bigl(z + \frac{\pi iK'}{2K}\bigr)}\Bigr) \\
  & \PadTo{{}=e^{iz}q^{\frac{1}{4}} \biggl[1}{}
           {} + q^{4} \Bigl(e^{4i\bigl(z + \frac{\pi iK'}{2K}\bigr)}
    + e^{-4i\bigl(z + \frac{\pi iK'}{2K}\bigr)}\Bigr) + \dotsb\biggr] \\
%% -----File: 081.png---Folio 75-------
  &= e^{iz} q^{\frac{1}{4}}
     \left[1 + q(qe^{2iz} + q^{-1}e^{-2iz})
             + q^{4}(q^{2}e^{4iz} + q^{-2}e^{-4iz}) + \dotsb \right] \\
%[** PP: Not breaking next two lines]
  &= e^{iz}q^{\frac{1}{4}}
     \left[1 + q^{2}e^{2iz} + q^{6}e^{4iz} + \dotsb
           + e^{-2iz} + q^{2}e^{-4iz} + \dotsb \right] \\
  &= q^{\frac{1}{4}} \left[e^{iz} + q^{2}e^{3iz} + q^{6}e^{5iz} + \dotsb
  + e^{-iz} + q^{2}e^{-3iz} + q^{6}e^{-5iz} + \dotsb \right] \\
  &= 2q^{\frac{1}{4}} \left[\cos z + q^{2} \cos 3z + q^{6} \cos 5z + \dotsb\right];
\end{align*}
whence
\[
\Tag{(25)}
\Eta_{1} \left( \frac{2Kz}{\pi} \right)
  = 2 \sqrt[4]{q} \cos z
  + 2 \sqrt[4]{q^{9}} \cos 3z
  + 2 \sqrt[4]{q^{25}} \cos 5z + \dotsb
\]

In this equation, changing~$z$ into $z - \dfrac{\pi}{2}$, and applying \Eqref{eq.}{}{(21)},
we get
\begin{align*}
\Tag{(26)}
\Eta \left( \frac{2Kz}{\pi} \right)
  &= 2 \sqrt[4]{q} \sin z
   - 2 \sqrt[4]{q^{9}} \sin 3z
   + 2 \sqrt[4]{q^{25}} \sin 5z - \dotsb, \\
\intertext{since}
\Eta_{1} \left( \frac{2Kz}{\pi} \right)
  &= \Eta \left( \frac{2Kz}{\pi} + K \right).
\end{align*}

We will now determine the constant~$A$ of \Eqref{eq.}{VIII}{(8)}, Chap.~VIII,
and \Eqref{eqs.}{}{(1)} and~\Eqref{}{}{(2)} of this chapter. Denote~$A$ by~$f(q)$,
and we have
\[
\Tag[26st]{(26)^*}
\Prod(1 - 2q^{2h - 1} \cos \frac{\pi u}{K} + q^{4h - 2}) = f(q)\Theta (u).
\]

Substituting herein $u = 0$ and $u = \dfrac{K}{2}$, we have
\begin{align*}
\Prod(1 - q^{2h - 1})^{2} &= f(q) \Theta (0);\\
\Prod(1 + q^{4h - 2}) &= f(q) \Theta \left( \frac{K}{2} \right).
\end{align*}
%% -----File: 082.png---Folio 76-------

From \Eqref{eq.}{VIII}{(9)}, Chap.~VIII, we get
\begin{align*}
\Tag{(27)}
\Theta (0)
  &= 1 - 2q + 2q^{4} - 2q^{9} + 2q^{16} - \dotsb; \\
\Tag{(28)}
\Theta \left( \frac{K}{2} \right)
  &= 1 - 2q^{4} + 2q^{16} - 2q^{36} + 2q^{64} - \dotsb;
\end{align*}
from which we see that $\Theta (0)$ is changed into $\Theta \left( \dfrac{K}{2} \right)$ when we
put $q^{4}$~in place of~$q$.

Whence
\[
\Prod(1 - q^{8h - 4})^{2} = f(q^{4})\Theta \left( \frac{K}{2} \right);
\]
and therefore
\begin{align*}
\frac{f(q)}{f(q^{4})}
  &= \Prod \frac{1 + q^{4h - 2}}{(1 - q^{8h - 4})^{2}}\\
\Tag{(29)}
  &= \Prod \frac{1}{(1 - q^{8h - 4})(1 - q^{4h - 2})}.
\end{align*}

Now, the expressions $4h - 2$, $8h - 4$, and~$8h$ give the
following series of numbers:
\begin{center}
\small
\begin{tabular}{l<{\qquad}*{18}{@{\,}c@{\,}}}
$4h - 2$, &2,&  &6,&  &10,&   &14,&   &18,&   &22,&   &26,&   &30,&   &34;&   \\
$8h - 4$, &  &4,&  &  &   &12,&   &   &   &20,&   &   &   &28,&   &   &   &36;\\
$8h$,     &  &  &  &8,&   &   &   &16,&   &   &   &24,&   &   &   &32.&   &
\end{tabular}
\end{center}
Hence, the three expressions taken together contain all the
even numbers, and
\[
\Prod(1 - q^{8h - 4})(1 - q^{4h - 2})(1 - q^{8h}) = \Prod(1 - q^{2h}).
\]
Therefore, multiplying \Eqref{eq.}{}{(29)} by
\begin{gather*}
\Prod \frac{1 - q^{8h}}{1 - q^{8h}},\\
\intertext{we have}
\frac{f(q)}{f(q^{4})} = \Prod \frac{1 - q^{8h}}{1 - q^{2h}}.
\end{gather*}
%% -----File: 083.png---Folio 77-------

Now in this equation, by successive substitutions of~$q^{4}$ for~$q$,
we get
\[
\begin{array}{r@{}l}
\dfrac{f(q^{4})}{f(q^{16})}   &{}= \Prod \dfrac{1 - q^{32h} }{1 - q^{8h}}; \\
\dfrac{f(q^{16})}{f(q^{64})}  &{}= \Prod \dfrac{1 - q^{128h}}{1 - q^{32h}}; \\
\dfrac{f(q^{64})}{f(q^{256})} &{}= \Prod \dfrac{1 - q^{512h}}{1 - q^{128h}}; \\
\Dots{2} \\
\end{array}
\]

Now $q$ being less than~$1$, $q^{n}$~tends towards the limit~$0$ as $n$~increases,
and consequently $1-q^{n}$ tends towards the limit~$1$.
Also, from \Eqref{eq.}{VIII}{(8)}, Chap.~VIII, we see that $f(0) = 1$. Hence,
multiplying the above equations together member by member,
we have
\begin{align*}
\Tag{(30)}
f(q) &= \Prod \frac{1}{1-q^{2h}}, \\
\intertext{or}
\Tag{(31)}
A &= \frac{1}{(1 - q^{2})(1 - q^{4})(1 - q^{6}) \dotsm}.
\end{align*}

Substituting this value in \Eqref{equation}{VIII}{(8)}, Chap.~VIII, we have,
after making $u = 0$,
\begin{align*}
(1 - q)^{2}(1 - q^{3})^{2}(1 - q^{5})^{2} \dotsm
  &= \frac{1- 2q + 2q^{4} - 2q^{9} + \dotsb}
          {(1 - q^{2})(1 - q^{4})(1 - q^{6}) \dotsm} \\
  &= \frac{\Theta (0)}{(1 - q^{2})(1 - q^{4})(1 - q^{6}) \dotsm}.
\end{align*}

(See \Eqref{equation}{VIII}{(9)}, Chap.~VIII.)

Transposing one of the series of products from the left-hand
member, we get
\[
(1 - q)(1 - q^{3}) \dotsm
  = \frac{\Theta (0)}{(1 - q)(1 - q^{2})(1 - q^{3})(1 - q^{4}) \dotsm}.
\]
%% -----File: 084.png---Folio 78-------

Introducing on both sides of the equation the factors $1 - q^{2}$,
$1 - q^{4}$, $1 - q^{6}$,~etc., we get
\begin{align*}
(1 - q) &(1 - q^{2})(1 - q^{3})(1 - q^{4}) \dotsm \\
  &= \Theta (0) \frac{1 - q^{2}}{1 - q}
              · \frac{1 - q^{4}}{1 - q^{2}}
              · \frac{1 - q^{6}}{1 - q^{3}}
              · \frac{1 - q^{8}}{1 - q^{4}} \dotsm \\
  &= \Theta (0) (1 + q)(1 + q^{2})(1 + q^{3}) \dotsm;
\intertext{whence}
\Tag{(32)}
\Theta (0)
  &= \frac{(1 - q)(1 - q^{2})(1 - q^{3})\DPtypo{}{\dotsm}}
          {(1 + q)(1 + q^{2})(1 + q^{3})\DPtypo{}{\dotsm}}.
\end{align*}

Resuming \Eqref{equation}{VII}{(20)}, Chap.~VII, and dividing both
members of the equation by~$u$, we have
\[
\frac{\sn u}{u}
  = \frac{2 \sqrt[4]{q}}{\sqrt{k}}\,
    \frac{\sin \dfrac{\pi u}{2K}}{u}
    \Prod \frac{1 - 2q^{2h}   \cos \dfrac{\pi u}{K} + q^{4h}}
               {1 - 2q^{2h-1} \cos \dfrac{\pi u}{K} + q^{4h-2}}.
\]
This, for~$u = 0$, since the limiting value of $\dfrac{\sn u}{u}$ for~$u = 0$ is~$1$,
and of $\dfrac{\sin \dfrac{\pi u}{2K}}{u}$ for~$x=0$ is~$\dfrac{\pi}{2K}$, becomes
\begin{align*}
1 &= \frac{\sqrt[4]{q}}{\sqrt{k}} · \frac{\pi}{K}
   · \frac{(1 - q^{2})^{2}(1 - q^{4})^{2}(1 - q^{6})^{2} \dotsm}
          {(1 - q)^{2}(1 - q^{3})^{2}(1 - q^{5})^{2} \dotsm},
\intertext{or}
\Tag{(33)}
\frac{\sqrt{k} K}{\pi \sqrt[4]{q}}
  &= \left[\frac{(1 - q^{2})(1 - q^{4})(1 - q^{6}) \dotsm}
                {(1 - q)(1 - q^{3})(1 - q^{5}) \dotsm}\right]^{2}.
\end{align*}

Further, from \Eqref{equation}{VII}{(21)}, Chap.~VII, for~$u=0$, we have
\[
\Tag{(34)}
\frac{\sqrt{k}}{2\sqrt{k'} \sqrt[4]{q}}
  = \left[\frac{(1 + q^{2})(1 + q^{4})(1 + q^{6}) \dotsm}
               {(1 - q)(1 - q^{3})(1 - q^{5}) \dotsm}\right]^{2}.
\]
%% -----File: 085.png---Folio 79-------

The quotient of these two equations gives
\[
\Tag{(35)}
\frac{2\sqrt{k'}K}{\pi}
  = \left[\frac{(1 - q^{2})(1 - q^{4})(1 - q^{6}) \dotsm}
               {(1 + q^{2})(1 + q^{4})(1 + q^{6}) \dotsm}\right]^{2};
\]
or, substituting the value of~$\sqrt{k'}$ from \Eqref{eqs.}{VII}{(18)} and~\Eqref{}{VII}{(19)}, Chap.~VII,
\[
\Tag{(36)}
\frac{2k'K}{\pi}
  = \left[\frac{(1 - q)(1 - q^{2})(1 - q^{3}) \dotsm}
               {(1 + q)(1 + q^{2})(1 + q^{3}) \dotsm}\right]^{2}.
\]

Comparing this with \Eqref{equation}{}{(32)}, we easily get
\[
\Tag{(37)}
\Theta (0) = \sqrt{\frac{2k'K}{\pi}}.
\]

From \Eqref{equation}{VIII}{(9)}, Chap.~VIII, making~$u=K$, we get
\[
\Tag{(38)}
\Theta (K) = 1 + 2q + 2q^{4} + 2q^{9} + 2q^{16} + \dotsb.
\]

Making $z=0$ in \Eqref{equation}{IX}{(24)}, Chap.~IX, we have
\[
\Tag{(39)}
\Theta_{1} (0) = 1 + 2q + 2q^{4} + 2q^{9} + \dotsb.
\]

This might also have been derived from \Eqref{eq.}{}{(38)} by observing
that
\[
\DPtypo{\Theta_{1}}{\Theta} (0 + K) = \Theta_{1}(0) = \Theta (K).
\]
Knowing $\Theta (0)$, it is easy to deduce $\Theta (K)$~and~$\Eta (K)$.

From \Eqref{equation}{}{(7)} we have
\[
\dn u = \sqrt{k'}\, \frac{\Theta (u + K)}{\Theta (u)}.
\]

Making $u=0$, we have, since $\dn (0) = 1$,
\[
\Tag{(40)}
\Theta (K) = \frac{\Theta (0)}{\sqrt{k'}}.
\]
%% -----File: 086.png---Folio 80-------

From \Eqref{equation}{}{(5)} we get, in the same manner,
\[
\Tag{(41)}
\Eta (K) = \sqrt{\frac{k'}{k}}\, \Theta (0).
\]

From \Eqref{eq.}{IX}{(12)}, Chap.~IX, we have
\[
\Tag[41st]{(41)^*}
\dn u
   = \sqrt{1 - k^{2} \sin^{2} \phi}
   = \sqrt{k'}\, \frac{\Theta_{1}(u)}{\Theta (u)};
\]
and putting $x = \dfrac{\pi u}{2K}$, we have
\[
\Tag{(42)}
\frac{\dn u}{\sqrt{k'}}
  = \frac{1 + 2q \cos 2x + 2q^{4} \cos 4x + 2q^{9} \cos 6x + \dotsb}
         {1 - 2q \cos 2x + 2q^{4} \cos 4x - 2q^{9} \cos 6x + \dotsb}.
\]

Putting
\[
\Tag[42st]{(42)^*}
\frac{\dn u}{\sqrt{k'}} = \cot \gamma,
\]
we have
\[
\frac{\cot \gamma - 1}{\cot \gamma + 1}
  = \tan (45° - \gamma)
  = 2q \frac{\cos2x + q^{8}(4 \cos^{3} 2x - 3 \cos 2x) + \dotsb}
            {1 + q^{4}(4 \cos^{2} 2x - 2)};
\]
whence
\begin{multline*}
\Tag{(43)}
\cos 2x = \frac{\tan (45° - \gamma) [1 + q^{4}(4 \cos^{2}2x - 2)]}{2q} \\
  - q^{8}(4 \cos^{3}2x - 3\cos2x) - \dotsb,
\end{multline*}
and approximately,
\[
\Tag{(44)}
\cos 2x = \frac{\tan (45° - \gamma)}{2q}.
\]

From \Eqref{equations}{IX}{(37)} and~\Eqref{}{IX}{(40)}, Chap.~IX, we have
\begin{align*}
\Tag{(45)}
x &= \frac{u}{\Theta^{2}(K)}; \\
\intertext{whence}
\Tag{(46)}
u &= x\Theta^{2}(K).
\end{align*}
%% -----File: 087.png---Folio 81-------


\Chapter{X}{Elliptic Integrals of the Second Order.}

\First{From} Chap.~I, \Eqref{equation}{I}{(19)}, we have
\[
E(k, \phi) = \int_{0}^{\phi} \sqrt{1 - k^{2} \sin^{2} \phi} · d\phi
  = \int_{0}^{\phi} \Delta \phi · d\phi.
\]

From this we have
\[
E(\phi) + E(\psi)
  = \int_{0}^{\phi} \Delta \phi · d\phi + \int_{0}^{\psi} \Delta \phi · d\phi.
\]

Put
\[
\Tag{(1)}
E\phi + E\psi = S.
\]

Differentiating, we get
\[
\Tag{(2)}
\Delta \phi · d\phi + \Delta \psi · d\psi = dS.
\]
But we have, Chap.~II, \Eqref{equation}{II}{(2)},
\[
\frac{d\phi}{\Delta \phi} + \frac{d\psi}{\Delta \psi} = 0,
\]
or
\[
\Tag{(3)}
\Delta \psi · d\phi + \Delta \phi · d\psi = 0.
\]

Adding equations \Eqref{}{}{(2)}~and~\Eqref{}{}{(3)}, we get
\[
\Tag{(4)}
(\Delta \phi + \Delta \psi)(d\phi + d\psi) = dS.
\]
%% -----File: 088.png---Folio 82-------

Substituting $\cos \mu$ from \Eqref{eq.}{II}{(5)}, in \Eqref[5st]{eq.}{II}{(5)^*}, Chap.~II, we get
\[
\Tag{(5)}
\left\{
\begin{aligned}
\Delta\phi
  &= \frac{\sin\phi \cos\psi\, \Delta\mu + \cos\phi \sin\psi}{\sin\mu}, \\
\Delta\psi
  &= \frac{\sin\psi \cos\phi\, \Delta\mu + \cos\psi \sin\phi}{\sin\mu};
\end{aligned}
\right.
\]
whence
\[
\Tag{(6)}
\Delta \phi ± \Delta \psi = \frac{\Delta \mu ± 1}{\sin \mu} \sin (\phi ± \psi).
\]

Substituting in \Eqref{equation}{}{(4)}, we have
\begin{align*}
dS &= \frac{\Delta \mu + 1}{\sin \mu} \sin (\phi + \psi)\,d(\phi + \psi) \\
\Tag{(7)}
   &= - \frac{\Delta \mu + 1}{\sin \mu}\,d \cos (\phi + \psi).
\end{align*}
Integrating equation~\Eqref{}{}{(7)}, we have
\[
E\phi + E\psi = \frac{\Delta \mu + 1}{\sin \mu} \left[C - \cos (\phi + \psi)\right].
\]

The constant of integration,~$C$, is determined by making
$\phi = 0$; in this case $\psi = \mu$, $E\phi = 0$, $E\psi = E\mu$, and $S = E\mu$;
whence
\[
E\mu = \frac{\Delta \mu + 1}{\sin \mu} (C - \cos \mu),
\]
and by subtraction,
\[
E\phi + E\psi - E\mu
  = \frac{\Delta \mu + 1}{\sin \mu}
    (\cos \mu - \cos \phi \cos \psi + \sin \phi \sin \psi).
\]
But, Chap.~II, \Eqref{eq.}{II}{(5)},
\[
\cos\mu - \cos\phi \cos\psi = - \sin\phi \sin\psi\, \Delta\mu;
\]
%% -----File: 089.png---Folio 83-------
whence
\[
E\phi + E\psi - E\mu = \frac{1 - \Delta^{2}\mu}{\sin \mu} \sin \phi \sin \psi
\]
whence
\[
\Tag{(8)}
E\phi + E\psi = E\mu + k^{2} \sin\phi \sin\psi \sin\mu.
\]

When $\phi = \psi$, we have
\begin{align*}
\Tag{(9)}
E\mu &= 2E\phi - k^{2} \sin^{2} \phi \sin\mu. \\
\intertext{But in that case}
\Tag{(10)}
\cos \mu &= \cos^{2} \phi - \sin^{2} \phi\, \Delta \mu; \\
\intertext{whence}
\Tag{(11)}
\sin \phi &= \sqrt{\frac{1 - \cos \mu}{1 + \Delta \mu}}.
\end{align*}

Let $\phi$, $\phi_{\frac{1}{2}}$, $\phi_{\frac{1}{4}}$,~etc., be such values as will satisfy the equations
\begin{align*}
\Tag{(12)}
E\phi
  &= 2E\phi_{\frac{1}{2}} - k^{2} \sin^{2} \phi_{\frac{1}{2}} \sin \phi,\\
E\phi_{\frac{1}{2}}
  &= 2E\phi_{\frac{1}{4}} - k^{2} \sin^{2} \phi_{\frac{1}{4}} \sin \phi_{\frac{1}{2}},\\
  &\PadTo{= 2E\phi_{\frac{1}{4}}}{\text{etc.}}
   \PadTo{ - k^{2} \sin^{2} \phi_{\frac{1}{4}} \sin \phi_{\frac{1}{2}}}{\text{etc.}}
\end{align*}

Assume an auxiliary angle~$\gamma$, such that
\[
\Tag{(13)}
\sin\gamma = k \sin\phi;
\]
whence
\[
\Delta\phi = \cos\gamma,
\]
and Chap.~IV, \Eqref{eq.}{IV}{(24)},
\[
\Tag{(14)}
\sin\phi_{\frac{1}{2}}
  = \frac{\sin \frac{1}{2}\phi}{\cos \frac{1}{2}\gamma}.
\]
%% -----File: 090.png---Folio 84-------

Applying eqs.\ \Eqref{}{}{(13)}~and~\Eqref{}{}{(14)} successively, we get
\[
\Tag{(15)}
\left\{
\begin{array}{r@{}l}
\sin \phi_{\frac{1}{2}}
  &{}= \dfrac{\sin \frac{1}{2} \phi}{\cos \frac{1}{2} \gamma},\quad
       \sin \gamma_{\frac{1}{2}} = k \sin \phi_{\frac{1}{2}}; \\
\sin \phi_{\frac{1}{4}}
  &{}= \dfrac{\sin \frac{1}{2} \phi_{\frac{1}{2}}}
             {\cos \frac{1}{2} \gamma_{\frac{1}{2}}},\quad
       \sin \gamma_{\frac{1}{4}} = k \sin \phi_{\frac{1}{4}}; \\
\Dots{2} \\
\sin \phi_{\frac{1}{2^{n}}}
  &{}= \dfrac{\sin \frac{1}{2} \phi_{\frac{1}{2^{n - 1}}}}
             {\cos \frac{1}{2} \gamma_{\frac{1}{2^{n - 1}}}};
\end{array}
\right.
\]
whence
\begin{align*}%[** PP: Unbalanced parentheses in original]
\Tag{(16)}
E \phi &= 2^{n} E \phi_{\frac{1}{2^{n}}}
  - \Bigl(\sin \phi \sin^{2} \gamma_{\frac{1}{2}}
      + 2 \sin \phi_{\frac{1}{2}} \sin^{2} \gamma_{\frac{1}{4}} \\
  &+ 2^{2} \sin \phi_{\frac{1}{4}} \sin^{2} \gamma_{\frac{1}{8}} + \dotsb
    2^{n-1} \sin \phi_{\frac{1}{2^{n}}} \sin^{2} \gamma_{\frac{1}{2^{n - 1}}} \Bigr)
\end{align*}

%[** PP: Modernizing factorial notation]
To find the limiting value, $E \phi_{\frac{1}{n}}$, we have, by the Binomial
Theorem, since $\sin \phi = 1 - \dfrac{\phi^{3}}{3!} + \dfrac{\phi^{5}}{5!} -{}$ etc.,
\begin{align*}
\Delta \phi
  &= (1 - k^{2} \sin^{2} \phi)^{\frac{1}{2}}\\
  &= 1 - \frac{k^{2}}{2} \left( \phi - \frac{\phi^{3}}{6} \right)^{2}
%[** PP: Fourth power in next line missing in original]
       - \frac{k^{4}}{8} \left( \phi - \frac{\phi^{3}}{6} \right)^{4} + \dotsb\\
  &= 1 - \frac{k^{2}}{2} \phi^{2}
       + \left( \frac{k^{2}}{6} - \frac{k^{4}}{8} \right) \phi^{4}.
%[** PP: Series truncated to polynomial, presumably 4th power approximation]
\end{align*}
Whence
\begin{align*}
E k \phi_{\frac{1}{2^{n}}}
  &= \int_{0}^{\phi_{n}} \Delta \phi_{\frac{1}{2^{n}}}\, d\phi\\
\Tag{(17)}
  &= \phi_{\frac{1}{2^{n}}} - \frac{k^{2}}{6} \phi^{3}_{\frac{1}{2^{n}}}
     + \frac{k^{2}(4 - 3k^{2})}{120} \phi^{5}_{\frac{1}{2^{n}}}.
\end{align*}
%% -----File: 091.png---Folio 85-------

Substituting in \Eqref{eq.}{}{(16)} the numerical values derived from
\Eqref{equations}{}{(15)} and~\Eqref{}{}{(17)}, we are enabled to determine the value
of~$E\phi$.

Landen's Transformation can also be applied to Elliptic
Integrals of this class.

From \Eqref{eq.}{IV}{(11)}, Chap.~IV, we get, by easy transformation,
\[
\Tag{(18)}
\sin^{2} 2\phi = \sin^{2} \phi_{1} (1 + k_{0} + 2k_{0} \cos 2\phi).
\]
From this we easily get
\begin{align*}
2k_{0} \cos 2\phi \sin^{2} \phi_{1}
  &= \sin^{2} 2\phi - \sin^{2} \phi_{1} - k_{0}^{2} \sin^{2} \phi_{1} \\
  &= 1 - \cos^{2} 2\phi - \sin^{2} \phi_{1} - k_{0}^{2} \sin^{2} \phi_{1} \\
  &= \Delta^{2}k_{0}\phi_{1} - \sin^{2}\phi_{1} - \cos^{2}2\phi;
\end{align*}
whence
\[
\cos^{2} 2\phi + 2k_{0} \sin^{2}\phi_{1} \cos 2\phi
  = \Delta^{2}k_{0}\phi_{1} - \sin^{2}\phi_{1};
\]
and from this,
\begin{align*}
\cos 2\phi
  &= -k_{0} \sin^{2} \phi_{1}
     ± \sqrt{\Delta^{2}k_{0}\phi_{1} - \sin^{2} \phi_{1}
             + k_{0}^{2} \sin^{4} \phi_{1}} \\
\Tag{(19)}
  &= \cos \phi_{1} \Delta k_{0}\phi_{1} - k_{0} \sin^{2} \phi_{1};
\end{align*}
whence, also,
\begin{align*}
1 - \cos^{2} 2\phi
  &= 1 - \cos^{2} \phi_{1}\, \Delta^{2}\phi_{1}
       + 2k \sin^{2}\phi_{1} \cos \phi_{1}\, \Delta k_{0}\phi_{1}
       - k_{0}^{2} \sin^{4} \phi_{1} \\
  &= \sin^{2} \phi_{1}
     (1 + k_{0}^{2} \cos^{2} \phi_{1}
       + 2k_{0} \cos \phi_{1}\, \Delta k_{0}\phi_{1}
       - k_{0}^{2} \sin^{2} \phi_{1})
\end{align*}
and
\[
\Tag{(20)}
\sin 2\phi = \sin \phi_{1} (\Delta k_{0}\phi_{1} + k_{0} \cos \phi_{1}).
\]

Differentiating \Eqref{equation}{}{(19)}, we get
\[
2 \sin 2\phi \frac{d\phi}{d\phi_{1}}
  = \sin \phi_{1}
    \frac{(k_{0} \cos \phi_{1} + \Delta k_{0}\phi_{1})^{2}}
         {\Delta k_{0} \phi_{1}}.
\]
Dividing this by \Eqref{equation}{}{(20)}, we have
\[
\frac{2d\phi}{d\phi_{1}}
  = \frac{k_{0} \cos \phi_{1} + \Delta k_{0}\phi_{1}}{\Delta k_{0}\phi_{1}}.
\]
%% -----File: 092.png---Folio 86-------
But from~\Eqref{}{}{(19)}, and \Eqref{eq.}{IV}{(6)}, Chap.~IV,
\begin{align*}
k^{2} \sin^{2} \phi
  &= \frac{k^{2}(1 - \cos 2\phi)}{2} \\
  &= \frac{2k_{0}}{(1 + k_{0})^{2}}
     \{1 + k_{0} \sin^{2} \phi_{1} - \cos \phi_{1} \Delta k_{0} \phi_{1}\};
\intertext{whence}
\Delta k\, \phi
  &= \frac{\Delta k_{0}\phi_{1} + k_{0} \cos \phi_{1}}{1 + k_{0}},
\intertext{and}
2\Delta k\, \phi · \frac{d\phi}{d\phi_{1}}
  &= \frac{(k_{0} \cos \phi_{1} + \Delta k_{0}\, \phi_{1})^{2}}
          {(1 + k_{0}) \Delta k_{0}\, \phi_{1}},
\intertext{and}
d\phi\, \Delta k\, \phi
  &= \frac{d\phi_{1}}{\Delta k_{0} \phi_{1}}
   · \frac{(k_{0} \cos \phi_{1} + \Delta k_{0}\, \phi_{1})^{2}}{2(1 + k_{0})}.
\end{align*}
This gives immediately, by integration,
\begin{align*}
Ek\phi
  &= \frac{1}{2(1+k_{0})}
     \int \frac{d\phi_{1}}{\Delta k_{0}\, \phi_{1}}
          \{k_{0} \cos \phi_{1} + \Delta k_{0} \phi_{1}\}^{2} \\
  &=\frac{1}{2(1+k_{0})}
     \int \frac{d\phi_{1}}{\Delta k_{0}\, \phi_{1}}
          \{2\Delta^{2}k_{0}\, \phi_{1}
          + 2k_{0} \cos \phi_{1} \Delta k_{0}\phi_{1}
          - k_{1}'^{2}\} \\
\Tag{(21)}
  &= \frac{Ek_{0}\phi_{1}}{1+k_{0}}
   + \frac{k_{0} \sin \phi_{1}}{1+k_{0}}
   - \tfrac{1}{2} (1 - k_{0})Fk_{0}\phi_{1}.
\end{align*}

Thus the value of~$Ek\phi$ is made to depend upon~$Ek_{0}\phi_{1}$
(containing a smaller modulus and a larger amplitude), and
upon the integral of the first class,~$Fk_{0}\phi_{1}$; $k_{0}$,~$\phi_{1}$,~etc., being
determined by the \Eqref{formulæ}{IV}{(6)} to~\Eqref{}{IV}{(12)} of Chap.~IV.

By successive applications of \Eqref{equation}{}{(21)}, $Ek\phi$~may be
made to depend ultimately upon~$Ek_{0n}\phi_{n}$, where $k_{0n}$~approximates
to zero and $Ek_{0n}\phi_{n}$ to~$\phi_{n}$.

Or, by reversing, it may be made to depend upon~$Ek_{n}\phi_{0n}$,
where $k_{n}$~approximates to unity and $Ek_{n}\phi_{0n}$ to~$-\cos \phi_{0n}$.
%% -----File: 093.png---Folio 87-------

To facilitate this, assume
\[
Gk\phi = Ek\phi - Fk\phi.
\]
Subtracting from \Eqref{equation}{}{(21)} the equation
\[
Fk\phi = \frac{1 + k_{0}}{2} Fk_{0}\phi_{1}
  \text{ (see \Eqref{eq.}{IV}{(13)}, Chap.~IV)},
\]
we have
\[
Gk\phi = \frac{1}{1 + k_{0}}
  (Gk_{0}\phi_{1} + k_{0} \sin \phi_{1} - k_{0}\, Fk_{0}\, \phi_{1}).
\]
Repeated applications of this give
\[
\begin{array}{r@{}l}
Gk_{0}\phi_{1}
  &{}= \dfrac{1}{1 + k_{00}}
     (Gk_{00}\phi_{2} + k_{00} \sin \phi_{2} - k_{00}\, Fk_{00}\, \phi_{2}),\\
\Dots{2} \\
\llap{$Gk_{0(n - 1)}\phi_{n - 1}$}
  &{}= \dfrac{1}{1 + k_{0n}} \rlap{$(Gk_{0n}\phi_{n}
       + k_{0n} \sin \phi_{n} - k_{0n}\, Fk_{0n}\, \phi_{n})$.}
\end{array}
\]
Whence
\[
\Tag{(22)}
Gk\phi = \sum_{n}^{1} %[** PP: \textstyle sum in original]
  \Biggl\{ \frac{k_{0n}(\sin \phi_{n} - Fk_{0n}\phi_{n})}
                {\Prodlim(1 + k_{0n})} \Biggr\}
  + \frac{Gk_{0n}\, \phi_{n}}{\Prodlim(1 + k_{0n})}.
\]
But since (compare \Eqref{eq.}{IV}{(13)}, Chap.~IV)
\[%[** PP: Next two displays aligned in original]
Fk\phi = \frac{Fk_{0n}\, \phi_{n}\Prodlim(1 + k_{0n})}{2^{n}},
\]
or
\[
\Tag{(23)}
\frac{Fk_{0n}\, \phi_{n}}{\Prodlim (1 + k_{0n})}
  = \frac{2^{n} Fk\, \phi}{\Prodlim(1 + k_{0n})^{2}};
\]
%% -----File: 094.png---Folio 88-------
and since, also, (compare \Eqref{eq.}{IV}{(6)}, Chap.~IV,)
\[
\frac{k^{2}_{0(n-1)}}{k_{0n}} = \frac{2^{2}}{(1 + k_{0n})^{2}},
\]
we have
\begin{align*}
\Tag{(24)}
\frac{2^{n}k_{0n}}{\Prodlim(1 + k_{0n})^{2}}
  &= \frac{k_{0n}}{2^{n}} \Prodlim \frac{k^{2}_{0(n-1)}}{k_{0n}} \\
  &= \frac{k_{0n}}{2^{n}} \Prodlim \frac{k_{0(n-1)}}{k_{0n}} \Prodlim k_{0(n-1)} \\
  &= \frac{k_{0n}}{2^{n}} · \frac{k}{k_{0}} · \frac{k_{0}}{k_{00}} \dotsm \frac{k_{0(n-1)}}{k_{0n}} · k\Prodlim[2] k_{0(n-1)} \\
  &= \frac{k^{2}}{2^{n}} \Prodlim[2] k_{0(n-1)}.
\end{align*}

Substituting these values in \Eqref{equation}{}{(22)}, and neglecting
the term containing $Gk_{0n}\phi_{n}$ since, carried to its limiting
value,
\begin{DPalign*}
Gk_{0n}\phi_{n}
  &= Ek_{0n}\phi_{n} - Fk_{0n}\phi_{n} \\
  &= \phi_{n} - \phi_{n} = 0,
\rintertext{\llap{($n =$ limiting value,)}}
\end{DPalign*}
we have
\begin{gather*}
\Tag{(25)}
Gk\phi = \sum_{n}^{1} \Biggl\{ \frac{k\sqrt{k_{0n}} \sin \phi_{n}
  \Prodlim[2] \sqrt{k_{0(n-1)}} - k^{2} \Prodlim[2] k_{0(n-1)}}{2^{n}}
  \Biggr\} \\
  \begin{aligned}
  &= k \left[\frac{\sqrt{k_{0}}}{2} \sin \phi_{1}
           + \frac{\sqrt{k_{0}k_{00}}}{2^{2}} \sin \phi_{2}
           + \frac{\sqrt{k_{0}k_{00}k_{03}}}{2^{3}} \sin \phi_{3} + \dotsb\right] \\
  & \quad - \frac{k^{2}}{2} \left[1 + \frac{k_{0}}{2} + \frac{k_{0}k_{00}}{2^{2}} + \frac{k_{0}k_{00}k_{03}}{2^{3}} + \dotsb\right];
\end{aligned}
\end{gather*}
whence
\begin{gather*}
\Tag{(26)}
Ek\phi = Fk\phi \left[1 - \frac{k^{2}}{2}
  \left(1 + \frac{k_{0}}{2} + \frac{k_{0}k_{00}}{2^{2}} + \dotsb\right)\right] \\
  + k \left[\frac{\sqrt{k_{0}}}{2} \sin \phi_{1}
    + \frac{\sqrt{k_{0}k_{00}}}{2^{2}} \sin \phi_{2}
    + \frac{\sqrt{k_{0}k_{00}k_{03}}}{2^{3}} \sin \phi_{3} + \dotsb\right].
\end{gather*}
%% -----File: 095.png---Folio 89-------

From \Eqref{eq.}{V}{(3)}, Chap.~V, we see that when $\phi = \dfrac{\pi}{2}$,
\[
\phi_{n} = 2^{n-1} \pi.
\]

Substituting these values in \Eqref{equation}{}{(26)}, we have for a
complete Elliptic Integral of the second class,
\begin{multline*}
\Tag{(27)}
E \left(k, \frac{\pi}{2}\right) = \\
F \left(k, \frac{\pi}{2}\right)
  \left[1 - \frac{k^{2}}{2}
    \left(1 + \frac{k_{0}}{2}
            + \frac{k_{0}k_{00}}{2^{2}}
            + \frac{k_{0}k_{00}k_{03}}{2^{3}} + \dotsb\right)\right].
\end{multline*}

In a similar manner we could have found the formula for
$E (k, \phi)$ in terms of an increasing modulus, viz.,
\begin{align*}
\Tag{(28)}
E (k, \phi)
  &= F (k, \phi)
     \biggl[1 + k\biggl(1 + \frac{2}{k_{1}} + \frac{2^{2}}{k_{1}k_{2}}
                          + \frac{2^{3}}{k_{1}k_{2}k_{3}} + \dotsb \\
  &{} + \frac{2^{n-2}}{k_{1}k_{2} \dotsm k_{n-2}}
    - \frac{2^{n-1}}{k_{1}k_{2} \dotsm k_{n-1}}\biggr)\biggr] \\
  &{} - k \biggl[\sin \phi
        + \frac{2}{\sqrt{k}} \sin \phi_{1}
        + \frac{2^{2}}{\sqrt{kk_{1}}} \sin \phi_{2} + \dotsb \\
  &{} + \frac{2^{n-1}}{\sqrt{kk_{1} \dotsm k_{n-2}}} \sin \phi_{n-1}
      - \frac{2^{n}}{\sqrt{kk_{1} \dotsm k_{n-1}}} \sin \phi_{n}\biggr].
\end{align*}
%% -----File: 096.png---Folio 90-------


\Chapter{XI}{Elliptic Integrals of the Third Order.}

\First{The} Elliptic Integral of the third order is
\[
\Tag{(1)}
\Pi (n, k, \phi) = \int_{0}^{\phi} \frac{d\phi}{(1 + n \sin^{2} \phi)\, \Delta \phi}.
\]

Put
\[
\Tag{(2)}
\Pi (\phi) + \Pi (\psi) = S;
\]
whence we have immediately
\[
\Tag{(3)}
dS = \frac{d\phi}{(1 + n \sin^{2} \phi)\, \Delta \phi}
   + \frac{d\psi}{(1 + n \sin^{2} \psi)\, \Delta \psi}.
\]
But, \Eqref{eq.}{II}{(2)}, Chap.~II,
\[
\Tag{(4)}
\frac{d\phi}{\Delta \phi} + \frac{d\psi}{\Delta \psi} = 0;
\]
whence
\begin{align*}
dS &= \left(\frac{1}{1 + n \sin^{2} \phi} - \frac{1}{1 + n \sin^{2} \psi}\right) \frac{d\phi}{\Delta \phi} \\
\Tag{(5)}
   &= \frac{n (\sin^{2} \psi - \sin^{2} \phi)}
           {(1 + n \sin^{2} \phi)(1 + n \sin^{2} \psi)} · \frac{d\phi}{\Delta \phi}.
\end{align*}

From \Eqref{equation}{X}{(8)}, Chap.~X, we get by differentiation, since
$\sigma$ (or~$\mu$) is constant,
\begin{align*}
\Delta \phi · d \phi + \Delta \psi · d\psi
  &= k^{2} \sin \sigma\, d(\sin \phi \sin \psi),
\intertext{or, from \Eqref{equation}{}{(3)},}
(\sin^{2} \psi - \sin^{2} \phi)\, \frac{d\phi}{\Delta \phi}
  &= \sin \sigma\, d(\sin \phi \sin \psi).
\end{align*}
%% -----File: 097.png---Folio 91-------
This, introduced into \Eqref{equation}{}{(5)}, gives
\[
dS = \frac{n \sin \sigma\, d (\sin \phi \sin \psi)}
          {1 + n (\sin^{2} \phi + \sin^{2} \psi) + n^{2} \sin^{2} \phi \sin^{2} \psi}.
\]

Put
\[
\sin \phi \sin \psi = q, \quad \sin^{2} \phi + \sin^{2} \psi = p;
\]
whence
\[
\Tag{(6)}
dS = \frac{n \sin \sigma\, dq}{1 + np + n^{2}p^{2}}.
\]

From \Eqref{equation}{II}{(5)}, Chap.~II, we have
\[
\cos \sigma = \cos\phi \cos\psi - \sin\phi \sin\psi\, \Delta\sigma,
\]
from which we easily get
\begin{align*}
(\cos \sigma + q \Delta \sigma)^{2}
  &= \cos^{2} \phi \cos^{2} \psi\DPtypo{)}{} \\
  &= (1 - \sin^{2} \phi)(1 - \sin^{2} \psi) \\
  &= 1 - p + q^{2},
\end{align*}
and thence
\begin{align*}
p &= 1 + q^{2} - (\cos \sigma + q\, \Delta \sigma)^{2} \\
  &= \sin^{2} \sigma - 2 \cos \sigma \Delta \sigma q + k^{2} \sin^{2} \sigma · q^{2}.
\end{align*}

This, substituted in \Eqref{eq.}{}{(6)}, gives
\begin{align*}
dS &= \frac{n \sin \sigma\, dq}
           {1 + n \sin^{2} \sigma - 2n \cos \sigma \Delta \sigma q + n(n + k^{2} \sin^{2} \sigma) q^{2}} \\
  &= \frac{n \sin \sigma\, dq}{A - 2Bq +Cq^{2}},
\end{align*}
where
\begin{align*}
A &= 1 + n \sin^{2} \sigma, \\
B &= n \cos \sigma\, \Delta \sigma, \\
C &= nk^{2} \sin^{2} \sigma + n^{2}.
\end{align*}
%% -----File: 098.png---Folio 92-------

From this we get
\[
S = n \sin \sigma \int \frac{dq}{A - 2Bq + Cq^{2}} + \text{Const.}
\]

In order to determine the constant of integration we must
observe that for $\phi = 0$, $\psi = \sigma$ and~$q = 0$; whence
\begin{align*}
\Pi \sigma &= n \sin \sigma \int_{q=0} \frac{dq}{A - 2Bq + Cq^{2}} + \text{Const.};
\intertext{whence}
S &= \Pi \sigma + n \sin \sigma \int_{0}^{q} \frac{dq}{A - 2Bq + Cq^{2}},
\intertext{or}
\Tag{(7)}
\Pi \phi + \Pi \psi &= \Pi \sigma + n \sin \sigma \int_{0}^{q} \frac{dq}{A - 2Bq + Cq^{2}}.
\end{align*}

But we have
\begin{align*}
dS &= \frac{CM\, dq}{AC - B^{2} + (Cq - B)^{2}} \\
  &= \frac{CM}{AC - B^{2}} · \frac{dq}{1 + \left(\dfrac{Cq - B}{\sqrt{AC - B^{2}}}\right)^{2}} \\
  &= \frac{M}{\sqrt{AC-B^{2}}}
   · \frac{\dfrac{C\, dq}{\sqrt{AC - B^{2}}}}
          {1 + \left(\dfrac{Cq - B}{\sqrt{AC - B^{2}}}\right)^{2}}
\end{align*}
where $M = n \sin \sigma$.

The integral of the second member is
\[
\frac{M}{\sqrt{AC - B^{2}}} \tan^{-1} \frac{Cq - B}{\sqrt{AC - B^{2}}};
\]
%% -----File: 099.png---Folio 93-------
whence
\[
\int_{0}^{q} dS = S_{1}
  = \frac{M}{\sqrt{AC - B^{2}}}
    \left[\tan^{-1} \frac{Cq - B}{\sqrt{AC - B^{2}}}
        + \tan^{-1} \frac{B}{\sqrt{AC - B^{2}}}\right];
\]
or, since
\begin{gather*}
\tan^{-1} x + \tan^{-1} y = \tan^{-1} \frac{x + y}{1 - xy}, \\
S_{1} = \frac{M}{\sqrt{AC - B^{2}}} \tan^{-1} \frac{q\sqrt{AC - B^{2}}}{A - Bq}.
\end{gather*}

Substituting the values of $A$,~$B$,~$C$ and~$M$, we have
\begin{align*}
AC - B^{2}
  &= n(1 + n - \Delta^{2} \sigma)(1 + n \sin^{2} \sigma) - n^{2} \cos^{2} \sigma\, \Delta^{2} \sigma \\
  &= n(1 + n - \Delta^{2} \sigma + n(1 + n) \sin^{2} \sigma - n\, \Delta^{2} \sigma) \\
  &= n(1 + n)(1 - \Delta^{2} \sigma + n \sin^{2} \sigma) \\
  &= n(1 + n)(k^{2} + n) \sin^{2} \sigma;
\end{align*}
and putting
\[
\frac{(1 + n)(k^{2} + n)}{n} = \Omega,
\]
we have
\[
\sqrt{AC - B^{2}} = n \sqrt{\Omega} \sin \sigma.
\]
Substituting these values in \Eqref{eq.}{}{(7)}, we have
\begin{gather*}
\Pi (n, k, \phi) + \Pi (n, k, \psi) - \Pi (n, k, \sigma) = S_{1} \\
= \frac{1}{\sqrt{\Omega}}
  \tan^{-1} \frac{n\sqrt{\Omega} \sin \phi \sin \psi \sin \sigma}
                 {1 + n \sin^{2} \sigma - n \sin \phi \sin \psi \cos \sigma\, \Delta \sigma}.
\end{gather*}
%% -----File: 100.png---Folio 94-------


\Chapter[Numerical Calculations. q.]{XII}{Numerical Calculations. $q$.}

\Section{CALCULATION OF THE VALUE OF~$q$.}

\First{From} \Eqref{eq.}{IX}{(7)}, Chap.~IX, we have
\[
\dn u = \sqrt{k'}\, \frac{\Theta (u + K)}{\Theta (u)};
\]
whence, \Eqref{eq.}{IV}{(9)}, Chap.~IV, \Eqref{eqs.}{IX}{(27)} and~\Eqref{}{IX}{(39)}, Chap.~IX,
\begin{align*}
\Tag{(1)}
\sqrt{\cos \theta}
  &= \frac{1 - 2q + 2q^{4} - 2q^{9} + 2q^{16} - \dotsb}
          {1 + 2q + 2q^{4} + 2q^{9} + 2q^{16} + \dotsb} \\
  &= 1 - 4q + 8q^{2} - 16q^{3} + 32q^{4} - 56q^{5} + \dotsb.
\end{align*}

The first five terms of this series can be represented by
\[
\sqrt{\cos \theta} = \frac{1 - 2q}{1 + 2q}.
\]
From this we get
\[
\Tag{(2)}
q = \frac{1}{2} · \frac{1 - \sqrt{\cos \theta}}{1 + \sqrt{\cos \theta}},
\]
which is exact up to the term containing~$q^{5}$.

Or we can deduce a more exact formula as follows: From
\Eqref{eq.}{}{(1)},
\begin{align*}
\frac{1 + \sqrt{\cos \theta}}{1 - \sqrt{\cos \theta}}
  &= \frac{\sqrt{1 + \tan^{2} \frac{1}{2} \theta}
         + \sqrt{1 - \tan^{2} \frac{1}{2} \theta}}
          {\sqrt{1 + \tan^{2} \frac{1}{2} \theta}
         - \sqrt{1 - \tan^{2} \frac{1}{2} \theta}} \\
  &= \frac{ 1 + 2q^{4} + 2q^{16} + \dotsb}
          {2q + 2q^{9} + 2q^{25} + \dotsb};
\end{align*}
%% -----File: 101.png---Folio 95-------
whence, by the method of indeterminate coefficients,
\begin{align*}
\Tag{(3)}
q &= \tfrac{1}{4}  \tan^{2} \frac{\theta}{2}
   + \tfrac{1}{16} \tan^{\DPtypo{2}{6}} \frac{\theta}{2}
   + \tfrac{\DPtypo{57}{17}}{512} \tan^{10} \frac{\theta}{2}
   + \tfrac{45}{2048} \tan^{14} \frac{\theta}{2} + \dotsb, \\
\intertext{or}
\log q %[** PP: Re-breaking]
  &= 2 \log \tan \frac{\theta}{2} - \log 4 \\
  &\qquad
   + \log \Bigl(1 + \tfrac{1}{4} \tan^{4} \frac{\theta}{2}
   + \tfrac{17}{128} \tan^{8} \frac{\theta}{2}
   + \tfrac{45}{512} \tan^{12} \frac{\theta}{2} \dotsb\Bigr) \\
\Tag{(4)}
  &= 2 \log \tan \frac{\theta}{2} - \log 4 \\
  &\qquad
   + M\Bigl(\tfrac{1}{4} \tan^{4} \frac{\theta}{2}
   + \tfrac{13}{128} \tan^{8} \frac{\theta}{2}
   + \tfrac{23}{384} \tan^{12} \frac{\theta}{2} + \dotsb\Bigr),
\end{align*}
$M$~being the modulus of the common system of logarithms.

Put
\[
\Tag{(5)}%[** PP: Re-breaking]
\log q = 2 \log \tan \frac{\theta}{2}
  + 9.397940 + a \tan^{4} \frac{\theta}{2}
  + b \tan^{8} \frac{\theta}{2}
  + c \tan^{12} \frac{\theta}{2} + \dotsb,
\]
in which
\begin{align*}
\log a &= 9.0357243; \\
\log b &= 8.64452; \\
\log c &= 8.41518; \\
\log d &= 8.25283.
\end{align*}

\Example. Let $k' = \cos 10°\ 23'\ 46''$. To find~$q$.
\[
\begin{array}[t]{r@{}c@{}l@{}}
4 \log \tan \dfrac{\theta}{2}
       &{}={}& 5.835 \\
\log a &{}={}& 9.036 \\
\cline{3-3}
            && 4.871 \\
&& \\
a \tan^{4} \dfrac{\theta}{2}
       &{}={}& \rlap{$0.0000074$}
\end{array}
\qquad\qquad
\begin{array}[t]{r@{}c@{}l@{}}
2 \log \tan \dfrac{\theta}{2}
       &{}={}& 7.9176842 \\
       &     & 9.3979400 \\
       &     & \PadTo[r]{9.3979400}{74} \\
\cline{3-3}
\log q &{}={}& 7.3156316
\end{array}
\]
%% -----File: 102.png---Folio 96-------

When $\theta$ approaches~$90°$, $\tan \dfrac{\theta}{2}$~differs little from unity, and
the series in \Eqref{eq.}{}{(5)} is not very converging, but $q$~can be calculated
by means of \Eqref{eq.}{VII}{(6)}, Chap.~VII, viz.,
\[
q = e^{-\frac{\pi K'}{K}}, \qquad q' = e^{-\frac{\pi K}{K'}}.
\]

By comparing these equations with \Eqref{eqs.}{IV}{(6)} and~\Eqref{}{IV}{(9)}, Chap.~IV,
we see that if
\begin{align*}
q &= f(k) = f(\theta),
\intertext{then}
q' &= f(k') = f(90° - \theta).
\end{align*}

Therefore, having $\theta$, we can from its complement, $90° - \theta$,
find~$q'$ by \Eqref{eq.}{}{(5)}, and thence~$q$ by the following process. We
have
\[
\frac{1}{q}  = e^{\frac{\pi K'}{K}}, \qquad
\frac{1}{q'} = e^{\frac{\pi K}{K'}};
\]
whence
\begin{align*}
\log \frac{1}{q} \log \frac{1}{q'} = M^{2}\pi^{2} &= 1.8615228, \\
\Tag{(6)}
\log \log \frac{1}{q} + \log \log \frac{1}{q'} &= 0.2698684,
\end{align*}
by which we can deduce $q$ from~$q'$.

\Example. Let $\theta = 79°\ 36'\ 14''$. To find~$q$.
\[
90° - \theta = 10°\ 23'\ 46''.
\]

By \Eqref{eq.}{}{(5)} we get
\begin{gather*}
\log q' = 7.3156316, \qquad \log \frac{1}{q'} = 2.6843684, \\
\text{and} \quad \log \log \frac{1}{q'} = .4288421;
\end{gather*}
%% -----File: 103.png---Folio 97-------
and by \Eqref{eq.}{}{(6)},
\begin{align*}
\log \log \frac{1}{q} &= 9.8410263;
\intertext{whence}
\log q &= 1.3065321.
\end{align*}

When $k' = k = \cos 45° = \frac{1}{2} \sqrt{2}$, \Eqref{eq.}{}{(6)} becomes
\begin{DPalign*}
\Tag{(7)}
\log \frac{1}{q}
  &= M \pi = 1.3643763;
\rintertext{($k = k'$;)} \\
\intertext{whence}
\log q &= 2.6356237, \\
q &= 0.0432138.
\rintertext{($k = k'$.)}
\end{DPalign*}

\Example. Given $\theta = 10°\ 23'\ 46''$. Find~$q$. \\
\null\hfill\textit{Ans.} $\log q = 7.3156316$.

\Example. Given $\theta = 82°\ 45'$. Find~$q$. \\
\null\hfill\textit{Ans.} $\log q = 9.37919$.
%% -----File: 104.png---Folio 98-------


\Chapter[Numerical Calculations. K.]{XIII}{Numerical Calculations. $K$.}

\Section{CALCULATION OF THE VALUE OF $K$.}

\First{We} have already found from \Eqref{eq.}{IX}{(37)}, Chap.~IX,
\[
\Tag{(1)}
\Theta (0) = \sqrt{\frac{2k'K}{\pi}},
\]
and from \Eqref{eq.}{IX}{(40)}, same chapter,
\[
\Tag{(2)}
\Theta (K) = \frac{\Theta (0)}{\sqrt{k'}} = \sqrt{\frac{2K}{\pi}}.
\]

But, \Eqref{eqs.}{IX}{(38)} and~\Eqref{}{IX}{(27)}, Chap.~IX,
\begin{align*}
\Theta (K) &= 1 + 2q + 2q^{4} + 2q^{9} + 2q^{16} + \dotsb, \\
\Theta (0) &= 1 - 2q + 2q^{4} - 2q^{9} + 2q^{16} - \dotsb;
\end{align*}
whence, \Eqref{eq.}{}{(2)},
\[
\Tag{(3)}
K = \frac{\pi}{2} (1 + 2q + 2q^{4} + 2q^{9} + \dotsb)^{2}.
\]

By adding \Eqref{eqs.}{}{(1)}~and~\Eqref{}{}{(2)} we get
\[
\Theta (0) + \Theta (K) = \sqrt{\frac{2K}{\pi}} (1 + \sqrt{k'});
\]
whence
\begin{align*}
K &= \frac{\pi}{2} \left(\frac{\Theta (0) + \Theta (K)}{1 + \sqrt{k'}}\right)^{2} \\
  &= \frac{\pi}{2} \left[\frac{2(1 + 2q^{4} + 2q^{16} + \dotsb)}{1 + \sqrt{k'}}\right]^{2} \\
%% -----File: 105.png---Folio 99-------
\Tag{(4)}
  &= \frac{\pi}{2} \left(\frac{2}{1 + \sqrt{k'}}\right)^{2}
     (1 + 2q^{4} + 2q^{16} + \dotsb)^{2}.
\end{align*}

\Example. Let $k = \sin \theta = \sin 19°\ 30'$. Required~$K$.

\emph{First Method.}\quad By \Eqref{eq.}{}{(3)}.

By \Eqref{eq.}{XII}{(5)}, Chap.~XII, we find $\log q = 8.6356236$. Applying
\Eqref{eq.}{}{(3)}, using only two terms of the series, we have
\[
\begin{array}{r@{}l@{}}
1 + 2q &{} = 1.0147662 \\
\PadTo[r]{1+2q}{\log (1 + 2q)} &{} = 0.0063660 \\
\PadTo[r]{1+2q}{2 \log (1 + 2q)} &{} = 0.0127320 \\
\log \dfrac{\pi}{2} &{} = 0.1961199 \\
\cline{1-2}
\log K &{} = 0.2088519 \\
K &{} = 1.615101
\end{array}
\]

\emph{Second Method.}\quad By \Eqref{eq.}{}{(4)}.

\Eqref{Equation}{}{(4)} may be written, neglecting~$q^{4}$,
\[
K = \frac{\pi}{2} \left(\frac{1 + \sqrt{\cos \theta}}{2}\right)^{-2};
\]
whence
\begin{align*}
\log \cos \theta & = 9.9743466, \\
\log \sqrt{\cos \theta} & = 9.9871733, \\
1 + \sqrt{\cos \theta} & = 1.9708973, \\
\frac{1 + \sqrt{\cos \theta}}{2} & = 0.98544865;
\intertext{and}
\log K & = 0.2088519, \\
K & = 1.615101,
\end{align*}
the same result as above.
%% -----File: 106.png---Folio 100-------

\emph{Third Method.}\quad By \Eqref{eq.}{V}{(7)}, Chap.~V.
\[
\begin{array}{r@{}c@{}l<{\quad}|>{\quad}r@{}c@{}l}
\theta &{}={} & 19°\ 30' &
  \theta_{0} &{}={}& 1°\ 41'\ 31''.1 \\
\frac{1}{2} \theta &{}={}& 9°\ 45' &
  \frac{1}{2} \theta_{0} &{}={}& 0°\ 50'\ 45''.5 \\
\log \tan \frac{1}{2} \theta &{}={}& 9.235103 & & & \\
\log \cos \frac{1}{2} \theta &{}={}& 9.993681 &
  \log \cos \frac{1}{2} \theta_{0}  &{}={}& 9.999953 \\
\begin{array}{r}
\log \tan^{2} \frac{1}{2} \theta \\ \log \sin \theta_{0}\end{array}\biggr\} & = & 8.470206 & & & \\
\theta_{0} & = & 1°\ 41'\ 31''.1 & & &
\end{array}
\]
\[
\begin{array}{r@{}c@{}l@{}}
\log \cos^{2} \tfrac{1}{2} \theta     &{}={}& 9.987362 \\
\log \cos^{2} \tfrac{1}{2} \theta_{0} &{}={}& 9.999906 \\
\cline{3-3}
                                      &     & 9.987268 \\
\log \dfrac{\pi}{2}                   &{}={}& 0.196120 \\
\cline{3-3}
\log K                                &{}={}& 0.208852
\end{array}
\]

$\theta_{00}$~is not calculated, as it is evident that its cosine will be~$1$.

\Example. Given $k = \sin 75°$. Find~$K$.

By \Eqref{eq.}{V}{(7)}, Chap.~V.

From \Eqref[14sub1]{eqs.}{IV}{(14_{1})}, Chap.~IV, we find
\settowidth{\TmpLen}{$\tan^{2} \tfrac{1}{2} \theta_{00}$}%
\begin{align*}
k &= \sin\theta = \sin 75°  & \log &= 9.9849438 \\
\PadTo[l]{k_{00}}{k_{0}} &= \biggl\{
  \begin{aligned}
    \makebox[\TmpLen][l]{$\tan^{2} \tfrac{1}{2} \theta$}
      &= \tan^{2} 37°\ 30' \\
    \makebox[\TmpLen][l]{$\sin \theta_{0}$}
      &= \sinP 36°\ \Z4'\ 16''.47\Z
  \end{aligned}\biggr\} & & \PadTo{{}={}}{} 9.7699610 \\
%
k_{00} &= \biggl\{
  \begin{aligned}
    \makebox[\TmpLen][l]{$\tan^{2} \tfrac{1}{2} \theta_{0}$}
      &= \tan^{2} 18°\ \Z2'\ \Z8''.235 \\
    \makebox[\TmpLen][l]{$\sin \theta_{00}$}
      &= \sinP \Z6°\ \Z5'\ \Z9''.38
  \end{aligned}\biggr\} & &  \PadTo{{}={}}{} 9.0253880 \\
%
k_{03} &= \biggl\{
  \begin{aligned}
    \makebox[\TmpLen][l]{$\tan^{2} \tfrac{1}{2} \theta_{00}$}
      &= \tan^{2} \Z3°\ \Z2'\ 34''.69\Z \\
    \makebox[\TmpLen][l]{$\sin \theta_{03}$}
      &= \PadTo[l]{\tan^2  18°}{\sin}\ \Z9'\ 42''.90
  \end{aligned}\biggr\} & &  \PadTo{{}={}}{} 7.4511672
\end{align*}
%% -----File: 107.png---Folio 101-------
\[
\begin{array}{l@{\,}c@{\,}l@{\,}r@{\ }lcc<{\quad}@{}c@{}}
 &&&&&   \log  &  2 \log  &  \ac 2 \log \\
\cos \frac{1}{2} \theta
  &=& \cos & 37° & 30'      & 9.8994667 & 9.7989334 & 0.2010666 \\
\cos \frac{1}{2} \theta_{0}
  &=& \cos & 18° & 2'.13725 & 9.9781184 & 9.9562368 & 0.0437632 \\
\cos \frac{1}{2} \theta_{02}
  &=& \cos &  3° & 2'.57817 & 9.9993873 & 9.9987746 & 0.0012254 \\
\cos \frac{1}{2} \theta_{03}
  &=& \cos &     & 4'.8575  & 9.9999995 & 9.9999990 & 0.0000010 \\
\cline{8-8}
& & & & & & &                                         0.2460562 \\
\multicolumn{1}{c}{\dfrac{\pi}{2}} & & & & & & \dfrac{\pi}{2} & \Z.1961199 \\
\cline{8-8}
& & & & & \multicolumn{3}{r@{}}{%
  \log K = \PadTo[r]{2.768064\quad \text{\textit{Ans.}}}{0.4421761}} \\
& & & & & \multicolumn{3}{r@{}}{%
       K = 2.768064\quad \text{\textit{Ans.}}}
\end{array}
\]

\Example. Given $k = \sin 45°$. Find~$K$.

Method of \Eqref{eq.}{V}{(7)}, Chap.~V.

From \Eqref[14sub1]{eqs.}{IV}{(14_{1})}, Chap.~IV, we have
\settowidth{\TmpLen}{$\tan^{2} \tfrac{1}{2} \theta_{00}$}%
\begin{align*}
& & \PadTo{9.2344486}{\log} \\
\PadTo[l]{k_{00}}{k_{0}} &= \biggl\{
  \begin{aligned}
    \makebox[\TmpLen][l]{$\tan^{2} \frac{1}{2} \theta$}
    &= \tan^{2} 22°\ 30' \\
    \makebox[\TmpLen][l]{$\sin \theta_{0}$}
    &= \sinP \Z9°\ 52'.75683
  \end{aligned}\biggr\} &  9.2344486 \\
%
k_{00} &= \biggl\{
  \begin{aligned}
    \makebox[\TmpLen][l]{$\tan^{2} \tfrac{1}{2} \theta_{0}$}
    &= \tan^{2} \Z4°\ 56'.37841 \\
    \makebox[\TmpLen][l]{$\sin \theta_{00}$}
    &= \PadTo[l]{\tan^2 22°}{\sin}\ 25'.679
  \end{aligned}\biggr\} & 7.8733009 \\
%
k_{03} &= \biggl\{
  \begin{aligned}
    \tan^{2} \tfrac{1}{2} \theta_{00}
    &= \PadTo[l]{\tan^2 22°}{\tan^{2}}\ 12'.3395\Z \\
    \makebox[\TmpLen][l]{$\sin \theta_{03}$}
    &= \PadTo[l]{\tan^2 22°}{\sin}\ \Z0'.05
  \end{aligned}\biggr\} & 5.1445523
\end{align*}
\[
\begin{array}{l@{}r@{}l@{}}
\ac \log \cos^{2} \frac{1}{2} \theta     &&  0.0687694 \\
\ac \log \cos^{2} \frac{1}{2} \theta_{0}  && 0.0032320 \\
\ac \log \cos^{2} \frac{1}{2} \theta_{00} && 0.0000060 \\
\multicolumn{1}{c}{\log \dfrac{\pi}{2}}  && 0.1961199 \\
\cline{3-3}
  & \log K = {}& 0.2681273 \\
  & K = {}& 1.8540747 \rlap{\quad\text{\textit{Ans.}}}
\end{array}
\]

\Example. Given $\theta = 63°\ 30'$. Find~$K$. \\
\null\hfill\textit{Ans.} $\log K = 0.3539686$.

\Example. Given $\theta = 34°\ 30'$. Find~$K$. \\
\null\hfill\textit{Ans.} $K = 1.72627$.
%% -----File: 108.png---Folio 102-------


\Chapter[Numerical Calculations. u.]{XIV}{Numerical Calculations. $u$}

\Section{CALCULATION OF THE VALUE OF~$u$.}

\First{When} $\theta° =  \sin^{-1}k < 45°$.

\Example. Let $\phi = 30°$, $k = \sin 45°$. Find~$u$.

\emph{First Method.} \Eqref{Eq.}{IV}{(23)}, Chap.~IV, and \Eqref[14sub1]{eqs.}{IV}{(14_{1})}, \Eqref[14sub2]{}{IV}{(14_{2})},~\Eqref[14sub3]{}{IV}{(14_{3})},
Chap.~IV\@.

By \Eqref[14sub1]{equations}{IV}{(14_{1})},
\begin{align*}
\frac{\theta}{2} &= 22°~30'; \\
\log \tan \frac{\theta}{2} &= 9.6172243; \\
\log \tan^{2} \frac{\theta}{2} &= 9.2344486 = \log k_{0} = \log \sin \theta_{0}; \\
\theta_{0} &= 9°~52'~45''.41; \\
\log \tan \frac{\theta_{0}}{2} &= 8.9366506; \\
\log \tan^{2} \frac{\theta_{0}}{2} &= 7.8733012 = \log k_{00} = \log \sin \theta_{00}; \\
\theta_{00} &= 0°~25'~40''.7; \\
\log \tan^{2} \frac{\theta_{00}}{2} &= 5.144552 = \log k_{03}.
\end{align*}
%% -----File: 109.png---Folio 103-------

By \Eqref[14sub2]{equations}{IV}{(14_{2})},
\[
\begin{array}{@{}r@{}l@{}}
\phi &{}= 30° \\
\log \tan \phi &{}= 9.761439 \\
\log \cos \theta &{}= 9.849485 \\
\cline{1-2}
\llap{$\log \tan (\phi_{1} - \phi)$} &{}= 9.610924 \\
\phi_{1} - \phi &{}= \rlap{$22°\ 12'\ 27''.56$} \\
\phi_{1} &{}= \rlap{$52°\ 12'\ 27''.56$} \\
\end{array}
\]
%
\[
\begin{array}{@{}r@{}l@{}}
\log \tan \phi_{1} &{}= 0.110438 \\
\log \cos \theta_{0} &{}= 9.993512 \\
\cline{1-2}
\llap{$\log \tan (\phi_{2} - \phi_{1})$} &{}= 0.103949 \\
\phi_{2} - \phi_{1} &{}= \rlap{$51°\ 47'\ 32''.59$} \\
\end{array}
\]
%
\[
\begin{array}{@{}r@{}l@{}}
\phi_{2} &{}= \rlap{$104°\ 0'\ 0''.15$} \\
\log \tan \phi_{2} &{}= 0.603228 \\
\log \cos \theta_{00} &{}= 9.999988 \\
\cline{1-2}
%
\llap{$\log \tan (\phi_{3} - \phi_{2})$} &{}= 0.603216 \\
\phi_{3} - \phi_{2} &{}= \rlap{$104°\ 0'\ 1''.5$} \\
& \\
%
\phi_{3} &{}= \rlap{$208°\ 0'\ 1''.65$}
\end{array}
\]

Since $\dfrac{\phi_{2}}{4} = 26°\ 0'\ 0''.04$ and $\dfrac{\phi_{3}}{8} = 26°\ 0'\ 0''.21$,
we need not calculate~$\phi_{4}$.
\[
\frac{\phi_{3}}{8} = 93600''.21.
\]

Reducing this to radians, we have
\[
\log \frac{\phi_{3}}{8} = 9.656852.
\]
%% -----File: 110.png---Folio 104-------

Substituting in \Eqref{eq.}{IV}{(23)}, Chap.~IV, we have, since $\cos \theta_{03} = 1$,
\[
\begin{array}{@{}r@{}c@{}l@{}}
\llap{$\ac$} \log \cos \theta &{}={}& 0.150515 \\
\log \cos \theta_{0}  &{}={}& 9.993512 \\
\log \cos \theta_{00} &{}={}& 9.999988 \\
\cline{1-3}
%
&& 0.144014 \\
&& 0.072007
  \rlap{${} = \log\sqrt{\dfrac{\cos \theta_{0} \cos \theta_{00}}{\cos \theta}}$} \\
\log \dfrac{\phi_{3}}{8} &{}={}& 9.656852 \\
\cline{3-3}
%
\log u &{}={}& 9.728859 \\
u &{}={}& 0.535623\rlap{,\quad\text{\textit{Ans.}}}
\end{array}
\]

When $\theta = \sin^{-1} k > 45°$. %[** PP: Scan unclear]

\Example. Given $k = \sin 75°$, $\tan \phi = \sqrt{\dfrac{2}{\sqrt{3}}}$. To find~$F(k, \phi)$.

\emph{First Method. Bisected Amplitudes.}

By \Eqref{equations}{IV}{(24)} and~\Eqref{}{IV}{(25)}, Chap.~IV, we get
\begin{align*}
\PadTo[l]{\phi_{\frac{1}{32}}}{\phi}
  &= 47°\ \Z3'\ 30''.91,                   &       & \\
%
\PadTo[l]{\phi_{\frac{1}{32}}}{\phi_{\frac{1}{2}}}
  &= 25°\ 36'\ \Z5''.64, &
\PadTo[l]{\beta_{04}}{\beta} &= 45°; \\
%
\PadTo[l]{\phi_{\frac{1}{32}}}{\phi_{\frac{1}{4}}}
  &= 13°\ \Z6'\ 30''.98, &
\PadTo[l]{\beta_{04}}{\beta_{0}}  &= 24°\ 40'\ 10''.94; \\
%
\PadTo[l]{\phi_{\frac{1}{32}}}{\phi_{\frac{1}{8}}}
  &= \Z6°\ 35'\ 40''.74, &
\beta_{00} &= 12°\ 39'\ 15''.83; \\
%
\PadTo[l]{\phi_{\frac{1}{32}}}{\phi_{\frac{1}{16}}}
  &= \Z3°\ 18'\ \Z8''.75, &
\beta_{03} &= \Z6°\ 22'\ \Z8''.40; \\
%
\phi_{\frac{1}{32}} &= \Z1°\ 39'\ \Z7''.43, &
\beta_{04} &= {}
\end{align*}

Substituting in \Eqref{equation}{IV}{(26)}, Chap.~IV, we have
\begin{align*}
F(k, \phi)
  &= 32 × 1°\ 39'\ 7''.43 \\
  &= 52°\ 51'\ 58''.03 \\
  &= 0.9226878.\quad \text{\textit{Ans.}}
\end{align*}
%% -----File: 111.png---Folio 105-------

\emph{Second Method.} \Eqref{Equation}{IV}{(29)}, Chap.~IV.

From \Eqref[18sub3]{equations}{IV}{(18_{3})}, Chap.~IV, we have
\[
\begin{array}{r@{}l@{}l@{\;}c}
&&& \log \\
k       &{}= \cos \eta &\begin{aligned}{}= \cosP 15°\ \Z0'\ \Z0''.00\end{aligned} & 9.9849438 \\
k'      &{}= \sin \eta &\begin{aligned}{}= \sinP 15°\ \Z0'\ \Z0''.00\end{aligned} & 9.4129962 \\
k_{0}'  &{}= \biggl\{\begin{aligned}
           &\tan^{2} \tfrac{1}{2} \eta \\
           &\sin \eta_{0}
         \end{aligned} &
         \begin{aligned}
           {}= \tan^{2} \Z7°\ 30'\ \Z0''.00 \\
           {}= \sinP    \Z0°\ 59'\ 35''.25
         \end{aligned}\;\biggr\} &
         8.2388582 \\
k_{1}   &{}= \cos \eta_{0} &\begin{aligned}{}= \cosP \Z0°\ 59'\ 35''.25\end{aligned} & 9.9999348 \\
k'_{00} &{}= \biggl\{\begin{aligned}
           &\tan^{2} \tfrac{1}{2} \eta_{0} \\
           &\sin \eta_{00}
         \end{aligned}&
         \begin{aligned}
           {}= \tan^{2} \Z0°\ 29'\ 47''.62 \\
           {}= \sinP \Z0°\ \Z0'\ 15''.49
         \end{aligned}\;\biggr\} &
         5.8757219 \\
k_{2} &{}= \cos \eta_{00} &\begin{aligned}{}= \cosP \Z0°\ \Z0'\ 15''.49\end{aligned} &  0.0000000 \\
k'_{03} &{}= \left(\tfrac{1}{2} k'_{00}\right)^{2} && 1.1493838
\end{array}
\]

From \Eqref[18sub2]{equations}{IV}{(18_{2})}, Chap.~IV, we get
\begin{align*}
\phi &= 47°\ 3'\ 30''.95; \\
2 \phi_{0} - \phi &= 45°; \\
\phi_{0} &= 46°\ 1'\ 45''.475; \\
\phi_{02} &= 46°\ 1'\ 29''.41; \\
\phi_{03} &= 46°\ 1'\ 29''.41; \\
45° + \tfrac{1}{2} \phi_{3} &= 68°\ 0'\ 44''.705.
\end{align*}

Substituting these values in \Eqref{eq.}{IV}{(29)}, Chap.~IV, we get
\begin{align*}
F(k, \phi)
  &= \sqrt{\frac{k_{1}}{k}} · \frac{1}{M} · \log \tan 68°\ 0'\ 44''.705 \\
  &= 0.9226877.\quad\text{\textit{Ans.}}
\end{align*}

\emph{Third Method.} \Eqref{Equation}{IV}{\DPtypo{(23)^*}{(23)}}, Chap.~IV\@.
%% -----File: 112.png---Folio 106-------

From \Eqref[14sub1]{equations}{IV}{(14_{1})}, Chap.~IV, we have
\[
\begin{array}{r@{}l@{}l@{\;}c}
&&& \log \\ % [** Moving to prev. line, cf. prev. page]
k       &{}= \sin \theta &\begin{aligned}{}= \sinP 75°\ \Z0'\ \Z0''\phantom{.00}\end{aligned}  & 9.9849438 \\
k'      &{}= \cos \theta &\begin{aligned}{}= \cosP 75°\phantom{\ 00'\ 00''.00}\end{aligned}  & 9.4129962 \\
k_{0}   &{}= \biggl\{\begin{aligned}
           &\tan^{2} \tfrac{1}{2} \theta \\
           &\sin \theta_{0}
         \end{aligned} &
         \begin{aligned}
           &{}= \tan^{2} 37°\ 30' \\
           &{}= \sinP 36°\ \Z4'\ 16''.47
         \end{aligned}\;\biggr\} &
         9.7699610 \\
k_{1}'  &{}= \cos \theta_{0} && 9.9075648 \\
k_{02}  &{}= \biggl\{\begin{aligned}
           &\tan^{2} \tfrac{1}{2} \theta_{0} \\
           &\sin \theta_{00}
         \end{aligned} &
         \begin{aligned}
           &{}= \tan^{2} 18°\ \Z2'\ 8''.235 \\
           &{}= \sinP \Z6°\ \Z5'\ 9''.38
       \end{aligned}\;\biggr\} &
       9.0253880 \\
k_{2}'  &{}= \cos \theta_{00} && 9.9975452 \\
k_{03}  &{}= \biggl\{\begin{aligned}
           &\tan^{2} \tfrac{1}{2} \theta_{00} \\
           &\sin \theta_{03}
         \end{aligned} &
         \begin{aligned}
           &{}= \tan^{2} \Z3°\ \Z2'\ 34''.69 \\
           &{}= \sinP \phantom{00°}\ \Z9'\ 42''.90
       \end{aligned}\;\biggr\} &
       7.4511672 \\
k'_{3} &{}= \cos \theta_{03} && 9.9999982 \\
k_{04} &{}= \left(\tfrac{1}{2} k_{03}\right)^{2} && 4.3002761 \\
k_{4}' &{}={} && 0.0000000
\end{array}
\]

From \Eqref[14sub2]{equations}{IV}{(14_{2})}, Chap.~IV, we have
\begin{align*}
\phi &= \Z47°\ \Z3'\ 30''.94; \\
\phi_{1} &= \Z62°\ 36'\ \Z3''.10; \\
\phi_{2} &= 119°\ 55'\ 47''.67; \\
\phi_{3} &= 240°\ \Z0'\ \Z0''.19; \\
\phi_{4} &= 480°\ \Z0'\ \Z0''.
\end{align*}
Therefore the limit of $\phi$, $\dfrac{\phi_{1}}{2}$, $\dfrac{\phi_{2}}{4}$, or~$\dfrac{\phi_{n}}{2^{n}}$ is $30° = \dfrac{\pi}{6}$.

Substituting these values in \Eqref{eq.}{IV}{\DPtypo{(23)^*}{(23)}}, Chap.~IV, we have
\begin{align*}
F(k, \phi)
  &= \sqrt{\frac{k_{1}' k_{2}' k_{3}' k_{4}'}{k'}} · \frac{\pi}{6} \\
  &= 0.9226874.\quad\text{\textit{Ans.}}
\end{align*}

\Example. Given $\phi = 30°$, $k = \sin 89°$. Find~$u$.

Method of \Eqref{eq.}{IV}{(28)}, Chap.~IV\@.
%% -----File: 113.png---Folio 107-------

From \Eqref[18sub1]{eqs.}{IV}{(18_{1})} we find
\[
k_{1} = \sin \theta_{1}\quad \text{and}\quad
\tan^{2} \tfrac{1}{2} \theta_{1} = k = \sin \theta,
\]
from which we find that $k_{1} = 1$ as far as seven decimal places.

From \Eqref[18sub2]{eqs.}{IV}{(18_{2})} we have
\[
\begin{array}{r@{}c@{}l@{}}
\sin \phi &{}={}& 9.6989700 \\
k &{}={}& 9.9999338 \\
\cline{3-3}
%
\sin (2 \phi_{0} - \phi) &{}={}& 9.6989038 \\
2 \phi_{0} - \phi &{}={}& \rlap{$29°\ 59'.69733$} \\
2 \phi_{0} &{}={}& \rlap{$59°\ 59'.69733$} \\
45° + \tfrac{1}{2} \phi_{0}\footnotemark &{}={}& \rlap{$59°\ 59'.92433$} \\
\log \left(45° + \tfrac{1}{2} \phi_{0}\right) &{}={}& 0.2385385
\end{array}
\]
\footnotetext{Since $k_{1} = 1$, $\phi_{00} = \phi_{0}$, and we need not carry the calculation further.}%

From \Eqref[18sub3]{eqs.}{IV}{(18_{3})}, Chap.~IV, we have
\[
k = \cos \eta = \cos 1°,\qquad \tfrac{1}{2} \eta = 30'.
\]

Substituting in \Eqref{eq.}{IV}{(28)}, Chap.~IV, we have
\[
\begin{array}{r@{}c@{}l@{}}
\ac \log \cos \tfrac{1}{2} \eta && 0.0000330 \\
\log \log \left(45° + \tfrac{1}{2} \phi_{0}\right) && 9.3775585 \\
\ac \log M && 0.3622157 \\
\cline{3-3}
%
\log F(k, \phi) &{}={}& 9.7398072 \\
F(k, \phi) &{}={}& 0.549297\PadTo[l]{0}{.}\rlap{\quad\text{\textit{Ans.}}}
\end{array}
\]

\Example. Given $\phi = 79°$, $k = 0.25882$. Find~$u$. \\
\null\hfill\textit{Ans.} $u = 0.39947$.

\Example. Given $\phi = 37°$, $k = 0.86603$. Find~$u$. \\
\null\hfill\textit{Ans.} $u = 0.68141$.
%% -----File: 114.png---Folio 108-------


\Chapter[Numerical Calculations. phi.]{XV}{Numerical Calculations. $\phi$.}

\Example. Given $u = 1.368407$, $\theta = 38°$. Find~$\phi$.

\emph{First Method.} From \Eqref{eqs.}{IX}{(46)} \Eqref[41st]{and}{IX}{(41)^*}, Chap.~IX, we have
\begin{align*}
u &= x \Theta^{2}(K), \\
\Delta \phi &= \sqrt{k'} \frac{\Theta_{1}(x)}{\Theta (x)}.
\end{align*}

From \Eqref{equations}{XII}{(5)}, Chap.~XII, and~\Eqref{}{IX}{(38)}, Chap.~IX, we
have
\[
\begin{array}{r@{}c@{}l@{}}
\log q &{}={}& 8.4734187 \\
\log \Theta^{2}(K) &{}={}& 0.0501955 \\
\log u &{}={}& 0.1362153 \\
\cline{3-3}
%
\log x &{}={}& 0.0860198 \\
x &{}={}& \rlap{$69°\ 50'\ 46''.12$}
\end{array}
\]

From \Eqref{equations}{IX}{(23)} and~\Eqref{}{IX}{(24)}, Chap.~IX, we get
\[
\begin{array}{r@{}c@{}l@{}}
\log \Theta_{1}(x) &{}={}& 9.9798368 \\
\log \Theta (x) &{}={}& 0.0192687 \\
\cline{3-3}
%
&& 9.9605681 \\
\log \sqrt{k'} &{}={}& 9.9482661 \\
\cline{3-3}
%
\log \Delta \phi &{}={}& 9.9088342 \rlap{${} = \log \sin \lambda$}
\end{array}
\]

But
\begin{align*}
k^{2} \sin^{2} \phi &= 1 - \Delta^{2} \phi, \\
k \sin \phi &= \cos \lambda;
\end{align*}
%% -----File: 115.png---Folio 109-------
whence
\[
\begin{array}{r@{}c@{}l@{}}
\log \cos \lambda &{}={}& 9.7675483 \\
\log k &{}={}& 9.7893420 \\
\cline{3-3}
%
\log \sin \phi &{}={}& 9.9782063 \\
\phi &{}={}& \rlap{$72°$.\quad\text{\textit{Ans.}}}
\end{array}
\]

\emph{Second Method.} From \Eqref{eq.}{VI}{(1)}, Chap.~VI\@.

From \Eqref[14sub1]{eqs.}{IV}{(14_{1})} Chap.~IV, we find
\[
\begin{array}{r@{}l@{}l@{\;}c}
&&& \log \\ %[** log on its own line, as on 113]
\PadTo[l]{k_{00}}{k_{0}}   &{}= \biggl\{\begin{aligned}
           &\tan^{2} \tfrac{1}{2} \theta\\
           &\sin \theta_{0}
         \end{aligned} &
         \begin{aligned}
           {}= \tan^{2} 19°\phantom{\ 48'.54569} \\
           {}= \sinP \Z6°\ 48'.54569
         \end{aligned}\;\biggr\} &
         9.0739438 \\
        & \phantom{{}={}}\quad \begin{aligned}\cos \theta_{0}\end{aligned} && 9.9969260 \\
%
k_{00}  &{}= \biggl\{\begin{aligned}
           &\tan^{2} \tfrac{1}{2} \theta_{0} \\
           &\sin \theta_{00}
         \end{aligned} &
         \begin{aligned}
           {}= \tan^{2} \Z3°\ 24'.2784\Z \\
           {}= \DPtypo{\phantom{\sinP}}{\sinP} \phantom{00°}\ 12'.16659
         \end{aligned}\;\biggr\} &
         7.5488952 \\
        &\phantom{{}={}}\quad \begin{aligned}\cos \theta_{00}\end{aligned} &&9.9999974 \\
%
k_{03}  &{}= \biggl\{\begin{aligned}
           &\tan^{2} \tfrac{1}{2} \theta_{00} \\
           &\sin \theta_{03}
         \end{aligned} &
         \begin{aligned}
           {}= \tan^{2} \phantom{00°}\ \Z6'.08329 \\
           {}
         \end{aligned}\;\biggr\} &
         4.4957316 \\
        &\phantom{{}={}}\quad \begin{aligned}\cos \theta_{03}\end{aligned} &&0.0000000
\end{array}
\]

Substituting these values in \Eqref{eq.}{VI}{(1)}, Chap.~VI, we have
\[
\begin{array}{r<{\quad}@{}l@{}}
\log \cos \theta_{0}  & 9.9969260 \\
\log \cos \theta_{00} & 9.9999974 \\
\cline{2-2}
%
& 9.9969234 \\
\log \sqrt{\cos \theta_{0} \cos \theta_{00}} & 9.9984617 \\
\ac \log \PadTo{\cos \theta_{0}}{\text{``}}
         \PadTo{\cos \theta_{00}}{\text{``}} & 0.0015383 \\
\log u & \Z.1362153 \\
\log \sqrt{\cos \theta} & 9.9482660 \\
\log 2^{3} & \Z.9030900\rlap{\footnotemark} \\
\ac \log \sqrt{\cos \theta_{0} \cos \theta_{00}} & 0.0015383 \\
\cline{2-2}
%
& 0.9891096 \\
& 2.2418773 \\
\cline{2-2}
%
\log \phi_{3}\addtocounter{footnote}{-1}\footnotemark & 2.7472323 \\
\phi_{3} & \rlap{$558°\ 46'.140$}
\end{array}
\]
\footnotetext{$n$~is taken equal to~$3$, because $\cos\DPtypo{}{\theta}_{03} = 1$.}
%% -----File: 116.png---Folio 110-------

Whence, by \Eqref{equations}{VI}{\DPtypo{(1)^*}{(1)}} of Chap.~VI, we get
\[
\begin{array}{r@{}c@{}l@{}}%[** PP: Re-aligning first group]
k_{03} \log &{}={}& 4.4957316 \\
\sin \phi_{3} && 9.5075232_{n} \\
\cline{3-3}
%
\sin (2 \phi_{2} - \phi_{3}) && 4\DPtypo{\,}{.}0032548_{n} \\
2 \phi_{2} - \phi_{3} &{}={}& -0'.00346 \\
\phi_{2} &{}={}& \rlap{$279°\ 23'.06827$}
\end{array}
\]
%
\[
\begin{array}{r@{}c@{}l@{}}
k_{00} \log &{}={}& 7.5488952 \\
\sin \phi_{2} && 9.9941484_{n} \\
\cline{3-3}
\sin (2 \phi_{1} - \phi_{2}) && 7.5430436_{n} \\
2 \phi_{1} - \phi_{2} &{}={}& -12'.0039 \\
\phi_{1} &{}={}& \rlap{$139°\ 35'.5321$}
\end{array}
\]
%
\[
\begin{array}{r@{}c@{}l@{}}
k_{0} \log &{}={}& 9.0739438 \\
\sin \phi_{1} && 9.8117249 \\
\cline{3-3}
\sin (2 \phi - \phi_{1}) && 8.8856687 \\
2 \phi - \phi_{1} &{}={}& \rlap{$\Z4°\ 24'.467$} \\
\phi &{}={}& \rlap{$71°\ 59'.9999$} \\
     &{}={}& \rlap{$72°$.\quad\text{\textit{Ans.}}}
\end{array}
\]

\Example. Given $u = 2.41569$, $\theta = 80°$. Find~$\phi$. \\
\null\hfill\textit{Ans.} $\phi = 82°$.

\Example. Given $u = 1.62530$, $k = \frac{1}{2}$. Find~$\phi$. \\
\null\hfill\textit{Ans.} $\phi = 87°$.
%% -----File: 117.png---Folio 111-------


\Chapter[Numerical Calculations. E(k, phi).]{XVI}
{Numerical Calculations. $E(k, \phi)$.}

\emph{First Method.} By Chap.~X, \Eqref{eqs.}{X}{(15)},~\Eqref{}{X}{(16)}, and~\Eqref{}{X}{(17)}.

\Example. Given $k = 0.9327$, $\phi = 80°$. Find $E(k, \phi)$.

By \Eqref{eq.}{X}{(15)}, Chap.~X,
\[
\begin{array}{l@{}c@{}l<{\qquad\qquad}l@{}c@{}l}
\phi              &{}={}&  80°;   &
    \gamma              &{}={}& 67°\ 44'.\Z; \\
\phi_{\frac{1}{2}}  &{}={}&  50° 43'.6, &
    \gamma_{\frac{1}{2}}  &{}={}& 46°\ 40'.4; \\
\phi_{\frac{1}{4}}  &{}={}&  27° 48'.5, &
    \gamma_{\frac{1}{4}}  &{}={}& 26°\ \Z0'.1; \\
\phi_{\frac{1}{8}}  &{}={}&  14° 16'.7, &
    \gamma_{\frac{1}{8}}  &{}={}& 13°\ 24'.0; \\
\phi_{\frac{1}{16}} &{}={}& \Z7° 11'.3, &
    \gamma_{\frac{1}{16}} &{}={}& \Z6°\ 45'.2; \\
\phi_{\frac{1}{32}} &{}={}& \Z3° 36'.0, &
    \llap{$\log \sin{}$} \gamma_{\frac{1}{32}} &{}={}& 8.77094; \\
\phi_{\frac{1}{32}} &{}={}& 0.062831.   && \\
\llap{$\therefore$ } \phi^{5}_{\frac{1}{32}} &{}<{}& 0.0000001. &&
\end{array}
\]
Whence, by \Eqref{eq.}{X}{(17)},
\[
\begin{array}{r@{}c@{}l@{}}
E(k, \phi_{\frac{1}{32}}) &{}={}& 0.06279\rlap{$4$} \\
\PadTo[l]{\sin \phi_{\frac{1}{16}}}{\sin \phi}
  \PadTo[l]{\sin^{2} \gamma_{\frac{1}{32}}}{\sin^{2} \gamma_{\frac{1}{2}}} &{}={}& 0.52116 \\
2\, \PadTo[l]{\sin \phi_{\frac{1}{16}}}{\sin \phi_{\frac{1}{2}}}
  \PadTo[l]{\sin^{2} \gamma_{\frac{1}{32}}}{\sin^{2} \gamma_{\frac{1}{4}}} &{}={}& 0.29757 \\
4\, \PadTo[l]{\sin \phi_{\frac{1}{16}}}{\sin \phi_{\frac{1}{4}}}
  \PadTo[l]{\sin^{2} \gamma_{\frac{1}{32}}}{\sin^{2} \gamma_{\frac{1}{8}}} &{}={}& 0.10023 \\
8\, \PadTo[l]{\sin \phi_{\frac{1}{16}}}{\sin \phi_{\frac{1}{8}}}
  \sin^{2} \gamma_{\frac{1}{16}} &{}={}& 0.02728 \\
16 \sin \phi_{\frac{1}{16}} \sin^{2} \gamma_{\frac{1}{32}} &{}={}& 0.00697 \\
\cline{3-3}
&& 0.95321
\end{array}
\]
Hence, by \Eqref{eq.}{X}{(16)},
\begin{align*}
E(k, \phi)
  &= 32E(k, \phi_{\frac{1}{32}}) - 0.95321 \\
  &= 2.0094 - 0.9532 = 1.0562.
\end{align*}
%% -----File: 118.png---Folio 112-------

\emph{Second Method.} By Chap.~X, \Eqref{eq.}{X}{(26)}.

\Example. Given $k = \sin 75°$, $\tan \phi = \sqrt{\dfrac{2}{\sqrt{3}}}$. Find~$E(k, \phi)$.

From \Eqref[14sub1]{eqs.}{IV}{(14_{1})}, Chap.~IV, we have
\begin{align*}%XXXX
k  &= \sin \theta = \sin 75°\ 0'\ 0'' & \log ={}& 9.9849438 \\
k' &= \cos \theta = \cos 75°                 && 9.4129962 \\
k_{0}
   &= \biggl\{
\begin{aligned}
  \tan^{2} \tfrac{1}{2} \theta &= \tan^{2} 37°\ 30' \\
  \PadTo[l]{\tan^2 \tfrac{1}{2} \theta}{\sin \theta_{0}} &= \PadTo[l]{\tan^2}{\sin}\ 36°\ \Z4'\ 16''.47
\end{aligned}
\biggr\} && 9.7699610 \\
k_{1}' &= \cos \theta_{0} && 9.9075648 \\
k_{02} &= \biggl\{
\begin{aligned}
  \tan^{2} \tfrac{1}{2} \theta_{0} &= \tan^{2} 18°\ \Z2'\ \Z8''.235 \\
  \PadTo[l]{\tan^2 \tfrac{1}{2} \theta_{0}}{\sin \theta_{00}} &= \PadTo[l]{\tan^2}{\sin}\ \Z6°\ \Z5'\ \Z9''.38
\end{aligned}
\biggr\} && 9.0253880 \\
k_{2}' &= \cos \theta_{00} && 9.9975452 \\
k_{03} &= \biggl\{
\begin{aligned}
  \tan^{2} \tfrac{1}{2} \theta_{00} &= \tan^{2} \Z3°\ \Z2'\ 34''.69 \\
  \PadTo[l]{\tan^2 \tfrac{1}{2} \theta_{00}}{\sin \theta_{03}} &= \PadTo[l]{\tan^2 22°}{\sin}\ \Z9'\ 42''.90
\end{aligned}
\biggr\} && 7.4511672 \\
k_{3}' &= \cos \theta_{03} && 9.9999982 \\
k_{04} &= \left(\tfrac{1}{2}k_{03}\right)^{2} && 4.3002761 \\
k_{4}' &={} && 0.0000000
\end{align*}

From \Eqref[14sub2]{eqs.}{IV}{(14_{2})}, Chap.~IV, we have
\begin{align*}
\phi &= \Z47°\ \Z3'\ 30''.94; \\
\phi_{1} &= \Z62°\ 36'\ \Z3''.10; \\
\phi_{2} &= 119°\ 55'\ 47''.67; \\
\phi_{3} &= 240°\ \Z0'\ \Z0''.19.
\end{align*}
%% -----File: 119.png---Folio 113-------

Applying \Eqref{eq.}{X}{(26)}, Chap.~X, we have
\[
\begin{array}{rr@{}l@{}c<{\qquad}@{}c@{}}
k^{2} & \log  ={}& 9.9698876 && \\
\ac 2 & & 9.6989700 && \\
\cline{3-3}
                 & & 9.6688576 && .4665064
\end{array}
\]
\[
\begin{array}{rr@{}l@{}c<{\qquad}@{}c@{}}
k_{0} & \phantom{\log  ={}} & 9.7699610 && \\
\ac 2 & & 9.6989700 && \\
\cline{3-3}
                 & & 9.1377886 && .1373373
\end{array}
\]
\[
\begin{array}{rr@{}l@{}c<{\qquad}@{}c@{}}
k_{00}& \phantom{\log  ={}} & 9.0253880 && \\
\ac 2 & & 9.6989700 && \\
\cline{3-3}
                 & & 7.8621466 && .0072802
\end{array}
\]
\[
\begin{array}{rr@{}l@{}c<{\qquad}@{}c@{}}
k_{03}& \phantom{\log  ={}} & 7.4511672 && \\
\ac 2 & & 9.6989700 && \\
\cline{3-3}
                 & & 5.0132838 && .0000103 \\
\cline{5-5}
                 & &           && .6111342
\end{array}
\]
\[
1 - .6111342 = 0.3888658.
\]

From \Eqref{eq.}{IV}{\DPtypo{(23)^*}{(23)}}, Chap.~IV, we find $F(k, \phi) = 0.9226874$.

Hence
\[
\begin{array}{r@{}l@{}c@{}} %[** PP: Re-aligning first equation]
F(k, \phi)
  \left[1 - \dfrac{k^{2}}{2} \left(1 + \dfrac{k_{0}}{2} + \dotsb\right)\right]
  &{}=& 0.3588016 \\[4pt]
\dfrac{k \sqrt{k_{0}}}{2} \sin \phi_{1}
  &{}=& 0.3290186 \\[4pt]
\dfrac{k \sqrt{k_{0}k_{00}}}{4} \sin \phi_{2}
  &{}=& 0.0522872 \\[4pt]
\dfrac{k \sqrt{k_{0}k_{02}k_{03}}}{8} \sin \phi_{3}
  &{}= -& 0.0013888 \\[4pt]
\dfrac{k \sqrt{k_{0} \dotsm k_{04}}}{16} \sin \phi_{4}
  &{}=& 0.0000010 \\[4pt]
\cline{3-3}
  && 0.3799180
\end{array}
\]
%% -----File: 120.png---Folio 114-------
Whence
\[
E(k, \phi) = 0.3588016 + 0.3799180 = 0.7387196.\quad\textit{Ans.}
\]

\Example. Given $k = \sin 75°$. Find~$E\left(k, \dfrac{\pi}{2}\right)$.

From Example~2, Chap.~XIII, we find
\[
\begin{array}{r@{}c@{}l@{}}
\log F\left(k, \dfrac{\pi}{2}\right) &{}={}& 0.4421761 \\
\log 0.3888658 &{}={}& 1.5897998 \\
\cline{3-3}
\log E \left(k, \dfrac{\pi}{2}\right) &{}={}& 0.0319759 \\
E\left(k, \frac{\pi}{2}\right) &{}={}& 1.076405\rlap{.\quad\text{\textit{Ans.}}}
\end{array}
\]

\Example. Given $k = \sin 30°$, $\phi = 81°$. Find~$E(k, \phi)$. \\
\null\hfill\textit{Ans.} $E(k, \phi) = 1.33124$.

\Example. Find~$E(\sin 80°, 55°)$. \\
\null\hfill\textit{Ans.} $0.82417$.

\Example. Find~$E\left(\sin 27°, \dfrac{\pi}{2}\right)$. \\
\null\hfill\textit{Ans.} $1.48642$.

\Example. Find~$E(\sin 19°, 27°)$. \\
\null\hfill\textit{Ans.} $0.46946$.
%% -----File: 121.png---Folio 115-------


\Chapter{XVII}{Applications.}

\Section{RECTIFICATION OF THE LEMNISCATE.}

\First{The} polar equation of the Lemniscate is $r = a \sqrt{\cos 2\theta}$,
referred to the centre as the origin. From this we get
\[
\frac{dr}{d\theta} = -\frac{a\sin 2\theta}{\sqrt{\cos 2\theta}};
\]
whence the length of the arc measured from the vertex to any
point whose co-ordinates are $r$~and~$\theta$
\begin{align*}
s &= \int \biggl\{\left(\frac{dr}{d\theta}\right)^{2} + r^{2} \biggr\}^{\frac{1}{2}} d\theta
   = a \int \biggl\{\frac{\sin^{2} 2\theta}{\cos 2\theta} + \cos 2\theta \biggr\}^{\frac{1}{2}} \DPtypo{}{d\theta} \\
  &= a \int \frac{d\theta}{\sqrt{\cos 2\theta}}
   = a \int \frac{d\theta}{\sqrt{1 - 2 \sin^{2} \theta}}.
\end{align*}

Let $\cos 2\theta = \cos^{2} \phi$, whence
\begin{align*} %[** PP: Aligning on equal signs]
s &= a \int \frac{\dfrac{d\theta}{d\phi}\, d\phi}{\cos \phi}
   = a \int \frac{\sin \phi\, d\phi}{\sqrt{1 - \cos^{4} \phi}} \\
  &= a \int_{0}^{\phi} \frac{d\phi}{\sqrt{1 + \cos^{2} \phi}}
   = \frac{a}{\sqrt{2}} \int_{0}^{\phi} \frac{d\phi}{\sqrt{1 - \frac{1}{2} \sin^{2} \phi}} \\
  &= \frac{a}{\sqrt{2}} F\left(\frac{1}{\sqrt{2}}, \phi\right).
\end{align*}
%% -----File: 122.png---Folio 116-------

Since $r = a \sqrt{\cos 2\theta} = a \cos \phi$, the angle~$\phi$ can be easily
constructed by describing upon the axis~$a$ of the Lemniscate a
semicircle, and then revolving the radius vector until it cuts
this semicircle. In the right-angled triangle of which this is one
side, and the axis the hypotenuse, $\phi$~is evidently the angle between
the axis and the revolved position of the radius vector.


\Section{RECTIFICATION OF THE ELLIPSE.}

Since the equation of the ellipse is $\dfrac{x^{2}}{a^{2}} + \dfrac{y^{2}}{b^{2}} = 1$, we can
assume $x = a\sin \phi$, $y = b\cos \phi$, so that $\phi$~is the complement
of the \emph{eccentric angle}. Hence
\begin{align*}
s &= \int \sqrt{dx^{2} + dy^{2}}
   = a \int d\phi \sqrt{1 - e^{2} \sin^{2} \phi} \\
  &= aE(e, \phi),
\end{align*}
in which $e$, the eccentricity of the ellipse, is the modulus of the
Elliptic Integral.

The length of the Elliptic Quadrant is
\[
s' = aE\left(e, \frac{\pi}{2}\right).
\]

\Example. The equation of an ellipse is
\[ %[** PP: Displaying
\frac{x^{2}}{16.81} + \dfrac{y^{2}}{16} = 1;
\]
required the length of an arc whose abscissas are $1.061162$ and
$4.100000$: of the quadrantal arc. \\
\null\hfill\textit{Ans.} $5.18912$; $6.36189$.


\Section{RECTIFICATION OF THE HYPERBOLA.}

On the curve of the hyperbola, construct a straight line
perpendicular to the axis~$x$, and at a distance from the centre
equal to the projection of~$b$, the transverse axis, upon the
asymptote, i.e.~equal to $\dfrac{b^{2}}{\sqrt{a^{2} + b^{2}}}$\DPtypo{}{.} Join the projection of the
%% -----File: 123.png---Folio 117-------
given point of the hyperbola on this line with the centre. The
angle which this joining line makes with the axis of~$x$ we will
call~$\phi$. If~$y$ is the ordinate of the point on the hyperbola, then
evidently
\[
y = \frac{b^{2} \tan \phi}{\sqrt{a^{2} + b^{2}}},
\]
and
\[
x = \frac{a}{\cos \phi} \sqrt{1 - \frac{a^{2} \sin^{2} \phi}{a^{2} + b^{2}}}
  = \frac{a}{\cos \phi} \sqrt{1 - \frac{1}{e^{2}} \sin^{2} \phi};
\]
whence
\begin{align*}
s &= \int \sqrt{\DPtypo{ax}{dx}^{2} + dy^{2}}
   = \frac{b^{2}}{c} \int_{0}^{\phi} \frac{d \phi}{\cos^{2} \phi \sqrt{1 - \dfrac{1}{e^{2}} \sin^{2} \phi}} \\
  &= \frac{b^{2}}{c} \int_{0}^{\phi} \frac{d \phi}{\cos^{2} \phi \sqrt{1 - k^{2} \sin^{2} \phi}}.
\end{align*}
But
\begin{gather*}
d(\tan \phi \sqrt{1 - k^{2} \sin^{2} \phi})
   = d \phi \sqrt{1 - k^{2} \sin^{2} \phi}
   + d \phi \frac{1 - k^{2}}{\sqrt{1 - e^{2} \sin^{2} \phi}} \\
%
- \frac{1 - k^{2}}{\cos^{2} \phi \sqrt{1 - e^{2} \sin^{2} \phi}}\, d \phi.
\end{gather*}
Consequently
\begin{align*}
s &= \frac{b^{2}}{c} \int_{0}^{\phi} \frac{d \phi}{\cos^{2} \phi \sqrt{1 - k^{2} \sin^{2} \phi}} \\
  &= \frac{b^{2}}{c} F(k, \phi) - cE(k, \phi) + c \tan \phi \Delta (k, \phi) \\
  &= \frac{b^{2}}{ae} F\left(\frac{1}{e}, \phi\right) - aeE\left(\frac{1}{e}, \phi\right) + ae \tan \phi \Delta \left(\frac{1}{e}, \phi\right).
\end{align*}
%% -----File: 124.png---Folio 118-------

\Example. Find the length of the arc of the hyperbola
\[ %[** PP: Displaying here, below]
\frac{x^{2}}{20.25} - \frac{y^{2}}{400} = 1
\]
from the vertex to the point whose ordinate
is~$\dfrac{40}{2.05} \tan 15°$. \\
\null\hfill\textit{Ans.} $5.231184$.

\Example. Find the length of the arc of the hyperbola
\[
\frac{x^{2}}{144} - \frac{y^{2}}{81} = 100
\]
from the vertex to the point whose ordinate
is~$0.6$. \\
\null\hfill\textit{Ans.} $0.6582$.


%%%%%%%%%%%%%%%%%%%%%%%%% GUTENBERG LICENSE %%%%%%%%%%%%%%%%%%%%%%%%%%

\cleardoublepage

\backmatter
\phantomsection
\pdfbookmark[-1]{Back Matter}{Back Matter}
\phantomsection
\pdfbookmark[0]{PG License}{Project Gutenberg License}
\fancyhead[C]{\textsc{LICENSING}}

\begin{PGtext}
End of the Project Gutenberg EBook of Elliptic Functions, by Arthur L. Baker

*** END OF THIS PROJECT GUTENBERG EBOOK ELLIPTIC FUNCTIONS ***

***** This file should be named 31076-pdf.pdf or 31076-pdf.zip *****
This and all associated files of various formats will be found in:
        http://www.gutenberg.org/3/1/0/7/31076/

Produced by Andrew D. Hwang, Brenda Lewis and the Online
Distributed Proofreading Team at http://www.pgdp.net (This
file was produced from images from the Cornell University
Library: Historical Mathematics Monographs collection.)


Updated editions will replace the previous one--the old editions
will be renamed.

Creating the works from public domain print editions means that no
one owns a United States copyright in these works, so the Foundation
(and you!) can copy and distribute it in the United States without
permission and without paying copyright royalties.  Special rules,
set forth in the General Terms of Use part of this license, apply to
copying and distributing Project Gutenberg-tm electronic works to
protect the PROJECT GUTENBERG-tm concept and trademark.  Project
Gutenberg is a registered trademark, and may not be used if you
charge for the eBooks, unless you receive specific permission.  If you
do not charge anything for copies of this eBook, complying with the
rules is very easy.  You may use this eBook for nearly any purpose
such as creation of derivative works, reports, performances and
research.  They may be modified and printed and given away--you may do
practically ANYTHING with public domain eBooks.  Redistribution is
subject to the trademark license, especially commercial
redistribution.



*** START: FULL LICENSE ***

THE FULL PROJECT GUTENBERG LICENSE
PLEASE READ THIS BEFORE YOU DISTRIBUTE OR USE THIS WORK

To protect the Project Gutenberg-tm mission of promoting the free
distribution of electronic works, by using or distributing this work
(or any other work associated in any way with the phrase "Project
Gutenberg"), you agree to comply with all the terms of the Full Project
Gutenberg-tm License (available with this file or online at
http://gutenberg.org/license).


Section 1.  General Terms of Use and Redistributing Project Gutenberg-tm
electronic works

1.A.  By reading or using any part of this Project Gutenberg-tm
electronic work, you indicate that you have read, understand, agree to
and accept all the terms of this license and intellectual property
(trademark/copyright) agreement.  If you do not agree to abide by all
the terms of this agreement, you must cease using and return or destroy
all copies of Project Gutenberg-tm electronic works in your possession.
If you paid a fee for obtaining a copy of or access to a Project
Gutenberg-tm electronic work and you do not agree to be bound by the
terms of this agreement, you may obtain a refund from the person or
entity to whom you paid the fee as set forth in paragraph 1.E.8.

1.B.  "Project Gutenberg" is a registered trademark.  It may only be
used on or associated in any way with an electronic work by people who
agree to be bound by the terms of this agreement.  There are a few
things that you can do with most Project Gutenberg-tm electronic works
even without complying with the full terms of this agreement.  See
paragraph 1.C below.  There are a lot of things you can do with Project
Gutenberg-tm electronic works if you follow the terms of this agreement
and help preserve free future access to Project Gutenberg-tm electronic
works.  See paragraph 1.E below.

1.C.  The Project Gutenberg Literary Archive Foundation ("the Foundation"
or PGLAF), owns a compilation copyright in the collection of Project
Gutenberg-tm electronic works.  Nearly all the individual works in the
collection are in the public domain in the United States.  If an
individual work is in the public domain in the United States and you are
located in the United States, we do not claim a right to prevent you from
copying, distributing, performing, displaying or creating derivative
works based on the work as long as all references to Project Gutenberg
are removed.  Of course, we hope that you will support the Project
Gutenberg-tm mission of promoting free access to electronic works by
freely sharing Project Gutenberg-tm works in compliance with the terms of
this agreement for keeping the Project Gutenberg-tm name associated with
the work.  You can easily comply with the terms of this agreement by
keeping this work in the same format with its attached full Project
Gutenberg-tm License when you share it without charge with others.

1.D.  The copyright laws of the place where you are located also govern
what you can do with this work.  Copyright laws in most countries are in
a constant state of change.  If you are outside the United States, check
the laws of your country in addition to the terms of this agreement
before downloading, copying, displaying, performing, distributing or
creating derivative works based on this work or any other Project
Gutenberg-tm work.  The Foundation makes no representations concerning
the copyright status of any work in any country outside the United
States.

1.E.  Unless you have removed all references to Project Gutenberg:

1.E.1.  The following sentence, with active links to, or other immediate
access to, the full Project Gutenberg-tm License must appear prominently
whenever any copy of a Project Gutenberg-tm work (any work on which the
phrase "Project Gutenberg" appears, or with which the phrase "Project
Gutenberg" is associated) is accessed, displayed, performed, viewed,
copied or distributed:

This eBook is for the use of anyone anywhere at no cost and with
almost no restrictions whatsoever.  You may copy it, give it away or
re-use it under the terms of the Project Gutenberg License included
with this eBook or online at www.gutenberg.org

1.E.2.  If an individual Project Gutenberg-tm electronic work is derived
from the public domain (does not contain a notice indicating that it is
posted with permission of the copyright holder), the work can be copied
and distributed to anyone in the United States without paying any fees
or charges.  If you are redistributing or providing access to a work
with the phrase "Project Gutenberg" associated with or appearing on the
work, you must comply either with the requirements of paragraphs 1.E.1
through 1.E.7 or obtain permission for the use of the work and the
Project Gutenberg-tm trademark as set forth in paragraphs 1.E.8 or
1.E.9.

1.E.3.  If an individual Project Gutenberg-tm electronic work is posted
with the permission of the copyright holder, your use and distribution
must comply with both paragraphs 1.E.1 through 1.E.7 and any additional
terms imposed by the copyright holder.  Additional terms will be linked
to the Project Gutenberg-tm License for all works posted with the
permission of the copyright holder found at the beginning of this work.

1.E.4.  Do not unlink or detach or remove the full Project Gutenberg-tm
License terms from this work, or any files containing a part of this
work or any other work associated with Project Gutenberg-tm.

1.E.5.  Do not copy, display, perform, distribute or redistribute this
electronic work, or any part of this electronic work, without
prominently displaying the sentence set forth in paragraph 1.E.1 with
active links or immediate access to the full terms of the Project
Gutenberg-tm License.

1.E.6.  You may convert to and distribute this work in any binary,
compressed, marked up, nonproprietary or proprietary form, including any
word processing or hypertext form.  However, if you provide access to or
distribute copies of a Project Gutenberg-tm work in a format other than
"Plain Vanilla ASCII" or other format used in the official version
posted on the official Project Gutenberg-tm web site (www.gutenberg.org),
you must, at no additional cost, fee or expense to the user, provide a
copy, a means of exporting a copy, or a means of obtaining a copy upon
request, of the work in its original "Plain Vanilla ASCII" or other
form.  Any alternate format must include the full Project Gutenberg-tm
License as specified in paragraph 1.E.1.

1.E.7.  Do not charge a fee for access to, viewing, displaying,
performing, copying or distributing any Project Gutenberg-tm works
unless you comply with paragraph 1.E.8 or 1.E.9.

1.E.8.  You may charge a reasonable fee for copies of or providing
access to or distributing Project Gutenberg-tm electronic works provided
that

- You pay a royalty fee of 20% of the gross profits you derive from
     the use of Project Gutenberg-tm works calculated using the method
     you already use to calculate your applicable taxes.  The fee is
     owed to the owner of the Project Gutenberg-tm trademark, but he
     has agreed to donate royalties under this paragraph to the
     Project Gutenberg Literary Archive Foundation.  Royalty payments
     must be paid within 60 days following each date on which you
     prepare (or are legally required to prepare) your periodic tax
     returns.  Royalty payments should be clearly marked as such and
     sent to the Project Gutenberg Literary Archive Foundation at the
     address specified in Section 4, "Information about donations to
     the Project Gutenberg Literary Archive Foundation."

- You provide a full refund of any money paid by a user who notifies
     you in writing (or by e-mail) within 30 days of receipt that s/he
     does not agree to the terms of the full Project Gutenberg-tm
     License.  You must require such a user to return or
     destroy all copies of the works possessed in a physical medium
     and discontinue all use of and all access to other copies of
     Project Gutenberg-tm works.

- You provide, in accordance with paragraph 1.F.3, a full refund of any
     money paid for a work or a replacement copy, if a defect in the
     electronic work is discovered and reported to you within 90 days
     of receipt of the work.

- You comply with all other terms of this agreement for free
     distribution of Project Gutenberg-tm works.

1.E.9.  If you wish to charge a fee or distribute a Project Gutenberg-tm
electronic work or group of works on different terms than are set
forth in this agreement, you must obtain permission in writing from
both the Project Gutenberg Literary Archive Foundation and Michael
Hart, the owner of the Project Gutenberg-tm trademark.  Contact the
Foundation as set forth in Section 3 below.

1.F.

1.F.1.  Project Gutenberg volunteers and employees expend considerable
effort to identify, do copyright research on, transcribe and proofread
public domain works in creating the Project Gutenberg-tm
collection.  Despite these efforts, Project Gutenberg-tm electronic
works, and the medium on which they may be stored, may contain
"Defects," such as, but not limited to, incomplete, inaccurate or
corrupt data, transcription errors, a copyright or other intellectual
property infringement, a defective or damaged disk or other medium, a
computer virus, or computer codes that damage or cannot be read by
your equipment.

1.F.2.  LIMITED WARRANTY, DISCLAIMER OF DAMAGES - Except for the "Right
of Replacement or Refund" described in paragraph 1.F.3, the Project
Gutenberg Literary Archive Foundation, the owner of the Project
Gutenberg-tm trademark, and any other party distributing a Project
Gutenberg-tm electronic work under this agreement, disclaim all
liability to you for damages, costs and expenses, including legal
fees.  YOU AGREE THAT YOU HAVE NO REMEDIES FOR NEGLIGENCE, STRICT
LIABILITY, BREACH OF WARRANTY OR BREACH OF CONTRACT EXCEPT THOSE
PROVIDED IN PARAGRAPH F3.  YOU AGREE THAT THE FOUNDATION, THE
TRADEMARK OWNER, AND ANY DISTRIBUTOR UNDER THIS AGREEMENT WILL NOT BE
LIABLE TO YOU FOR ACTUAL, DIRECT, INDIRECT, CONSEQUENTIAL, PUNITIVE OR
INCIDENTAL DAMAGES EVEN IF YOU GIVE NOTICE OF THE POSSIBILITY OF SUCH
DAMAGE.

1.F.3.  LIMITED RIGHT OF REPLACEMENT OR REFUND - If you discover a
defect in this electronic work within 90 days of receiving it, you can
receive a refund of the money (if any) you paid for it by sending a
written explanation to the person you received the work from.  If you
received the work on a physical medium, you must return the medium with
your written explanation.  The person or entity that provided you with
the defective work may elect to provide a replacement copy in lieu of a
refund.  If you received the work electronically, the person or entity
providing it to you may choose to give you a second opportunity to
receive the work electronically in lieu of a refund.  If the second copy
is also defective, you may demand a refund in writing without further
opportunities to fix the problem.

1.F.4.  Except for the limited right of replacement or refund set forth
in paragraph 1.F.3, this work is provided to you 'AS-IS' WITH NO OTHER
WARRANTIES OF ANY KIND, EXPRESS OR IMPLIED, INCLUDING BUT NOT LIMITED TO
WARRANTIES OF MERCHANTIBILITY OR FITNESS FOR ANY PURPOSE.

1.F.5.  Some states do not allow disclaimers of certain implied
warranties or the exclusion or limitation of certain types of damages.
If any disclaimer or limitation set forth in this agreement violates the
law of the state applicable to this agreement, the agreement shall be
interpreted to make the maximum disclaimer or limitation permitted by
the applicable state law.  The invalidity or unenforceability of any
provision of this agreement shall not void the remaining provisions.

1.F.6.  INDEMNITY - You agree to indemnify and hold the Foundation, the
trademark owner, any agent or employee of the Foundation, anyone
providing copies of Project Gutenberg-tm electronic works in accordance
with this agreement, and any volunteers associated with the production,
promotion and distribution of Project Gutenberg-tm electronic works,
harmless from all liability, costs and expenses, including legal fees,
that arise directly or indirectly from any of the following which you do
or cause to occur: (a) distribution of this or any Project Gutenberg-tm
work, (b) alteration, modification, or additions or deletions to any
Project Gutenberg-tm work, and (c) any Defect you cause.


Section  2.  Information about the Mission of Project Gutenberg-tm

Project Gutenberg-tm is synonymous with the free distribution of
electronic works in formats readable by the widest variety of computers
including obsolete, old, middle-aged and new computers.  It exists
because of the efforts of hundreds of volunteers and donations from
people in all walks of life.

Volunteers and financial support to provide volunteers with the
assistance they need, are critical to reaching Project Gutenberg-tm's
goals and ensuring that the Project Gutenberg-tm collection will
remain freely available for generations to come.  In 2001, the Project
Gutenberg Literary Archive Foundation was created to provide a secure
and permanent future for Project Gutenberg-tm and future generations.
To learn more about the Project Gutenberg Literary Archive Foundation
and how your efforts and donations can help, see Sections 3 and 4
and the Foundation web page at http://www.pglaf.org.


Section 3.  Information about the Project Gutenberg Literary Archive
Foundation

The Project Gutenberg Literary Archive Foundation is a non profit
501(c)(3) educational corporation organized under the laws of the
state of Mississippi and granted tax exempt status by the Internal
Revenue Service.  The Foundation's EIN or federal tax identification
number is 64-6221541.  Its 501(c)(3) letter is posted at
http://pglaf.org/fundraising.  Contributions to the Project Gutenberg
Literary Archive Foundation are tax deductible to the full extent
permitted by U.S. federal laws and your state's laws.

The Foundation's principal office is located at 4557 Melan Dr. S.
Fairbanks, AK, 99712., but its volunteers and employees are scattered
throughout numerous locations.  Its business office is located at
809 North 1500 West, Salt Lake City, UT 84116, (801) 596-1887, email
business@pglaf.org.  Email contact links and up to date contact
information can be found at the Foundation's web site and official
page at http://pglaf.org

For additional contact information:
     Dr. Gregory B. Newby
     Chief Executive and Director
     gbnewby@pglaf.org


Section 4.  Information about Donations to the Project Gutenberg
Literary Archive Foundation

Project Gutenberg-tm depends upon and cannot survive without wide
spread public support and donations to carry out its mission of
increasing the number of public domain and licensed works that can be
freely distributed in machine readable form accessible by the widest
array of equipment including outdated equipment.  Many small donations
($1 to $5,000) are particularly important to maintaining tax exempt
status with the IRS.

The Foundation is committed to complying with the laws regulating
charities and charitable donations in all 50 states of the United
States.  Compliance requirements are not uniform and it takes a
considerable effort, much paperwork and many fees to meet and keep up
with these requirements.  We do not solicit donations in locations
where we have not received written confirmation of compliance.  To
SEND DONATIONS or determine the status of compliance for any
particular state visit http://pglaf.org

While we cannot and do not solicit contributions from states where we
have not met the solicitation requirements, we know of no prohibition
against accepting unsolicited donations from donors in such states who
approach us with offers to donate.

International donations are gratefully accepted, but we cannot make
any statements concerning tax treatment of donations received from
outside the United States.  U.S. laws alone swamp our small staff.

Please check the Project Gutenberg Web pages for current donation
methods and addresses.  Donations are accepted in a number of other
ways including checks, online payments and credit card donations.
To donate, please visit: http://pglaf.org/donate


Section 5.  General Information About Project Gutenberg-tm electronic
works.

Professor Michael S. Hart is the originator of the Project Gutenberg-tm
concept of a library of electronic works that could be freely shared
with anyone.  For thirty years, he produced and distributed Project
Gutenberg-tm eBooks with only a loose network of volunteer support.


Project Gutenberg-tm eBooks are often created from several printed
editions, all of which are confirmed as Public Domain in the U.S.
unless a copyright notice is included.  Thus, we do not necessarily
keep eBooks in compliance with any particular paper edition.


Most people start at our Web site which has the main PG search facility:

     http://www.gutenberg.org

This Web site includes information about Project Gutenberg-tm,
including how to make donations to the Project Gutenberg Literary
Archive Foundation, how to help produce our new eBooks, and how to
subscribe to our email newsletter to hear about new eBooks.
\end{PGtext}

% %%%%%%%%%%%%%%%%%%%%%%%%%%%%%%%%%%%%%%%%%%%%%%%%%%%%%%%%%%%%%%%%%%%%%%% %
%                                                                         %
% End of the Project Gutenberg EBook of Elliptic Functions, by Arthur L. Baker
%                                                                         %
% *** END OF THIS PROJECT GUTENBERG EBOOK ELLIPTIC FUNCTIONS ***          %
%                                                                         %
% ***** This file should be named 31076-t.tex or 31076-t.zip *****        %
% This and all associated files of various formats will be found in:      %
%         http://www.gutenberg.org/3/1/0/7/31076/                         %
%                                                                         %
% %%%%%%%%%%%%%%%%%%%%%%%%%%%%%%%%%%%%%%%%%%%%%%%%%%%%%%%%%%%%%%%%%%%%%%% %

\end{document}
###
@ControlwordReplace = (
  ['\\Example', 'Example']
  );

@MathEnvironments = (
  ['\\begin{DPalign*}','\\end{DPalign*}','<DPALIGN>'],
  ['\\begin{DPgather*}','\\end{DPgather*}','<DPGATHER>']
  );

@ControlwordArguments = (
  ['\\hyperref', 0, 0, '', ''],
  ['\\IntroChapter', 1, 0, 'Introductory Chapter. ', '', 1, 1, '', ''],
  ['\\Chapter', 0, 0, '', '', 1, 1, 'Chapter ', '.  ', 1, 1, '', ''],
  ['\\Chapref', 1, 1, '', '', 1, 1, ' ', ''],
  ['\\Eqref', 0, 0, '', '', 1, 1, '', '', 1, 0, '', '', 1, 1, ' ', ''],
  ['\\Pageref', 1, 1, '', ' ', 1, 1, '', ''],
  ['\\DPtypo', 1, 0, '', '', 1, 1, '', ''],
  ['\\DPnote', 1, 0, '', ''],
  ['\\First', 1, 1, '', '']
  );
###
This is pdfTeXk, Version 3.141592-1.40.3 (Web2C 7.5.6) (format=pdflatex 2009.12.9)  25 JAN 2010 08:06
entering extended mode
 %&-line parsing enabled.
**31076-t.tex
(./31076-t.tex
LaTeX2e <2005/12/01>
Babel <v3.8h> and hyphenation patterns for english, usenglishmax, dumylang, noh
yphenation, arabic, farsi, croatian, ukrainian, russian, bulgarian, czech, slov
ak, danish, dutch, finnish, basque, french, german, ngerman, ibycus, greek, mon
ogreek, ancientgreek, hungarian, italian, latin, mongolian, norsk, icelandic, i
nterlingua, turkish, coptic, romanian, welsh, serbian, slovenian, estonian, esp
eranto, uppersorbian, indonesian, polish, portuguese, spanish, catalan, galicia
n, swedish, ukenglish, pinyin, loaded.
(/usr/share/texmf-texlive/tex/latex/base/book.cls
Document Class: book 2005/09/16 v1.4f Standard LaTeX document class
(/usr/share/texmf-texlive/tex/latex/base/leqno.clo
File: leqno.clo 1998/08/17 v1.1c Standard LaTeX option (left equation numbers)
) (/usr/share/texmf-texlive/tex/latex/base/bk12.clo
File: bk12.clo 2005/09/16 v1.4f Standard LaTeX file (size option)
)
\c@part=\count79
\c@chapter=\count80
\c@section=\count81
\c@subsection=\count82
\c@subsubsection=\count83
\c@paragraph=\count84
\c@subparagraph=\count85
\c@figure=\count86
\c@table=\count87
\abovecaptionskip=\skip41
\belowcaptionskip=\skip42
\bibindent=\dimen102
) (/usr/share/texmf-texlive/tex/latex/base/inputenc.sty
Package: inputenc 2006/05/05 v1.1b Input encoding file
\inpenc@prehook=\toks14
\inpenc@posthook=\toks15
(/usr/share/texmf-texlive/tex/latex/base/latin1.def
File: latin1.def 2006/05/05 v1.1b Input encoding file
)) (/usr/share/texmf-texlive/tex/latex/psnfss/mathpazo.sty
Package: mathpazo 2005/04/12 PSNFSS-v9.2a Palatino w/ Pazo Math (D.Puga, WaS)
\symupright=\mathgroup4
) (/usr/share/texmf-texlive/tex/latex/base/ifthen.sty
Package: ifthen 2001/05/26 v1.1c Standard LaTeX ifthen package (DPC)
) (/usr/share/texmf-texlive/tex/latex/amsmath/amsmath.sty
Package: amsmath 2000/07/18 v2.13 AMS math features
\@mathmargin=\skip43
For additional information on amsmath, use the `?' option.
(/usr/share/texmf-texlive/tex/latex/amsmath/amstext.sty
Package: amstext 2000/06/29 v2.01
(/usr/share/texmf-texlive/tex/latex/amsmath/amsgen.sty
File: amsgen.sty 1999/11/30 v2.0
\@emptytoks=\toks16
\ex@=\dimen103
)) (/usr/share/texmf-texlive/tex/latex/amsmath/amsbsy.sty
Package: amsbsy 1999/11/29 v1.2d
\pmbraise@=\dimen104
) (/usr/share/texmf-texlive/tex/latex/amsmath/amsopn.sty
Package: amsopn 1999/12/14 v2.01 operator names
)
\inf@bad=\count88
LaTeX Info: Redefining \frac on input line 211.
\uproot@=\count89
\leftroot@=\count90
LaTeX Info: Redefining \overline on input line 307.
\classnum@=\count91
\DOTSCASE@=\count92
LaTeX Info: Redefining \ldots on input line 379.
LaTeX Info: Redefining \dots on input line 382.
LaTeX Info: Redefining \cdots on input line 467.
\Mathstrutbox@=\box26
\strutbox@=\box27
\big@size=\dimen105
LaTeX Font Info:    Redeclaring font encoding OML on input line 567.
LaTeX Font Info:    Redeclaring font encoding OMS on input line 568.
\macc@depth=\count93
\c@MaxMatrixCols=\count94
\dotsspace@=\muskip10
\c@parentequation=\count95
\dspbrk@lvl=\count96
\tag@help=\toks17
\row@=\count97
\column@=\count98
\maxfields@=\count99
\andhelp@=\toks18
\eqnshift@=\dimen106
\alignsep@=\dimen107
\tagshift@=\dimen108
\tagwidth@=\dimen109
\totwidth@=\dimen110
\lineht@=\dimen111
\@envbody=\toks19
\multlinegap=\skip44
\multlinetaggap=\skip45
\mathdisplay@stack=\toks20
LaTeX Info: Redefining \[ on input line 2666.
LaTeX Info: Redefining \] on input line 2667.
) (/usr/share/texmf-texlive/tex/latex/amsfonts/amssymb.sty
Package: amssymb 2002/01/22 v2.2d
(/usr/share/texmf-texlive/tex/latex/amsfonts/amsfonts.sty
Package: amsfonts 2001/10/25 v2.2f
\symAMSa=\mathgroup5
\symAMSb=\mathgroup6
LaTeX Font Info:    Overwriting math alphabet `\mathfrak' in version `bold'
(Font)                  U/euf/m/n --> U/euf/b/n on input line 132.
)) (/usr/share/texmf-texlive/tex/latex/base/alltt.sty
Package: alltt 1997/06/16 v2.0g defines alltt environment
) (/usr/share/texmf-texlive/tex/latex/tools/array.sty
Package: array 2005/08/23 v2.4b Tabular extension package (FMi)
\col@sep=\dimen112
\extrarowheight=\dimen113
\NC@list=\toks21
\extratabsurround=\skip46
\backup@length=\skip47
) (/usr/share/texmf-texlive/tex/latex/footmisc/footmisc.sty
Package: footmisc 2005/03/17 v5.3d a miscellany of footnote facilities
\FN@temptoken=\toks22
\footnotemargin=\dimen114
\c@pp@next@reset=\count100
Package footmisc Info: Declaring symbol style bringhurst on input line 817.
Package footmisc Info: Declaring symbol style chicago on input line 818.
Package footmisc Info: Declaring symbol style wiley on input line 819.
Package footmisc Info: Declaring symbol style lamport-robust on input line 823.

Package footmisc Info: Declaring symbol style lamport* on input line 831.
Package footmisc Info: Declaring symbol style lamport*-robust on input line 840
.
) (/usr/share/texmf-texlive/tex/latex/bigfoot/perpage.sty
Package: perpage 2006/07/15 1.12 Reset/sort counters per page
\c@abspage=\count101
) (/usr/share/texmf-texlive/tex/latex/graphics/graphicx.sty
Package: graphicx 1999/02/16 v1.0f Enhanced LaTeX Graphics (DPC,SPQR)
(/usr/share/texmf-texlive/tex/latex/graphics/keyval.sty
Package: keyval 1999/03/16 v1.13 key=value parser (DPC)
\KV@toks@=\toks23
) (/usr/share/texmf-texlive/tex/latex/graphics/graphics.sty
Package: graphics 2006/02/20 v1.0o Standard LaTeX Graphics (DPC,SPQR)
(/usr/share/texmf-texlive/tex/latex/graphics/trig.sty
Package: trig 1999/03/16 v1.09 sin cos tan (DPC)
) (/etc/texmf/tex/latex/config/graphics.cfg
File: graphics.cfg 2007/01/18 v1.5 graphics configuration of teTeX/TeXLive
)
Package graphics Info: Driver file: pdftex.def on input line 90.
(/usr/share/texmf-texlive/tex/latex/pdftex-def/pdftex.def
File: pdftex.def 2007/01/08 v0.04d Graphics/color for pdfTeX
\Gread@gobject=\count102
))
\Gin@req@height=\dimen115
\Gin@req@width=\dimen116
) (/usr/share/texmf-texlive/tex/latex/wrapfig/wrapfig.sty
\wrapoverhang=\dimen117
\WF@size=\dimen118
\c@WF@wrappedlines=\count103
\WF@box=\box28
\WF@everypar=\toks24
Package: wrapfig 2003/01/31  v 3.6
) (/usr/share/texmf-texlive/tex/latex/tools/indentfirst.sty
Package: indentfirst 1995/11/23 v1.03 Indent first paragraph (DPC)
) (/usr/share/texmf-texlive/tex/latex/textcase/textcase.sty
Package: textcase 2004/10/07 v0.07 Text only upper/lower case changing (DPC)
) (/usr/share/texmf-texlive/tex/latex/tools/calc.sty
Package: calc 2005/08/06 v4.2 Infix arithmetic (KKT,FJ)
\calc@Acount=\count104
\calc@Bcount=\count105
\calc@Adimen=\dimen119
\calc@Bdimen=\dimen120
\calc@Askip=\skip48
\calc@Bskip=\skip49
LaTeX Info: Redefining \setlength on input line 75.
LaTeX Info: Redefining \addtolength on input line 76.
\calc@Ccount=\count106
\calc@Cskip=\skip50
) (/usr/share/texmf-texlive/tex/latex/base/fix-cm.sty
Package: fix-cm 2006/03/24 v1.1n fixes to LaTeX
(/usr/share/texmf-texlive/tex/latex/base/ts1enc.def
File: ts1enc.def 2001/06/05 v3.0e (jk/car/fm) Standard LaTeX file
)) (/usr/share/texmf-texlive/tex/latex/soul/soul.sty
Package: soul 2003/11/17 v2.4 letterspacing/underlining (mf)
\SOUL@word=\toks25
\SOUL@lasttoken=\toks26
\SOUL@cmds=\toks27
\SOUL@buffer=\toks28
\SOUL@token=\toks29
\SOUL@spaceskip=\skip51
\SOUL@ttwidth=\dimen121
\SOUL@uldp=\dimen122
\SOUL@ulht=\dimen123
) (/usr/share/texmf-texlive/tex/latex/fancyhdr/fancyhdr.sty
\fancy@headwidth=\skip52
\f@ncyO@elh=\skip53
\f@ncyO@erh=\skip54
\f@ncyO@olh=\skip55
\f@ncyO@orh=\skip56
\f@ncyO@elf=\skip57
\f@ncyO@erf=\skip58
\f@ncyO@olf=\skip59
\f@ncyO@orf=\skip60
) (/usr/share/texmf-texlive/tex/latex/geometry/geometry.sty
Package: geometry 2002/07/08 v3.2 Page Geometry
\Gm@cnth=\count107
\Gm@cntv=\count108
\c@Gm@tempcnt=\count109
\Gm@bindingoffset=\dimen124
\Gm@wd@mp=\dimen125
\Gm@odd@mp=\dimen126
\Gm@even@mp=\dimen127
\Gm@dimlist=\toks30
(/usr/share/texmf-texlive/tex/xelatex/xetexconfig/geometry.cfg)) (/usr/share/te
xmf-texlive/tex/latex/hyperref/hyperref.sty
Package: hyperref 2007/02/07 v6.75r Hypertext links for LaTeX
\@linkdim=\dimen128
\Hy@linkcounter=\count110
\Hy@pagecounter=\count111
(/usr/share/texmf-texlive/tex/latex/hyperref/pd1enc.def
File: pd1enc.def 2007/02/07 v6.75r Hyperref: PDFDocEncoding definition (HO)
) (/etc/texmf/tex/latex/config/hyperref.cfg
File: hyperref.cfg 2002/06/06 v1.2 hyperref configuration of TeXLive
) (/usr/share/texmf-texlive/tex/latex/oberdiek/kvoptions.sty
Package: kvoptions 2006/08/22 v2.4 Connects package keyval with LaTeX options (
HO)
)
Package hyperref Info: Option `hyperfootnotes' set `false' on input line 2238.
Package hyperref Info: Option `bookmarks' set `true' on input line 2238.
Package hyperref Info: Option `linktocpage' set `false' on input line 2238.
Package hyperref Info: Option `pdfdisplaydoctitle' set `true' on input line 223
8.
Package hyperref Info: Option `pdfpagelabels' set `true' on input line 2238.
Package hyperref Info: Option `bookmarksopen' set `true' on input line 2238.
Package hyperref Info: Option `colorlinks' set `true' on input line 2238.
Package hyperref Info: Hyper figures OFF on input line 2288.
Package hyperref Info: Link nesting OFF on input line 2293.
Package hyperref Info: Hyper index ON on input line 2296.
Package hyperref Info: Plain pages OFF on input line 2303.
Package hyperref Info: Backreferencing OFF on input line 2308.
Implicit mode ON; LaTeX internals redefined
Package hyperref Info: Bookmarks ON on input line 2444.
(/usr/share/texmf-texlive/tex/latex/ltxmisc/url.sty
\Urlmuskip=\muskip11
Package: url 2005/06/27  ver 3.2  Verb mode for urls, etc.
)
LaTeX Info: Redefining \url on input line 2599.
\Fld@menulength=\count112
\Field@Width=\dimen129
\Fld@charsize=\dimen130
\Choice@toks=\toks31
\Field@toks=\toks32
Package hyperref Info: Hyper figures OFF on input line 3102.
Package hyperref Info: Link nesting OFF on input line 3107.
Package hyperref Info: Hyper index ON on input line 3110.
Package hyperref Info: backreferencing OFF on input line 3117.
Package hyperref Info: Link coloring ON on input line 3120.
\Hy@abspage=\count113
\c@Item=\count114
)
*hyperref using driver hpdftex*
(/usr/share/texmf-texlive/tex/latex/hyperref/hpdftex.def
File: hpdftex.def 2007/02/07 v6.75r Hyperref driver for pdfTeX
\Fld@listcount=\count115
)
\c@pp@a@footnote=\count116
\TmpLen=\skip61
\DP@lign@no=\count117
\DP@lignb@dy=\toks33
(./31076-t.aux)
\openout1 = `31076-t.aux'.

LaTeX Font Info:    Checking defaults for OML/cmm/m/it on input line 571.
LaTeX Font Info:    ... okay on input line 571.
LaTeX Font Info:    Checking defaults for T1/cmr/m/n on input line 571.
LaTeX Font Info:    ... okay on input line 571.
LaTeX Font Info:    Checking defaults for OT1/cmr/m/n on input line 571.
LaTeX Font Info:    ... okay on input line 571.
LaTeX Font Info:    Checking defaults for OMS/cmsy/m/n on input line 571.
LaTeX Font Info:    ... okay on input line 571.
LaTeX Font Info:    Checking defaults for OMX/cmex/m/n on input line 571.
LaTeX Font Info:    ... okay on input line 571.
LaTeX Font Info:    Checking defaults for U/cmr/m/n on input line 571.
LaTeX Font Info:    ... okay on input line 571.
LaTeX Font Info:    Checking defaults for TS1/cmr/m/n on input line 571.
LaTeX Font Info:    ... okay on input line 571.
LaTeX Font Info:    Checking defaults for PD1/pdf/m/n on input line 571.
LaTeX Font Info:    ... okay on input line 571.
LaTeX Font Info:    Try loading font information for OT1+pplj on input line 571
.
(/usr/share/texmf-texlive/tex/latex/psnfss/ot1pplj.fd
File: ot1pplj.fd 2004/09/06 font definitions for OT1/pplj.
) (/usr/share/texmf/tex/context/base/supp-pdf.tex
[Loading MPS to PDF converter (version 2006.09.02).]
\scratchcounter=\count118
\scratchdimen=\dimen131
\scratchbox=\box29
\nofMPsegments=\count119
\nofMParguments=\count120
\everyMPshowfont=\toks34
\MPscratchCnt=\count121
\MPscratchDim=\dimen132
\MPnumerator=\count122
\everyMPtoPDFconversion=\toks35
)
-------------------- Geometry parameters
paper: class default
landscape: --
twocolumn: --
twoside: true
asymmetric: --
h-parts: 9.03374pt, 379.4175pt, 9.03375pt
v-parts: 6.55762pt, 513.5613pt, 9.83646pt
hmarginratio: 1:1
vmarginratio: 2:3
lines: --
heightrounded: --
bindingoffset: 0.0pt
truedimen: --
includehead: true
includefoot: true
includemp: --
driver: pdftex
-------------------- Page layout dimensions and switches
\paperwidth  397.48499pt
\paperheight 529.95537pt
\textwidth  379.4175pt
\textheight 451.6875pt
\oddsidemargin  -63.23625pt
\evensidemargin -63.23624pt
\topmargin  -65.71237pt
\headheight 12.0pt
\headsep    19.8738pt
\footskip   30.0pt
\marginparwidth 98.0pt
\marginparsep   7.0pt
\columnsep  10.0pt
\skip\footins  10.8pt plus 4.0pt minus 2.0pt
\hoffset 0.0pt
\voffset 0.0pt
\mag 1000
\@twosidetrue \@mparswitchtrue
(1in=72.27pt, 1cm=28.45pt)
-----------------------
(/usr/share/texmf-texlive/tex/latex/graphics/color.sty
Package: color 2005/11/14 v1.0j Standard LaTeX Color (DPC)
(/etc/texmf/tex/latex/config/color.cfg
File: color.cfg 2007/01/18 v1.5 color configuration of teTeX/TeXLive
)
Package color Info: Driver file: pdftex.def on input line 130.
)
Package hyperref Info: Link coloring ON on input line 571.
(/usr/share/texmf-texlive/tex/latex/hyperref/nameref.sty
Package: nameref 2006/12/27 v2.28 Cross-referencing by name of section
(/usr/share/texmf-texlive/tex/latex/oberdiek/refcount.sty
Package: refcount 2006/02/20 v3.0 Data extraction from references (HO)
)
\c@section@level=\count123
)
LaTeX Info: Redefining \ref on input line 571.
LaTeX Info: Redefining \pageref on input line 571.
(./31076-t.out) (./31076-t.out)
\@outlinefile=\write3
\openout3 = `31076-t.out'.

LaTeX Font Info:    Try loading font information for OT1+ppl on input line 609.

(/usr/share/texmf-texlive/tex/latex/psnfss/ot1ppl.fd
File: ot1ppl.fd 2001/06/04 font definitions for OT1/ppl.
)
LaTeX Font Info:    Try loading font information for OML+zplm on input line 609
.
(/usr/share/texmf-texlive/tex/latex/psnfss/omlzplm.fd
File: omlzplm.fd 2002/09/08 Fontinst v1.914 font definitions for OML/zplm.
)
LaTeX Font Info:    Try loading font information for OMS+zplm on input line 609
.
(/usr/share/texmf-texlive/tex/latex/psnfss/omszplm.fd
File: omszplm.fd 2002/09/08 Fontinst v1.914 font definitions for OMS/zplm.
)
LaTeX Font Info:    Try loading font information for OMX+zplm on input line 609
.
(/usr/share/texmf-texlive/tex/latex/psnfss/omxzplm.fd
File: omxzplm.fd 2002/09/08 Fontinst v1.914 font definitions for OMX/zplm.
)
LaTeX Font Info:    Try loading font information for OT1+zplm on input line 609
.
(/usr/share/texmf-texlive/tex/latex/psnfss/ot1zplm.fd
File: ot1zplm.fd 2002/09/08 Fontinst v1.914 font definitions for OT1/zplm.
)
LaTeX Font Info:    Font shape `U/msa/m/n' will be
(Font)              scaled to size 11.46208pt on input line 609.
LaTeX Font Info:    Font shape `U/msa/m/n' will be
(Font)              scaled to size 9.37807pt on input line 609.
LaTeX Font Info:    Font shape `U/msa/m/n' will be
(Font)              scaled to size 8.33606pt on input line 609.
LaTeX Font Info:    Font shape `U/msb/m/n' will be
(Font)              scaled to size 11.46208pt on input line 609.
LaTeX Font Info:    Font shape `U/msb/m/n' will be
(Font)              scaled to size 9.37807pt on input line 609.
LaTeX Font Info:    Font shape `U/msb/m/n' will be
(Font)              scaled to size 8.33606pt on input line 609.
[1

{/var/lib/texmf/fonts/map/pdftex/updmap/pdftex.map}]
LaTeX Font Info:    Font shape `OT1/pplj/bx/n' in size <14.4> not available
(Font)              Font shape `OT1/pplj/b/n' tried instead on input line 633.
[2

] [1

] [2

] [3

] [4]
LaTeX Font Info:    Font shape `OT1/pplj/bx/n' in size <24.88> not available
(Font)              Font shape `OT1/pplj/b/n' tried instead on input line 776.
(./31076-t.toc
LaTeX Font Info:    Font shape `U/msa/m/n' will be
(Font)              scaled to size 11.40997pt on input line 15.
LaTeX Font Info:    Font shape `U/msa/m/n' will be
(Font)              scaled to size 6.25204pt on input line 15.
LaTeX Font Info:    Font shape `U/msb/m/n' will be
(Font)              scaled to size 11.40997pt on input line 15.
LaTeX Font Info:    Font shape `U/msb/m/n' will be
(Font)              scaled to size 6.25204pt on input line 15.
)
\tf@toc=\write4
\openout4 = `31076-t.toc'.

[5


]
LaTeX Font Info:    Font shape `OT1/pplj/bx/n' in size <17.28> not available
(Font)              Font shape `OT1/pplj/b/n' tried instead on input line 834.
LaTeX Font Info:    Font shape `U/msa/m/n' will be
(Font)              scaled to size 18.00587pt on input line 834.
LaTeX Font Info:    Font shape `U/msa/m/n' will be
(Font)              scaled to size 12.50409pt on input line 834.
LaTeX Font Info:    Font shape `U/msa/m/n' will be
(Font)              scaled to size 10.42007pt on input line 834.
LaTeX Font Info:    Font shape `U/msb/m/n' will be
(Font)              scaled to size 18.00587pt on input line 834.
LaTeX Font Info:    Font shape `U/msb/m/n' will be
(Font)              scaled to size 12.50409pt on input line 834.
LaTeX Font Info:    Font shape `U/msb/m/n' will be
(Font)              scaled to size 10.42007pt on input line 834.
LaTeX Font Info:    Font shape `U/msa/m/n' will be
(Font)              scaled to size 7.91925pt on input line 835.
LaTeX Font Info:    Font shape `U/msb/m/n' will be
(Font)              scaled to size 7.91925pt on input line 835.
LaTeX Font Info:    Calculating math sizes for size <7.6> on input line 835.
LaTeX Font Info:    Font shape `U/msa/m/n' will be
(Font)              scaled to size 6.01859pt on input line 835.
LaTeX Font Info:    Font shape `U/msa/m/n' will be
(Font)              scaled to size 4.75159pt on input line 835.
LaTeX Font Info:    Font shape `U/msb/m/n' will be
(Font)              scaled to size 6.01859pt on input line 835.
LaTeX Font Info:    Font shape `U/msb/m/n' will be
(Font)              scaled to size 4.75159pt on input line 835.
[1


] [2]
LaTeX Font Info:    Font shape `U/msa/m/n' will be
(Font)              scaled to size 7.29405pt on input line 915.
LaTeX Font Info:    Font shape `U/msa/m/n' will be
(Font)              scaled to size 5.21004pt on input line 915.
LaTeX Font Info:    Font shape `U/msb/m/n' will be
(Font)              scaled to size 7.29405pt on input line 915.
LaTeX Font Info:    Font shape `U/msb/m/n' will be
(Font)              scaled to size 5.21004pt on input line 915.
[3] [4

]
Overfull \hbox (0.64647pt too wide) in paragraph at lines 963--966
\OT1/pplj/m/n/12 tran-scen-den-tal func-tions which de-pend upon these in-te-gr
als, and which
 []

[5] [6] [7] [8] [9] [10] [11]
Overfull \hbox (1.20306pt too wide) in paragraph at lines 1287--1287
[]
 []


Overfull \hbox (1.20306pt too wide) in alignment at lines 1287--1287
[] []
 []

[12] [13] [14] [15]
Overfull \hbox (0.63611pt too wide) in paragraph at lines 1425--1427
[]\OT1/pplj/m/n/12 The quan-tity $\OML/zplm/m/it/11 k$ \OT1/pplj/m/n/12 is call
ed the \OT1/pplj/m/it/12 mod-u-lus\OT1/pplj/m/n/12 , and the ex-pres-sion $[]$
 []

[16

] <./images/023a.pdf, id=316, 76.285pt x 106.3975pt>
File: ./images/023a.pdf Graphic file (type pdf)
<use ./images/023a.pdf> [17 <./images/023a.pdf>] <./images/024a.pdf, id=335, 82
.3075pt x 126.4725pt>
File: ./images/024a.pdf Graphic file (type pdf)
<use ./images/024a.pdf>
Overfull \hbox (1.00354pt too wide) in paragraph at lines 1532--1533
[][]
 []

<./images/025a.pdf, id=336, 146.5475pt x 157.58875pt>
File: ./images/025a.pdf Graphic file (type pdf)
<use ./images/025a.pdf> [18 <./images/024a.pdf>] [19 <./images/025a.pdf>] [20]
[21] [22] [23] [24

] [25] [26] [27] [28] [29] [30] [31] [32] <./images/036a.pdf, id=500, 190.7125p
t x 107.40125pt>
File: ./images/036a.pdf Graphic file (type pdf)
<use ./images/036a.pdf> [33

 <./images/036a.pdf>] [34] [35] [36] [37] [38] [39] [40] [41] [42] [43] [44] [4
5] [46] [47] [48] [49] [50

] [51] [52]
LaTeX Font Info:    Font shape `U/msa/m/n' will be
(Font)              scaled to size 15.0049pt on input line 2931.
LaTeX Font Info:    Font shape `U/msb/m/n' will be
(Font)              scaled to size 15.0049pt on input line 2931.
[53

] [54] [55] [56

] [57] [58] [59] [60] [61] [62] [63] [64] [65] [66] [67] [68] [69] [70] [71

] [72] [73] [74

] [75] [76] [77] [78] [79] [80] [81] [82] [83] [84] [85] [86

] [87] [88] [89] [90] [91] [92] [93] [94] [95] [96

] [97] [98] [99] [100] [101

] [102] [103] [104] [105

] [106] [107] [108] [109] [110] [111

] [112] [113] [114] [115] [116] [117] [118] [119

] [120] [121] [122] [123

] [124] [125] [126] [127] [128

] [129] [130] [131] [132


] [133] [134] [135] [136] [137] [138] [139] [140] (./31076-t.aux)

 *File List*
    book.cls    2005/09/16 v1.4f Standard LaTeX document class
   leqno.clo    1998/08/17 v1.1c Standard LaTeX option (left equation numbers)
    bk12.clo    2005/09/16 v1.4f Standard LaTeX file (size option)
inputenc.sty    2006/05/05 v1.1b Input encoding file
  latin1.def    2006/05/05 v1.1b Input encoding file
mathpazo.sty    2005/04/12 PSNFSS-v9.2a Palatino w/ Pazo Math (D.Puga, WaS)
  ifthen.sty    2001/05/26 v1.1c Standard LaTeX ifthen package (DPC)
 amsmath.sty    2000/07/18 v2.13 AMS math features
 amstext.sty    2000/06/29 v2.01
  amsgen.sty    1999/11/30 v2.0
  amsbsy.sty    1999/11/29 v1.2d
  amsopn.sty    1999/12/14 v2.01 operator names
 amssymb.sty    2002/01/22 v2.2d
amsfonts.sty    2001/10/25 v2.2f
   alltt.sty    1997/06/16 v2.0g defines alltt environment
   array.sty    2005/08/23 v2.4b Tabular extension package (FMi)
footmisc.sty    2005/03/17 v5.3d a miscellany of footnote facilities
 perpage.sty    2006/07/15 1.12 Reset/sort counters per page
graphicx.sty    1999/02/16 v1.0f Enhanced LaTeX Graphics (DPC,SPQR)
  keyval.sty    1999/03/16 v1.13 key=value parser (DPC)
graphics.sty    2006/02/20 v1.0o Standard LaTeX Graphics (DPC,SPQR)
    trig.sty    1999/03/16 v1.09 sin cos tan (DPC)
graphics.cfg    2007/01/18 v1.5 graphics configuration of teTeX/TeXLive
  pdftex.def    2007/01/08 v0.04d Graphics/color for pdfTeX
 wrapfig.sty    2003/01/31  v 3.6
indentfirst.sty    1995/11/23 v1.03 Indent first paragraph (DPC)
textcase.sty    2004/10/07 v0.07 Text only upper/lower case changing (DPC)
    calc.sty    2005/08/06 v4.2 Infix arithmetic (KKT,FJ)
  fix-cm.sty    2006/03/24 v1.1n fixes to LaTeX
  ts1enc.def    2001/06/05 v3.0e (jk/car/fm) Standard LaTeX file
    soul.sty    2003/11/17 v2.4 letterspacing/underlining (mf)
fancyhdr.sty
geometry.sty    2002/07/08 v3.2 Page Geometry
geometry.cfg
hyperref.sty    2007/02/07 v6.75r Hypertext links for LaTeX
  pd1enc.def    2007/02/07 v6.75r Hyperref: PDFDocEncoding definition (HO)
hyperref.cfg    2002/06/06 v1.2 hyperref configuration of TeXLive
kvoptions.sty    2006/08/22 v2.4 Connects package keyval with LaTeX options (HO
)
     url.sty    2005/06/27  ver 3.2  Verb mode for urls, etc.
 hpdftex.def    2007/02/07 v6.75r Hyperref driver for pdfTeX
 ot1pplj.fd    2004/09/06 font definitions for OT1/pplj.
supp-pdf.tex
   color.sty    2005/11/14 v1.0j Standard LaTeX Color (DPC)
   color.cfg    2007/01/18 v1.5 color configuration of teTeX/TeXLive
 nameref.sty    2006/12/27 v2.28 Cross-referencing by name of section
refcount.sty    2006/02/20 v3.0 Data extraction from references (HO)
 31076-t.out
 31076-t.out
  ot1ppl.fd    2001/06/04 font definitions for OT1/ppl.
 omlzplm.fd    2002/09/08 Fontinst v1.914 font definitions for OML/zplm.
 omszplm.fd    2002/09/08 Fontinst v1.914 font definitions for OMS/zplm.
 omxzplm.fd    2002/09/08 Fontinst v1.914 font definitions for OMX/zplm.
 ot1zplm.fd    2002/09/08 Fontinst v1.914 font definitions for OT1/zplm.
./images/023a.pdf
./images/024a.pdf
./images/025a.pdf
./images/036a.pdf
 ***********

 )
Here is how much of TeX's memory you used:
 6525 strings out of 94074
 86973 string characters out of 1165153
 231131 words of memory out of 1500000
 8960 multiletter control sequences out of 10000+50000
 85938 words of font info for 221 fonts, out of 1200000 for 2000
 645 hyphenation exceptions out of 8191
 43i,26n,43p,1270b,483s stack positions out of 5000i,500n,6000p,200000b,5000s
{/usr/share/texmf-texlive/fonts/enc/dvips/base/8r.enc}</usr/share/texmf-texli
ve/fonts/type1/bluesky/cm/cmex10.pfb></usr/share/texmf-texlive/fonts/type1/blue
sky/cm/cmmi10.pfb></usr/share/texmf-texlive/fonts/type1/bluesky/cm/cmr10.pfb></
usr/share/texmf-texlive/fonts/type1/bluesky/cm/cmsy10.pfb></usr/share/texmf-tex
live/fonts/type1/bluesky/cm/cmtt9.pfb></usr/share/texmf-texlive/fonts/type1/pub
lic/mathpazo/fplmr.pfb></usr/share/texmf-texlive/fonts/type1/public/mathpazo/fp
lmri.pfb></usr/share/texmf-texlive/fonts/type1/public/fpl/fplrc8a.pfb></usr/sha
re/texmf-texlive/fonts/type1/bluesky/ams/msam10.pfb></usr/share/texmf-texlive/f
onts/type1/urw/palatino/uplr8a.pfb></usr/share/texmf-texlive/fonts/type1/urw/pa
latino/uplri8a.pfb>
Output written on 31076-t.pdf (147 pages, 597566 bytes).
PDF statistics:
 1836 PDF objects out of 2073 (max. 8388607)
 674 named destinations out of 1000 (max. 131072)
 245 words of extra memory for PDF output out of 10000 (max. 10000000)

